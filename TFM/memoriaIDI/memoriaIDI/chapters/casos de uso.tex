\subsection{Diagrama de Casos de Uso}
\begin{figure}[H]
\centering
\includegraphics[angle=90, width=0.85\columnwidth]{doc/DiagramaCasosdeUso.eps}
\caption{Diagrama de Casos de Uso del sistema}\label{fig:diagrama_casos_de_uso}
\end{figure}

\subsection{Descripción Casos de Uso}
Los casos de uso son la descripción de una acción o actividad que puede ser realizada dentro del sistema. El objetivo de la identificación y descripción de los casos de uso es definir las posibles acciones más relevantes que forman parte del sistema y como este se debe comportar durante su realización, para ello especificaremos el flujo normal del caso de uso, la realización exitosa de una acción o actividad, y aquellos flujos alternativos ante imposibilidades de realización o fallos, tanto del sistema como de los actores, a la hora de realizar cada uno de los casos de uso.

En esta sección se describirán los casos de uso identificados en la Figura \ref{fig:diagrama_casos_de_uso}.
%REGISTRAR UN NUEVO TRABAJADOR
\begin{table}[H]
\begin{tabularx}{1\linewidth}{P}
\toprule
\textbf{Caso de uso:} &  Registrar un nuevo trabajador\\ \midrule
\textbf{Descripción:} & El caso de uso permitirá registrar un nuevo trabajador en el sistema.\\ \hline
\textbf{Actores:} & Administradores.\\ \hline
\textbf{Pre-condiciones:} & El trabajador no puede estar registrado. El administrador tiene sesión iniciada en el sistema.\\ \hline
\textbf{Post-condiciones:} & El sistema reconoce al nuevo trabajador permitiéndole realizar los casos de uso que figure como actor. \\ \midrule
\multicolumn{2}{c}{\textbf{Secuencia normal}}\\ \midrule
\multicolumn{2}{Z}{\hspace{0.5cm}\textbf{1.} El caso de uso comienza cuando el administrador indica que desea añadir un nuevo trabajador en el sistema.}\\ 
\multicolumn{2}{Z}{\hspace{0.5cm}\textbf{2.} El sistema muestra y solicita la información necesaria para añadir los datos del nuevo trabajador.}\\ 
\multicolumn{2}{Z}{\hspace{0.5cm}\textbf{3.} El administrador rellena \textcolor{red}{los datos solicitados} por el sistema.}\\ 
\multicolumn{2}{Z}{\hspace{0.5cm}\textbf{4.} Mientras el trabajador introduce los datos solicitados por el sistema:}\\ 
\multicolumn{2}{Z}{\hspace{1cm}\textbf{4.1.} El sistema comprueba que los datos son válidos.}\\ 
\multicolumn{2}{Z}{\hspace{.5cm}\textbf{5.} El administrador solicita dar de alta al nuevo trabajador.}\\ 
\multicolumn{2}{Z}{\hspace{.5cm}\textbf{6.} El sistema registra el nuevo trabajador y el caso de uso finaliza.}\\\midrule
\multicolumn{2}{c}{\textbf{Excepciones}}\\ \midrule
\multicolumn{2}{Z}{\hspace{0.5cm}\textbf{2.a)} Si el sistema no puede realizar la acción muestra un mensaje de error y el caso de uso finaliza. }\\ 
\multicolumn{2}{Z}{\hspace{0.5cm}\textbf{4.1.a)} Si los datos no son correctos el sistema muestra un aviso al administrador imposibilitando el registro del nuevo trabajador. }\\ 
\multicolumn{2}{Z}{\hspace{0.5cm}\textbf{3.a)}, \textbf{5.a)} El usuario solicita finalizar el caso de uso, el caso de uso finaliza sin efectos.}\\ \hline
\bottomrule
\end{tabularx}
\caption{Descripción del caso de uso ``Registrar un nuevo trabajador''} \label{tab:caso_uso_registrar_trabajador}
\end{table}

%INICIAR SESIÓN
\begin{table}[H]
\begin{tabularx}{1\linewidth}{P}
\toprule
\textbf{Caso de uso:} &  Iniciar sesión en el sistema\\ \midrule
\textbf{Descripción:} & El caso de uso permitirá iniciar sesión para poder acceder al sistema.\\ \hline
\textbf{Actores:} & Administradores, trabajadores.\\ \hline
\textbf{Pre-condiciones:} & El trabajador o administrador deben estar registrados pero sin iniciar sesión.\\ \hline
\textbf{Post-condiciones:} & El sistema da acceso a los casos de uso disponibles para administradores y trabajadores con sesión iniciada. \\ \midrule
\multicolumn{2}{c}{\textbf{Secuencia normal}}\\ \midrule
\multicolumn{2}{Z}{\hspace{0.5cm}\textbf{1.} El caso de uso comienza cuando un actor
indica que quiere iniciar sesión en el sistema.}\\ 
\multicolumn{2}{Z}{\hspace{.5cm}\textbf{2.} El sistema muestra y solicita la información necesaria para iniciar sesión.}\\ 
\multicolumn{2}{Z}{\hspace{0.5cm}\textbf{3.} El actor rellena \textcolor{red}{los datos solicitados} por el sistema y solicita iniciar sesión.}\\
\multicolumn{2}{Z}{\hspace{0.5cm}\textbf{4.} El sistema comprueba los datos del actor, da acceso al actor al sistema y el caso de uso finaliza.}\\ \midrule
\multicolumn{2}{c}{\textbf{Excepciones}}\\ \midrule
\multicolumn{2}{Z}{\hspace{0.5cm}\textbf{2.a)} Si el sistema no puede realizar la acción muestra un mensaje de error y el caso de uso finaliza. }\\ 
\multicolumn{2}{Z}{\hspace{0.5cm}\textbf{4.a)} Si los datos no son correctos el sistema muestra un aviso al actor imposibilitando el inicio de sesión, el caso de uso vuelve al paso 2. }\\ 
\multicolumn{2}{Z}{\hspace{0.5cm}\textbf{3.a)} El usuario solicita finalizar el caso de uso, el caso de uso finaliza sin efectos.}\\ \hline
\bottomrule
\end{tabularx}
\caption{Descripción del caso de uso ``Iniciar sesión en el sistema''} \label{tab:caso_uso_iniciar_sesion}
\end{table}

%REGISTRAR PACIENTE
\begin{table}[H]
\begin{tabularx}{1\linewidth}{P}
\toprule
\textbf{Caso de uso:} &  Registrar un paciente.\\ \midrule
\textbf{Descripción:} & El caso de uso permitirá registrar un paciente nuevo en el sistema.\\ \hline
\textbf{Actores:} & Trabajadores.\\ \hline
\textbf{Pre-condiciones:} & El actor debe tener sesión iniciada en el sistema.\\ \hline
\textbf{Post-condiciones:} & El sistema queda en estado consistente con el nuevo paciente registrado en la base de datos. \\ \midrule
\multicolumn{2}{c}{\textbf{Secuencia normal}}\\ \midrule
\multicolumn{2}{Z}{\hspace{0.5cm}\textbf{1.} El caso de uso comienza cuando un actor
indica que quiere registrar un nuevo paciente.}\\ 
\multicolumn{2}{Z}{\hspace{.5cm}\textbf{2.} El sistema muestra y solicita la información necesaria para registrar un nuevo paciente.}\\ 
\multicolumn{2}{Z}{\hspace{0.5cm}\textbf{3.} El actor rellena \textcolor{red}{los datos solicitados} por el sistema y solicita registrar al nuevo paciente.}\\
\multicolumn{2}{Z}{\hspace{0.5cm}\textbf{4.} El sistema comprueba los datos introducidos por el actor, registra el nuevo paciente en el sistema y el caso de uso finaliza.}\\ \midrule
\multicolumn{2}{c}{\textbf{Excepciones}}\\ \midrule
\multicolumn{2}{Z}{\hspace{0.5cm}\textbf{2.a)} Si el sistema no puede realizar la acción muestra un mensaje de error y el caso de uso finaliza. }\\ 
\multicolumn{2}{Z}{\hspace{0.5cm}\textbf{4.a)} Si los datos no son correctos el sistema muestra un aviso al actor imposibilitando el registro del nuevo paciente, el caso de uso vuelve al paso 2. }\\ 
\multicolumn{2}{Z}{\hspace{0.5cm}\textbf{3.a)} El usuario solicita finalizar el caso de uso, el caso de uso finaliza sin efectos.}\\ \hline
\bottomrule
\end{tabularx}
\caption{Descripción del caso de uso ``Registrar un nuevo paciente''} \label{tab:caso_uso_registrar_nuevo_paciente}
\end{table}

%VER DATOS DE UN PACIENTE
\begin{table}[H]
\begin{tabularx}{1\linewidth}{P}
\toprule
\textbf{Caso de uso:} &  Ver datos de un paciente\\ \midrule
\textbf{Descripción:} & El caso de uso permitirá ver los datos de un paciente registrado en el sistema.\\ \hline
\textbf{Actores:} & Trabajadores.\\ \hline
\textbf{Pre-condiciones:} & El actor debe tener sesión iniciada en el sistema.\\ \hline
\textbf{Post-condiciones:} & El sistema da acceso a los datos del paciente solicitado por el actor. \\ \hline
\multirow{2}{*}{\textbf{Extend:}} & \ref{tab:caso_uso_modificar_datos_paciente} \nameref{tab:caso_uso_modificar_datos_paciente} \\& \ref{tab:caso_uso_borrar_paciente} \nameref{tab:caso_uso_borrar_paciente}  \\\midrule
\multicolumn{2}{c}{\textbf{Secuencia normal}}\\ \midrule
\multicolumn{2}{Z}{\hspace{0.5cm}\textbf{1.} El caso de uso comienza cuando un actor indica que quiere ver datos de un paciente registrado.}\\
\multicolumn{2}{Z}{\hspace{.5cm}\textbf{2.} El sistema muestra y solicita la información necesaria para encontrar al paciente buscado.}\\
\multicolumn{2}{Z}{\hspace{0.5cm}\textbf{3.} El actor rellena \textcolor{red}{los datos solicitados} por el sistema y solicita ver datos del paciente.}\\
\multicolumn{2}{Z}{\hspace{0.5cm}\textbf{4.} El sistema comprueba los datos del actor, muestra los datos del paciente indicado y el caso de uso finaliza.}\\ \midrule
\multicolumn{2}{c}{\textbf{Excepciones}}\\ \midrule
\multicolumn{2}{Z}{\hspace{0.5cm}\textbf{2.a)} Si el sistema no puede realizar la acción muestra un mensaje de error y el caso de uso finaliza. }\\
\multicolumn{2}{Z}{\hspace{0.5cm}\textbf{4.a)} Si los datos no son correctos el sistema muestra un aviso al actor imposibilitando ver los datos del paciente, el caso de uso vuelve al paso 2. }\\
\multicolumn{2}{Z}{\hspace{0.5cm}\textbf{3.a)} El usuario solicita finalizar el caso de uso, el caso de uso finaliza sin efectos.}\\ \hline
\bottomrule
\relax
\end{tabularx}
\caption{Descripción del caso de uso ``Ver datos de un paciente''} \label{tab:caso_uso_ver_datos_paciente}
\end{table}

%MODIFICAR DATOS DE UN PACIENTE
\begin{table}[H]
\begin{tabularx}{1\linewidth}{P}
\toprule
\textbf{Caso de uso:} &  Modificar datos de un paciente\\ \midrule
\textbf{Descripción:} & El caso de uso permitirá modificar los datos de un paciente registrado en el sistema.\\ \hline
\textbf{Actores:} & Trabajadores.\\ \hline
\textbf{Pre-condiciones:} & El actor debe tener sesión iniciada en el sistema.\\ \hline
\textbf{Post-condiciones:} & El sistema da acceso a los datos del paciente y permite su modificación y guardado de los cambios realizados.\\\midrule
\multicolumn{2}{c}{\textbf{Secuencia normal}}\\ \midrule
\multicolumn{2}{Z}{\hspace{.5cm}\textbf{1.} El caso de uso comienza cuando el actor solicita modificar los datos del paciente.}\\ 
\multicolumn{2}{Z}{\hspace{.5cm}\textbf{2.} El sistema permite modificar los datos del paciente registrado.}\\ 
\multicolumn{2}{Z}{\hspace{0.5cm}\textbf{3.} El actor modifica los datos del paciente y solicita guardar los cambios realizados.}\\
\multicolumn{2}{Z}{\hspace{0.5cm}\textbf{4.} El sistema comprueba los datos modificados por el actor y el caso de uso finaliza.}\\ \midrule
\multicolumn{2}{c}{\textbf{Excepciones}}\\ \midrule
\multicolumn{2}{Z}{\hspace{0.5cm}\textbf{2.a)} Si el sistema no puede realizar la acción muestra un mensaje de error y el caso de uso finaliza. }\\ 
\multicolumn{2}{Z}{\hspace{0.5cm}\textbf{4.a)} Si los datos no son correctos el sistema muestra un aviso al actor imposibilitando el guardado, el caso de uso vuelve al paso 3. }\\ 
\multicolumn{2}{Z}{\hspace{0.5cm}\textbf{3.a)} El usuario solicita finalizar el caso de uso, el caso de uso finaliza sin efectos.}\\ \hline
\bottomrule
\end{tabularx}
\caption{Descripción del caso de uso ``Modificar datos de un paciente''} \label{tab:caso_uso_modificar_datos_paciente}
\end{table}

%BORRAR UN PACIENTE
\begin{table}[H]
\begin{tabularx}{1\linewidth}{P}
\toprule
\textbf{Caso de uso:} &  Borrar un paciente\\ \midrule
\textbf{Descripción:} & El caso de uso permitirá borrar los datos de un paciente registrado en el sistema.\\ \hline
\textbf{Actores:} & Trabajadores.\\ \hline
\textbf{Pre-condiciones:} & El actor debe tener sesión iniciada en el sistema.\\ \hline
\textbf{Post-condiciones:} & El sistema da acceso a los datos del paciente y permite su borrado del paciente.\\ \midrule
\multicolumn{2}{c}{\textbf{Secuencia normal}}\\ \midrule
\multicolumn{2}{Z}{\hspace{.5cm}\textbf{1.} El caso de uso comienza cuando el actor solicita borrar los datos del paciente.}\\ 
\multicolumn{2}{Z}{\hspace{.5cm}\textbf{2.} El sistema advierte de la acción y solicita confirmación.}\\ 
\multicolumn{2}{Z}{\hspace{0.5cm}\textbf{3.} El actor confirma que quiere borrar los datos del paciente.}\\ 
\multicolumn{2}{Z}{\hspace{0.5cm}\textbf{4.} El sistema borra los datos del paciente y el caso de uso finaliza.}\\ \midrule
\multicolumn{2}{c}{\textbf{Excepciones}}\\ \midrule
\multicolumn{2}{Z}{\hspace{0.5cm}\textbf{2.a)} Si el sistema no puede realizar la acción muestra un mensaje de error y el caso de uso finaliza. }\\ 
\multicolumn{2}{Z}{\hspace{0.5cm}\textbf{3.a)} El usuario solicita finalizar el caso de uso, el caso de uso finaliza sin efectos.}\\ \hline
\bottomrule
\end{tabularx}
\caption{Descripción del caso de uso ``Borrar un paciente''} \label{tab:caso_uso_borrar_paciente}
\end{table}

%REALIZAR UNA PRUEBA
\begin{table}[H]
\begin{tabularx}{1\linewidth}{P}
\toprule
\textbf{Caso de uso:} &  Realizar una prueba\\ \midrule
\textbf{Descripción:} & El caso de uso permitirá realizar una prueba sobre un ensayo a un paciente registrado en el sistema.\\ \hline
\textbf{Actores:} & Trabajadores.\\ \hline
\textbf{Pre-condiciones:} & El actor debe tener sesión iniciada en el sistema.\\ \hline
\textbf{Post-condiciones:} & El sistema almacena los datos recogidos por la prueba en el sistema. \\ \midrule
\multicolumn{2}{c}{\textbf{Secuencia normal}}\\ \midrule
\multicolumn{2}{Z}{\hspace{0.5cm}\textbf{1.} El caso de uso comienza cuando un actor indica que quiere realizar una prueba nueva.}\\ 
\multicolumn{2}{Z}{\hspace{.5cm}\textbf{2.} El sistema muestra y solicita el tipo de ensayo sobre el que se va a realizar la prueba.}\\
\multicolumn{2}{Z}{\hspace{0.5cm}\textbf{3.} El actor selecciona el ensayo en el que se basa la prueba.}\\ 
\multicolumn{2}{Z}{\hspace{0.5cm}\textbf{4.} El sistema solicita al actor confirmación para iniciar la prueba sobre el paciente.}\\ 
\multicolumn{2}{Z}{\hspace{0.5cm}\textbf{5.} El actor confirma el inicio de la prueba.}\\ 
\multicolumn{2}{Z}{\hspace{0.5cm}\textbf{6.} El sistema comienza la recogida de datos de la prueba sobre el ensayo elegido.}\\\midrule
\multicolumn{2}{c}{\textbf{Excepciones}}\\ \midrule
\multicolumn{2}{Z}{\hspace{0.5cm}\textbf{2.a)} Si el sistema no puede realizar la acción muestra un mensaje de error y el caso de uso finaliza. }\\ 
\multicolumn{2}{Z}{\hspace{0.5cm}\textbf{4.a)} Si los datos no son correctos el sistema muestra un aviso al actor imposibilitando ver los datos del paciente, el caso de uso vuelve al paso 2. }\\ 
\multicolumn{2}{Z}{\hspace{0.5cm}\textbf{3.a), 5.a)} El usuario solicita finalizar el caso de uso, el caso de uso finaliza sin efectos.}\\ 
\multicolumn{2}{Z}{\hspace{0.5cm}\textbf{3.b)} El usuario desea crear un ensayo nuevo. Se inicia caso de uso: \ref{tab:caso_uso_crear_nuevo_ensayo}. \nameref{tab:caso_uso_crear_nuevo_ensayo}}\\ \hline
\bottomrule
\end{tabularx}
\caption{Descripción del caso de uso ``Realizar una prueba''} \label{tab:caso_uso_realizar_prueba}
\end{table}

%CONSULTAR UNA PRUEBA
\begin{table}[H]
\begin{tabularx}{1\linewidth}{P}
\toprule
\textbf{Caso de uso:} &  Consultar una prueba\\ \midrule
\textbf{Descripción:} & El caso de uso permitirá ver los datos de una prueba sobre paciente registrado en el sistema.\\ \hline
\textbf{Actores:} & Trabajadores.\\ \hline
\textbf{Pre-condiciones:} & El actor debe tener sesión iniciada en el sistema.\\ \hline
\textbf{Post-condiciones:} & El sistema da acceso a los datos de la prueba sobre el paciente solicitado por el actor. \\ \hline
\multirow{2}{*}{\textbf{Extend:}} & \ref{tab:caso_uso_modificar_datos_prueba} \nameref{tab:caso_uso_modificar_datos_prueba} \\& \ref{tab:caso_uso_borrar_prueba} \nameref{tab:caso_uso_borrar_prueba}  \\\midrule
\multicolumn{2}{c}{\textbf{Secuencia normal}}\\ \midrule
\multicolumn{2}{Z}{\hspace{0.5cm}\textbf{1.} El caso de uso comienza cuando un actor indica que quiere ver datos de una prueba registrada.}\\ 
\multicolumn{2}{Z}{\hspace{.5cm}\textbf{2.} El sistema muestra y solicita la información necesaria para encontrar la prueba buscada.}\\
\multicolumn{2}{Z}{\hspace{0.5cm}\textbf{3.} El actor rellena \textcolor{red}{los datos solicitados} por el sistema y solicita ver datos de la prueba.}\\
\multicolumn{2}{Z}{\hspace{0.5cm}\textbf{4.} El sistema comprueba los datos del actor, da acceso al actor a los datos de la prueba indicada y el caso de uso finaliza.}\\ \midrule
\multicolumn{2}{c}{\textbf{Excepciones}}\\ \midrule
\multicolumn{2}{Z}{\hspace{0.5cm}\textbf{2.a)} Si el sistema no puede realizar la acción muestra un mensaje de error y el caso de uso finaliza. }\\
\multicolumn{2}{Z}{\hspace{0.5cm}\textbf{4.a)} Si los datos no son correctos el sistema muestra un aviso al actor imposibilitando ver los datos de la prueba, el caso de uso vuelve al paso 2. }\\
\multicolumn{2}{Z}{\hspace{0.5cm}\textbf{3.a)} El usuario solicita finalizar el caso de uso, el caso de uso finaliza sin efectos.}\\ \hline
\bottomrule
\end{tabularx}
\caption{Descripción del caso de uso ``Ver datos de una prueba''} \label{tab:caso_uso_ver_datos_prueba}
\end{table}

%MODIFICAR UNA PRUEBA
\begin{table}[H]
\begin{tabularx}{1\linewidth}{P}
\toprule
\textbf{Caso de uso:} &  Modificar datos de una prueba\\ \midrule
\textbf{Descripción:} & El caso de uso permitirá modificar los datos de una prueba registrada en el sistema.\\ \hline
\textbf{Actores:} & Trabajadores.\\ \hline
\textbf{Pre-condiciones:} & El actor debe tener sesión iniciada en el sistema.\\ \hline
\textbf{Post-condiciones:} & El sistema da acceso a los datos de la prueba, permite la modificación y guardado de los cambios realizados.\\\midrule
\multicolumn{2}{c}{\textbf{Secuencia normal}}\\ \midrule
\multicolumn{2}{Z}{\hspace{.5cm}\textbf{1.} El caso de uso comienza cuando el actor solicita modificar los datos de la prueba.}\\ 
\multicolumn{2}{Z}{\hspace{.5cm}\textbf{2.} El sistema permite modificar los datos de la prueba registrada.}\\ 
\multicolumn{2}{Z}{\hspace{0.5cm}\textbf{3.} El actor modifica los datos de la prueba y solicita guardar los cambios realizados.}\\ 
\multicolumn{2}{Z}{\hspace{0.5cm}\textbf{4.} El sistema comprueba los datos modificados por el actor y el caso de uso finaliza.}\\ \midrule
\multicolumn{2}{c}{\textbf{Excepciones}}\\ \midrule
\multicolumn{2}{Z}{\hspace{0.5cm}\textbf{2.a)} Si el sistema no puede realizar la acción muestra un mensaje de error y el caso de uso finaliza. }\\
\multicolumn{2}{Z}{\hspace{0.5cm}\textbf{4.a)} Si los datos no son correctos el sistema muestra un aviso al actor imposibilitando el guardado, el caso de uso vuelve al paso 3. }\\ 
\multicolumn{2}{Z}{\hspace{0.5cm}\textbf{3.a)} El usuario solicita finalizar el caso de uso, el caso de uso finaliza sin efectos.}\\ \hline
\bottomrule
\end{tabularx}
\caption{Descripción del caso de uso ``Modificar datos de una prueba''} \label{tab:caso_uso_modificar_datos_prueba}
\end{table}

%BORRAR UNA PRUEBA
\begin{table}[H]
\begin{tabularx}{1\linewidth}{P}
\toprule
\textbf{Caso de uso:} &  Borrar una prueba\\ \midrule
\textbf{Descripción:} & El caso de uso permitirá borrar una prueba registrada en el sistema.\\ \hline
\textbf{Actores:} & Trabajadores.\\ \hline
\textbf{Pre-condiciones:} & El actor debe tener sesión iniciada en el sistema.\\ \hline
\textbf{Post-condiciones:} & El sistema da acceso a la prueba y permite su borrado. \\\midrule
\multicolumn{2}{c}{\textbf{Secuencia normal}}\\ \midrule
\multicolumn{2}{Z}{\hspace{.5cm}\textbf{1.} El caso de uso comienza cuando el actor solicita borrar la prueba.}\\ 
\multicolumn{2}{Z}{\hspace{.5cm}\textbf{2.} El sistema advierte de la acción y solicita confirmación.}\\ 
\multicolumn{2}{Z}{\hspace{0.5cm}\textbf{3.} El actor confirma que quiere borrar la prueba.}\\ 
\multicolumn{2}{Z}{\hspace{0.5cm}\textbf{4.} El sistema borra la prueba y el caso de uso finaliza.}\\ \midrule
\multicolumn{2}{c}{\textbf{Excepciones}}\\ \midrule
\multicolumn{2}{Z}{\hspace{0.5cm}\textbf{2.a)} Si el sistema no puede realizar la acción muestra un mensaje de error y el caso de uso finaliza. }\\ 
\multicolumn{2}{Z}{\hspace{0.5cm}\textbf{3.a)} El usuario solicita finalizar el caso de uso, el caso de uso finaliza sin efectos.}\\ \hline
\bottomrule
\end{tabularx}
\caption{Descripción del caso de uso ``Borrar una prueba''} \label{tab:caso_uso_borrar_prueba}
\end{table}

%CREAR UN ENSAYO
\begin{table}[H]
\begin{tabularx}{1\linewidth}{P}
\toprule
\textbf{Caso de uso:} &  Crear un nuevo ensayo.\\ \midrule
\textbf{Descripción:} & El caso de uso permitirá crear un ensayo nuevo en el sistema.\\ \hline
\textbf{Actores:} & Trabajadores.\\ \hline
\textbf{Pre-condiciones:} & El actor debe tener sesión iniciada en el sistema.\\ \hline
\textbf{Post-condiciones:} & El sistema queda en estado consistente con el nuevo ensayo creado en la base de datos y disponible para la realización de prueba en base a él. \\ \midrule
\multicolumn{2}{c}{\textbf{Secuencia normal}}\\ \midrule
\multicolumn{2}{Z}{\hspace{0.5cm}\textbf{1.} El caso de uso comienza cuando un actor indica que quiere crear un nuevo ensayo.}\\
\multicolumn{2}{Z}{\hspace{.5cm}\textbf{2.} El sistema muestra y solicita la información necesaria para crear un nuevo ensayo.}\\
\multicolumn{2}{Z}{\hspace{0.5cm}\textbf{3.} El actor rellena \textcolor{red}{los datos solicitados} por el sistema y solicita crear el nuevo ensayo.}\\
\multicolumn{2}{Z}{\hspace{0.5cm}\textbf{4.} El sistema comprueba los datos introducidos por el actor, crea el nuevo ensayo en el sistema y el caso de uso finaliza.}\\ \midrule
\multicolumn{2}{c}{\textbf{Excepciones}}\\ \midrule
\multicolumn{2}{Z}{\hspace{0.5cm}\textbf{2.a)} Si el sistema no puede realizar la acción muestra un mensaje de error y el caso de uso finaliza. }\\
\multicolumn{2}{Z}{\hspace{0.5cm}\textbf{4.a)} Si los datos no son correctos el sistema muestra un aviso al actor imposibilitando la creación del nuevo ensayo, el caso de uso vuelve al paso 2. }\\
\multicolumn{2}{Z}{\hspace{0.5cm}\textbf{3.a)} El usuario solicita finalizar el caso de uso, el caso de uso finaliza sin efectos.}\\ \hline
\bottomrule
\end{tabularx}
\caption{Descripción del caso de uso ``Crear un nuevo ensayo''} \label{tab:caso_uso_crear_nuevo_ensayo}
\end{table}

%AÑADIR PATOLOGÍAS
\begin{table}[H]\centering
\begin{tabularx}{1\linewidth}{P}
\toprule
\textbf{Caso de uso:} &  Añadir patología.\\ \midrule
\textbf{Descripción:} & El caso de uso permitirá añadir nuevas patologías en el sistema.\\ \hline
\textbf{Actores:} & Trabajadores.\\ \hline
\textbf{Pre-condiciones:} & El actor debe tener sesión iniciada en el sistema.\\ \hline
\textbf{Post-condiciones:} & El sistema queda en estado consistente con la nueva patología en la base de datos y la información asociada al ella. \\ \midrule
\multicolumn{2}{c}{\textbf{Secuencia normal}}\\ \midrule
\multicolumn{2}{Z}{\hspace{0.5cm}\textbf{1.} El caso de uso comienza cuando un actor indica que quiere añadir una patología nueva.}\\
\multicolumn{2}{Z}{\hspace{0.5cm}\textbf{2.} El sistema muestra y solicita la información necesaria para crear la nueva patología.}\\
\multicolumn{2}{Z}{\hspace{0.5cm}\textbf{3.} El actor rellena \textcolor{red}{los datos solicitados} por el sistema y solicita crear la nueva patología.}\\
\multicolumn{2}{Z}{\hspace{0.5cm}\textbf{4.} El sistema comprueba los datos introducidos por el actor, crea la nueva patología en el sistema y el caso de uso finaliza.}\\
\midrule
\multicolumn{2}{c}{\textbf{Excepciones}}\\ \midrule
\multicolumn{2}{Z}{\hspace{0.5cm}\textbf{2.a)} Si el sistema no puede realizar la acción muestra un mensaje de error y el caso de uso finaliza. }\\
\multicolumn{2}{Z}{\hspace{0.5cm}\textbf{4.a)} Si los datos no son correctos el sistema muestra un aviso al actor imposibilitando la creación del nuevo ensayo, el caso de uso vuelve al paso 2. }\\
\multicolumn{2}{Z}{\hspace{0.5cm}\textbf{3.a)} El usuario solicita finalizar el caso de uso, el caso de uso finaliza sin efectos.}\\ 
\hline
\bottomrule
\end{tabularx}
\caption{Descripción del caso de uso ``Añadir patología''} \label{tab:caso_uso_añadir_patologias}
\end{table}

%MODIFICAR PATOLOGÍAS
\begin{table}[H]\centering
\begin{tabularx}{1\linewidth}{P}
\toprule
\textbf{Caso de uso:} &  Modificar patología.\\ \midrule
\textbf{Descripción:} & El caso de uso permitirá modificar patologías existentes en el sistema.\\ \hline
\textbf{Actores:} & Trabajadores.\\ \hline
\textbf{Pre-condiciones:} & El actor debe tener sesión iniciada en el sistema.\\ \hline
\textbf{Post-condiciones:} & El sistema queda en estado consistente con la patología modificada en la base de datos y la información asociada al ella. \\ \midrule
\multicolumn{2}{c}{\textbf{Secuencia normal}}\\ \midrule
\multicolumn{2}{Z}{\hspace{0.5cm}\textbf{1.} El caso de uso comienza cuando un actor indica que quiere modificar una patología.}\\
\multicolumn{2}{Z}{\hspace{0.5cm}\textbf{2.} El sistema muestra la información asociada a la patología permitiendo realizar cambios.}\\
\multicolumn{2}{Z}{\hspace{0.5cm}\textbf{3.} El actor realiza los cambios sobre los datos de la patología y solicita guardar estos cambios.}\\
\multicolumn{2}{Z}{\hspace{0.5cm}\textbf{4.} El sistema comprueba los datos introducidos por el actor, modifica la patología en el sistema y el caso de uso finaliza.}\\
\midrule
\multicolumn{2}{c}{\textbf{Excepciones}}\\ \midrule
\multicolumn{2}{Z}{\hspace{0.5cm}\textbf{2.a)} Si el sistema no puede realizar la acción muestra un mensaje de error y el caso de uso finaliza. }\\
\multicolumn{2}{Z}{\hspace{0.5cm}\textbf{4.a)} Si los datos no son correctos el sistema muestra un aviso al actor imposibilitando la modificación del nuevo ensayo, el caso de uso vuelve al paso 2. }\\
\multicolumn{2}{Z}{\hspace{0.5cm}\textbf{3.a)} El usuario solicita finalizar el caso de uso, el caso de uso finaliza sin efectos.}\\ 
\hline
\bottomrule
\end{tabularx}
\caption{Descripción del caso de uso ``Modificar patología''} \label{tab:caso_uso_modificar_patologias}
\end{table}

%BORRAR PATOLOGÍAS
\begin{table}[H]\centering
\begin{tabularx}{1\linewidth}{P}
\toprule
\textbf{Caso de uso:} &  Borrar patología.\\ \midrule
\textbf{Descripción:} & El caso de uso permitirá borrar patologías existentes en el sistema.\\ \hline
\textbf{Actores:} & Trabajadores.\\ \hline
\textbf{Pre-condiciones:} & El actor debe tener sesión iniciada en el sistema.\\ \hline
\textbf{Post-condiciones:} & El sistema queda en estado consistente con la patología borrada en la base de datos y la información asociada al ella. \\ \midrule
\multicolumn{2}{c}{\textbf{Secuencia normal}}\\ \midrule
\multicolumn{2}{Z}{\hspace{0.5cm}\textbf{1.} El caso de uso comienza cuando un actor indica que quiere borrar una patología.}\\
\multicolumn{2}{Z}{\hspace{0.5cm}\textbf{2.} El sistema muestra la información asociada a la patología avisando que serán los datos borrados y solicita confirmación de la acción.}\\
\multicolumn{2}{Z}{\hspace{0.5cm}\textbf{3.} El actor confirma la acción de borrado.}\\
\multicolumn{2}{Z}{\hspace{0.5cm}\textbf{4.} El sistema borra la patología, la información y relaciones asociadas a ella, y el caso de uso finaliza.}\\
\midrule
\multicolumn{2}{c}{\textbf{Excepciones}}\\ \midrule
\multicolumn{2}{Z}{\hspace{0.5cm}\textbf{2.a), 4.a)} Si el sistema no puede realizar la acción muestra un mensaje de error y el caso de uso finaliza. }\\
\multicolumn{2}{Z}{\hspace{0.5cm}\textbf{3.a)} El usuario solicita finalizar el caso de uso, el caso de uso finaliza sin efectos.}\\ 
\hline
\bottomrule
\end{tabularx}
\caption{Descripción del caso de uso ``Borrar patología''} \label{tab:caso_uso_borrar_patologias}
\end{table}


%POBLAR BBDD
