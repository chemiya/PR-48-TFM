\section{Introducción}
En este capítulo se exponen tanto el Modelo del Dominio como el Modelo de Análisis del proyecto, ambos acompañados de sus correspondientes diagramas. Finalmente, se detalla la Realización de Casos de Análisis mediante Diagramas de Secuencia para los principales Casos de Uso del sistema. 

\section{Modelo del Dominio}
El Modelo de Dominio \cite{modelo-dominio} es un modelo conceptual que tiene como objetivo representar el vocabulario y los conceptos más importantes del dominio del problema. En él se describen las entidades, sus atributos, sus papeles y relaciones junto a las restricciones que rigen el dominio del sistema. 

\subsection{Diagrama de Clases del Modelo del Dominio}
\label{sec:clases}
En la Figura \ref{fig:diagrama-clases}  se muestra el Diagrama de Clases del Modelo de Dominio.

\begin{figure}
    \centering
    \begin{normalsize}
        \import{svg/}{clases-nuevo.pdf_tex}
    \end{normalsize}
    \caption{Diagrama de Clases del Modelo de Dominio}
     \label{fig:diagrama-clases}
 
 \end{figure}

\section{Modelo de Análisis}
El Modelo de Análisis \cite{modelo-analisis} se utiliza para describir la estructura de la aplicación o el sistema que se está creando. Incluye el Diagrama del Modelo de Dominio con los métodos asociados a cada clase junto a los diagramas de la Realización de Casos de Uso de Análisis que pueden ser Diagramas de Secuencia. Por lo tanto extiende el Modelo de Dominio y muestra el comportamiento del sistema.

\subsection{Clases de Análisis}
En la Figura \ref{fig:diagrama-clases-analisis}  se muestra el Diagrama de Clases de Análisis.

\begin{figure}
    \centering
    \begin{normalsize}
        \import{svg/}{analisis-nuevo.pdf_tex}
    \end{normalsize}
    \caption{Diagrama de Clases de Análisis}
     \label{fig:diagrama-clases-analisis}
 
 \end{figure}


\section{Realización de Casos de Uso de Análisis}
La Realización de Casos de Uso de Análisis sirve para describir cómo se realizará un Caso de Uso particular incluyendo los métodos, clases y actores que participarán en él. Para ello, se van a utilizar los Diagramas de Secuencia \cite{diagrama-secuencia} que muestran la interacción entre un conjunto de objetos a lo largo del tiempo en la aplicación. 

\subsection{Realización de Caso de Uso \textit{Registrarse}}
En la Figura \ref{fig:secuencia-registro}  se muestra el Diagrama de Secuencia del Caso de Uso \textit{Registrarse}.


\begin{figure}
    \centering
    \begin{normalsize}
        \import{svg/}{seq-registro-escalado.pdf_tex}
    \end{normalsize}
    \caption{Diagrama de Secuencia del Caso de Uso \textit{Registrarse}.}
     \label{fig:secuencia-registro}
 
 \end{figure}



\subsection{Realización de Caso de Uso \textit{Crear publicación}}
En la Figura \ref{fig:secuencia-crear-publicacion}  se muestra el Diagrama de Secuencia del Caso de Uso \textit{Crear publicación}.


\begin{figure}
    \centering
    \begin{normalsize}
        \import{svg/}{seq-crear-publicacion-escalado.pdf_tex}
    \end{normalsize}
    \caption{Diagrama de Secuencia del Caso de Uso \textit{Crear publicación}.}
     \label{fig:secuencia-crear-publicacion}
 
 \end{figure}



\subsection{Realización de Caso de Uso \textit{Crear receta}}
En la Figura \ref{fig:secuencia-crear-receta} se muestra el Diagrama de Secuencia del Caso de Uso \textit{Crear receta}.


\begin{figure}
    \centering
    \begin{normalsize}
        \import{svg/}{seq-crear-receta-escalado.pdf_tex}
    \end{normalsize}
    \caption{Diagrama de Secuencia del Caso de Uso \textit{Crear receta}.}
     \label{fig:secuencia-crear-receta}
 
 \end{figure}




\subsection{Realización de Caso de Uso \textit{Buscar alimento}}
En la Figura \ref{fig:secuencia-buscar-alimento}  se muestra el Diagrama de Secuencia del Caso de Uso \textit{Buscar alimento}.


\begin{figure}
    \centering
    \begin{normalsize}
        \import{svg/}{seq-buscar-alimento-escalado.pdf_tex}
    \end{normalsize}
    \caption{Diagrama de Secuencia del Caso de Uso \textit{Buscar alimento}.}
     \label{fig:secuencia-buscar-alimento}
 
 \end{figure}









