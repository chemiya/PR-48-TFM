\capitulo{2}{Objetivos del proyecto}

\section{Introducción}
En este capitulo se describen los objetivos que se pretenden conseguir con este proyecto, diferenciando entre objetivos de desarrollo y academicos.








\section{Objetivos de desarrollo}

\label{sec:desarrollo}
El principal objetivo de desarrollo es crear varios modelos con inteligencia artificial que ayuden a los entrenadores a tomar mejores decisiones sobre qué jugadores utilizar en un partido mediante los datos obtenidos en los partidos anteriores. Para ello, los principales objetivos de desarrollo para este proyecto son:
 \begin{enumerate}

    \item \textbf{Obtener los datos de los partidos de fútbol de varias ligas.}
    Para ello, mediante el \textit{scraping} se extraerán los datos de todos los partidos jugados en diferentes ligas y la información asociada a los jugadores.
    
    \item \textbf{Limpiar, transformar y analizar los datos obtenidos.}
    Se deberán limpiar y transformar los datos obtenidos para que puedan ser utilizados por los modelos que se pretenden crear. Además, se debe realizar un análisis previo sobre los datos para detectar posibles patrones.
    
    \item \textbf{Aplicar diferentes modelos de inteligencia artificial con los datos obtenidos.}
    En este punto, se debe evaluar el rendimiento que tienen los diferentes modelos sobre los datos obtenidos.

    \item \textbf{Optimizar el rendimiento de los modelos.}
    Después de entrenar los diferentes modelos sobre los datos, se debe realizar una optimización de sus parámetros para mejorar la precisión obtenida.

    \item \textbf{Seleccionar los mejores modelos y analizar su precisión obtenida.}
    Sobre todos los modelos evaluados, se deberán seleccionar los que mejor se comporten y se deberá de analizar que precisiones tienen.

    \item \textbf{Documentar los pasos seguidos en el proyecto.}
    Se deben documentar todos los pasos seguidos en el proyecto y justificar todas las decisiones tomadas incluyendo los objetivos, métodos y resultados de la investigación y desarrollo. Por otro lado se debe detallar la estructura, funcionamiento y uso del código.

 \end{enumerate}








 
\section{Objetivos académicos}
Estos objetivos se centran en seguir profundizando en los conocimientos aprendidos en diversas asignaturas de este máster relacionadas con el Deep Learning y el BigData y ponerlos en practica en un proyecto completo. A continuacion se detallan cada uno de estos objetivos:

\begin{enumerate}

    \item \textbf{Aplicar los conocimientos asociados a los pasos de extracción, transformación y carga de los datos.} Se pretende poner en práctica todos los conocimientos asociados al proceso de ETL seguido en los proyectos de BigData para estructurar los datos y que puedan ser utilizados por los modelos.
    \item \textbf{Aplicar las técnicas aprendidas sobre modelos de inteligencia artificial y redes neuronales.} Se pretende seguir profundizando y aplicar los conocimientos adquiridos sobre redes neuronales e inteligencia artificial para que los modelos creados tengan la mayor precisión posible.
    
 \end{enumerate}
