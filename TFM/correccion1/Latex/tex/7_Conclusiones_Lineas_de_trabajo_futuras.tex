\capitulo{7}{Conclusiones generales y líneas de trabajo futuras}

\section{Introducción}
En este capítulo se desarrollan las conclusiones del proyecto y se revisa si se han conseguido los objetivos que se plantearon al inicio. Finalmente, se proponen diferentes mejoras para realizar en el futuro para seguir desarrollando el proyecto.


\section{Conclusiones}
Las conclusiones de este proyecto destacan la trascendencia y el alcance que provoca la aplicación de diversas técnicas de inteligencia artificial en el contexto del fútbol, específicamente en la predicción del rendimiento de los equipos mediante el análisis de la rotación de jugadores. Este proyecto ha tenido la capacidad de revelar que la implementación de herramientas de inteligencia artificial, como modelos de aprendizaje automático y análisis de datos, pueden ayudar enormemente en la toma de decisiones a los entrenadores y directivos. Se ha podido observar cómo estas tecnologías pueden ofrecer información crucial que tengan una gran repercusión en la toma de decisiones estratégicas de entrenadores y directivos de equipos, permitiéndoles optimizar la rotación de jugadores de manera más precisa y efectiva y por tanto, mejorar su rendimiento.

Además, en este proyecto se destaca la importancia de disponer de conjuntos de datos completos y de calidad para proporcionar a estos modelos de inteligencia artificial de manera adecuada. La recopilación y preparación de datos precisos y relevantes sobre la rotación de jugadores y el rendimiento deportivo se ha establecido como un componente fundamental para el éxito de este proyecto como se ha podido observar. 

Este proyecto puede ayudar a destacar la necesidad de desarrollar herramientas y metodologías específicas que faciliten la integración de la inteligencia artificial en la gestión deportiva, lo que implicaría una colaboración conjunta entre expertos en deportes y científicos de datos y que, en muchos casos ya se está realizando.

Otro factor clave es que este proyecto ha pretendido mostrar el potencial de la inteligencia artificial para transformar y mejorar la gestión y el desempeño de los equipos de fútbol analizando los datos sobre la rotación de sus jugadores. Las conclusiones de este proyecto pretenden invitar a continuar investigando y desarrollando este campo, explorando nuevas tecnologías y metodologías que puedan maximizar el impacto positivo de la inteligencia artificial en el mundo del fútbol y así ayudar en todo lo posible a los directivos y entrenadores.

Previo a los modelos, el análisis previo de los datos ha revelado diferentes patrones que pueden 
ayudar enormemente a los entrenadores a tomar decisiones sobre los jugadores para aumentar el 
rendimiento del equipo, como evitar realizar muchos cambios en las alineaciones iniciales o evitar 
realizar muchos cambios antes del descanso, ya que ambos factores en este caso están 
estrechamente relacionados con los equipos de la zona baja de la clasificación. 

Por otro lado, realizar menos cambios en las alineaciones iniciales y menos cambios entre los 
minutos 61 y 75 se asocia a equipos que ocupan las posiciones más altas de la clasificación, y esto 
puede ser una buena señal de que estas estrategias ayudan al buen rendimiento del equipo.
En cuanto a las posiciones de los jugadores que son afectados por los cambios, realizar cambios 
ofensivos quitando defensas es una estrategia que se puede apreciar sobre todo en los equipos de 
la zona baja de la clasificación y por lo tanto no ayuda a su rendimiento. Por otra parte, realizar 
pocos cambios sacando delanteros e introduciendo jugadores más defensivos es una estrategia 
utilizada por los equipos de la zona alta de la clasificación y por lo tanto parece que contribuye a 
su éxito y buen rendimiento.

Sobre los minutos en los que se realizan los cambios, los equipos que de media realizan los 
cambios antes suelen estar en la zona alta de la clasificación y lo mismo es aplicable a los cambios 
que reemplazan jugadores amonestados con amarilla ya que los equipos con menos cambios de 
este tipo ocupan posiciones más altas.

Finalmente, otro dato bastante significativo y curioso es que los equipos de la zona alta de la 
clasificación tienen valores más bajos en el número de cambios por partido que hacen, es decir, 
no gastan todos los cambios de los que disponen. Esto puede contradecir la creencia generalizada 
de que al realizar más cambios el equipo debería de rendir más porque introduce jugadores 
totalmente frescos pero parece que, este análisis muestra lo contrario, que conviene mantener los 
jugadores que estén en el campo y no necesariamente gastar todos los cambios de los que 
disponen.

Por otro lado, como se ha podido apreciar no se ha obtenido un único modelo para predecir tanto el ganador del partido como el número de goles del local como el número de goles del visitante. Se han obtenido diferentes modelos en forma de redes neuronales, cada una con una arquitectura diferente optimizada para realizar predicciones sobre el ganador del partido, los goles de local o los goles del visitante. 

La arquitectura del modelo entrenado para predecir el ganador del partido consistía en una red neuronal con 3 capas, con descenso de gradiente
estocástico como optimizador, con entropía cruzada categórica como función de pérdida,
entrenada en 20 épocas con un tamaño de lote de 48 que no incorpora \textit{callbacks} que obtuvo una exactitud sobre el conjunto de prueba del 0,560.

La arquitectura del modelo entrenado para predecir los goles del local consistía en una red neuronal  con 5 capas, con descenso de gradiente
estocástico como optimizador, con entropía cruzada categórica como función de pérdida,
entrenada en 20 épocas con un tamaño de lote de 80 que no incorpora \textit{callbacks}. Esta red ha
obtenido una exactitud sobre el conjunto de prueba del 0,370.

La arquitectura del modelo entrenado para predecir los goles del visitante consistía una red neuronal,  con 4 capas, con ``adam'' como optimizador,
con entropía cruzada categórica como función de pérdida, entrenada en 20 épocas con un tamaño
de lote de 80 que no incorpora \textit{callbacks}. Esta red ha obtenido una exactitud sobre el conjunto
de prueba del 0,383.

Sin embargo, también se debe tener en cuenta que la naturaleza imprevisible del fútbol es uno de los mayores desafíos para cualquier modelo de inteligencia artificial entrenado para predecir sus resultados. El fútbol es un deporte en el que los resultados pueden variar considerablemente debido a factores aleatorios y circunstancias incontrolables. Eventos inesperados como lesiones de jugadores clave, decisiones arbitrales controvertidas, condiciones climáticas adversas o simplemente un mal día para un equipo o un jugador pueden influir determinantemente en el resultado final de un partido. Estas variables, imposibles de cuantificar y predecir, introducen un grado de aleatoriedad que dificulta la precisión de cualquier modelo predictivo.

Además, el rendimiento de un equipo de fútbol no solo depende de las estadísticas y datos históricos, sino que también esta influenciado por aspectos intangibles que no se pueden calibrar como la moral del equipo, la cohesión entre los jugadores y la estrategia del entrenador. Estos factores humanos y psicológicos son muy difíciles de cuantificar y aún más difíciles de incluir en un modelo de inteligencia artificial de manera correcta. Por ejemplo, un equipo puede tener un rendimiento muy alto en un partido debido a una motivación, como un derbi entre equipos locales o un partido decisivo en un torneo, factores que no se reflejan debidamente en los datos históricos y estadísticas habituales.

Por último, el fútbol es un juego de un número bajo de goles, lo que significa que pequeños errores en la predicción pueden tener un gran impacto en la precisión del modelo. A diferencia de otros deportes con puntuaciones más altas y continuas, en el fútbol, la diferencia entre un gol y ningún gol es capaz de decidir el resultado de un partido. Este margen estrecho de resultados hace que cualquier modelo tenga que ser extremadamente preciso para lograr una alta exactitud. Además, la dinámica táctica de los equipos, que suele variar de un partido a otro en función del rival, introduce otra variabilidad adicional que los modelos predictivos tradicionales habitualmente no son capaces de capturar completamente. En resumen, la combinación de factores aleatorios, humanos y la naturaleza del deporte en sí mismo hace que la predicción de resultados en el fútbol sea especialmente compleja, difícil y sujeta a una baja exactitud en los modelos de inteligencia artificial pero esto no impide que aún así, un correcto análisis de los datos y la aplicación de técnicas de inteligencia artificial, ayuden a tomar mejores decisiones a los entrenadores y directivos que permitan mejorar el rendimiento de los equipos.



\section{Revisión sobre la consecución de los objetivos}



Sobre los objetivos académicos, sí que se han cumplido los objetivos establecidos al iniciar el proyecto. 

En primer lugar, en el proyecto se han aplicado los conocimientos sobre \textit{Big Data} adquiridos en el Máster para optimizar las etapas del ciclo de vida de los datos, asegurando así su máxima utilidad y precisión. Inicialmente, se extrajeron los datos en bruto mediante diversos programas de \textit{scraping}. Posteriormente, se llevó a cabo un proceso de limpieza y preparación de datos para que los modelos se pudiesen entrenar con estos datos y para eliminar valores inconsistentes. A continuación, se realizó un análisis de los datos para detectar patrones y conclusiones que pudiesen ser útiles en el proyecto para así poder ayudar a los entrenadores y documentar cuáles son las mejores estrategias de rotación y qué estrategias aumentan el rendimiento de los equipos. Estas técnicas permitieron estandarizar los datos y aumentar su calidad. De esta manera, al aplicar los conocimientos sobre \textit{Big Data} no solo se consiguió una gestión más efectiva de los datos, sino que también proporcionaron conclusiones más útiles para la toma de decisiones estratégicas por parte de los entrenadores sobre como rotar a los jugadores.

En segundo lugar, se han aplicado los conocimientos sobre \textit{deep learning}, redes neuronales y \textit{machine learning} para desarrollar modelos que traten los problemas planteados en este proyecto y proporcionen resultados que ayuden a las personas a tomar mejores decisiones, en este caso a los entrenadores sobre como rotar a los jugadores. El uso de \textit{deep learning}, con redes neuronales profundas, permitió la creación de modelos mediante redes neuronales con diferentes parámetros capaces de aprender y generalizar a partir de grandes cantidades de datos, pudiendo identificar patrones y relaciones. Además al realizar la optimización de los parámetros y de la estructura de estas redes neuronales, se siguió profundizando en los conocimientos sobre ellas lo que conllevó un gran aprendizaje. Además, también se implementaron técnicas de \textit{machine learning} para entrenar diversos algoritmos, utilizando métodos como los árboles de decisión, máquinas de vectores de soporte y bosques aleatorios. En conjunto, el uso de \textit{deep learning}, redes neuronales y \textit{machine learning} permitió la creación de varios modelos muy útiles para la toma de decisiones que pueden ayudar a los entrenadores a aplicar las mejores estrategias para rotar a los jugadores lo que puede marcar una diferencia significativa.



\section{Líneas de trabajo futuras}


En este proyecto se ha analizado el rendimiento de diferentes modelos de inteligencia artificial sobre el desempeño de los equipos de fútbol basándose principalmente en los datos sobre la rotación de sus jugadores. Sin embargo, por la naturaleza del proyecto, debido a que es un proyecto académico, no se ha podido profundizar al máximo en estos aspectos y por tanto a continuación se definen posibles mejoras que puede tener el proyecto en el futuro y que no se han podido realizar en este trabajo.
\begin{itemize}
    \item \textbf{Incorporación de más ligas: } este aspecto podría incrementar la utilidad del sistema desarrollado de manera que sea capaz de ayudar a dirigentes y entrenadores de más clubes y países. Al abarcar más ligas más usuarios podrían utilizar el sistema.
    \item \textbf{Incorporación de más parámetros relacionados con la rotación de los jugadores:} este aspecto podría ayudar a mejorar el rendimiento de los modelos creados y por tanto proporcionar mejores resultados. En este proyecto desarrollado se pretenden utilizar los parámetros y variables más útiles, pero como mejora futura, se podría considerar analizar más parámetros que analicen diferentes datos.
    \item \textbf{Automatizar todo el código para que actualice los datos con los resultados de los últimos partidos:} en este proyecto, de manera inicial, se ha planteado que se deban ejecutar de manera manual los \textit{scripts} para la obtención de los datos de los últimos partidos, pero sin embargo, esta tarea sería importante automatizarla para el futuro.
    \item \textbf{Desarrollar una aplicación web para mostrar los datos obtenidos:} como mejora final, se podría desarrollar una aplicación web que muestre de una forma más amigable los datos obtenidos de los modelos y que puedan ayudar a los entrenadores y directivos. 
    \item \textbf{Integración de datos biométricos y físicos:} se podrían incorporar datos biométricos y físicos (frecuencia cardíaca, niveles de fatiga, velocidad, etc.) de los jugadores para afinar las estrategias de rotación basadas en la condición física real de los jugadores. De esta forma, todavía se podrían tomar decisiones más concretas y justificadas.
    \item \textbf{Ayuda para la realización de cambios en directo durante el transcurso del partido:} una importante mejora podría ser que los modelos fuesen capaces de ayudar a los entrenadores a tomar las decisiones sobre los cambios a realizar en el propio partido con avisos y notificaciones para así tomar las mejores decisiones posibles basadas en datos. Sería útil desarrollar simulaciones avanzadas que permitan a los entrenadores evaluar diferentes escenarios de rotación en tiempo real durante un partido, para tomar decisiones más informadas y así mejorar su capacidad en la toma de decisiones.
    \item \textbf{Comparar calidad y precisión de los modelos creados mediante redes neuronales con alternativas que utilizan modelos econométricos:} de esta manera se puede evaluar qué herramienta es más útil y proporciona mejores resultados permitiendo proporcionar conclusiones válidas a los entrenadores sobre cómo gestionar los jugadores.
\end{itemize}



\section{Limitaciones.}

Una de las principales limitaciones de este trabajo se relaciona con la precisión baja de los modelos entrenados. Esto se puede deber como se ha comentado, en gran medida a la naturaleza impredecible y aleatoria del fútbol. Factores como el desempeño individual de los jugadores, decisiones tácticas del entrenador, condiciones climáticas o decisiones arbitrales durante el transcurso de los partidos influyen considerablemente en los resultados y son complicadas de considerar con exactitud. Además, el fútbol es un deporte donde las dinámicas de los equipos y los estados de los jugadores desempeñan un papel fundamental, lo que añade un factor de complejidad que es complicado de recoger en los modelos de inteligencia artificial.

Otra limitación importante es la escasez de estudios previos y ejemplos comparables que recojan indicadores relacionados con la gestión de jugadores en el contexto del fútbol. Los trabajos existentes sobre modelos de rotación de empleados en otros sectores diferentes no son directamente aplicables a los equipos deportivos, donde se ha podido apreciar que las métricas y los factores más relevantes son diferentes. Esta falta de trabajos y ejemplos referentes dificulta la evaluación comparativa de la precisión y efectividad de los modelos desarrollados en este trabajo, lo que por otro lado, subraya la necesidad de realizar más investigación en esta área para seguir mejorando.