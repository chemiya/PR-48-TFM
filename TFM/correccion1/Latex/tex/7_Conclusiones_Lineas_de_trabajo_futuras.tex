\capitulo{7}{Conclusiones generales y Líneas de trabajo futuras}

Todo proyecto debe incluir las conclusiones que se derivan de su desarrollo. Éstas pueden ser de diferente índole, dependiendo de la tipología del proyecto, pero normalmente van a estar presentes un conjunto de conclusiones relacionadas con los resultados del proyecto y un conjunto de conclusiones técnicas. 
Además, resulta muy útil realizar un informe crítico indicando cómo se puede mejorar el proyecto, o cómo se puede continuar trabajando en la línea del proyecto realizado. 


\section{Conclusiones}
Las conclusiones de este proyecto destacan la trascendencia y el alcance que provoca la aplicación de diversas técnicas de inteligencia artificial en el contexto del fútbol, específicamente en la predicción del rendimiento de los equipos mediante el análisis de la rotación de jugadores. Este proyecto puede ser capaz de revelar que la implementación de herramientas de inteligencia artificial, como modelos de aprendizaje automático y análisis de datos, puede ayudar en la toma de decisiones a los entrenadores y directivos. En este proyecto se va a poder observar cómo estas tecnologías pueden ofrecer información crucial que tengan una gran repercusión en la toma de decisiones estratégicas de entrenadores y directivos de equipos, permitiéndoles optimizar la rotación de jugadores de manera más precisa y efectiva.

Además, en este proyecto destaca la importancia de disponer de conjuntos de datos completos y de calidad para proporcionar a estos modelos de inteligencia artificial de manera adecuada. La recopilación y preparación de datos precisos y relevantes sobre la rotación de jugadores y el rendimiento deportivo se ha establecido como un componente fundamental para el éxito de este proyecto. 

Este proyecto puede ayudar a destacar la necesidad de desarrollar herramientas y metodologías específicas que faciliten la integración de la inteligencia artificial en la gestión deportiva, lo que implicaría una colaboración conjunta entre expertos en deportes y científicos de datos.

En última instancia, este proyecto pretende mostrar el potencial de la inteligencia artificial para transformar y mejorar la gestión y el desempeño de los equipos de fútbol analizando los datos sobre la rotación de sus jugadores. Los hallazgos de este proyecto pretenden invitar a continuar investigando y desarrollando este campo, explorando nuevas tecnologías y metodologías que puedan maximizar el impacto positivo de la inteligencia artificial en el mundo del fútbol.












\section{Líneas de trabajo futuras}


En este proyecto se pretenden analizar el rendimiento de diferentes modelos de inteligencia artificial sobre el desempeño de los equipos de futbol basándose en los datos sobre la rotación de sus jugadores. Sin embargo, por la naturaleza del proyecto, debido a que es un proyecto académico, no se pretende profundizar al máximo en estos aspectos y por tanto a continuación se definen posibles mejoras que puede tener el proyecto en el futuro y que no se pretenden realizar en este trabajo.
\begin{itemize}
    \item \textbf{Incorporación de más ligas: } este aspecto podría incrementar la utilidad del sistema desarrollado de manera que sea capaz de ayudar a dirigentes y entrenadores de más clubes y países. Al abarcar más ligas más usuarios podrían utilizar el sistema.
    \item \textbf{Incorporación de más parámetros relacionados con la rotación de los jugadores:} este aspecto podría ayudar a mejorar el rendimiento de los modelos creados y por tanto proporcionar mejores resultados. En este proyecto desarrollado se pretenden utilizar los parámetros y variables más útiles, pero como mejora futura, se podría considerar analizar más parámetros que analicen diferentes datos.
    \item \textbf{Automatizar todo el código para que actualice los datos con los resultados de los últimos partidos:} en este proyecto, de manera inicial, se ha planteado que se deban ejecutar de manera manual los \textit{scripts} para la obtención de los datos de los últimos partidos, pero sin embargo, esta tarea seria importante automatizarla para el futuro.
    \item \textbf{Desarrollar una aplicación web para mostrar los datos obtenidos:} como mejora final, se podría desarrollar una aplicación web que muestre de una forma más amigable los datos obtenidos de los modelos y que puedan ayudar a los entrenadores y directivos. 
\end{itemize}

