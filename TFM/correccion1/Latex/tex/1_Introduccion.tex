\capitulo{1}{Introducción}



\section{Contexto}
Este proyecto se desarrolla como el Trabajo de Fin de Máster del Máster en Ingeniería Informática No Presencial de la Escuela de Ingeniería Informática de la Universidad de Valladolid y como continuación del trabajo realizado durante la estancia en un GIR para la asignatura de I+D+i.

En los últimos años, el campo de la inteligencia artificial ha sufrido un crecimiento considerable, alterando diferentes aspectos de la sociedad moderna \cite{ia2}. En este contexto, el deporte, y en especial el fútbol, no se ha mantenido al margen. La capacidad de la inteligencia artificial para analizar grandes volúmenes de datos y extraer patrones útiles ha encontrado una aplicación cada vez más importante en el ámbito deportivo, proporcionando nuevas herramientas para optimizar el rendimiento de los equipos y la toma de decisiones por parte de los directivos y entrenadores \cite{ia1}.

El fútbol, es algo más que un simple juego, se ha convertido en un fenómeno global que supera todas las fronteras. Los clubes de fútbol son empresas con mucho dinero y todo lo relacionado con este deporte, de manera general, mueve grandes cantidades de dinero. En este contexto altamente competitivo, la presión por obtener resultados positivos es máxima, tanto en términos deportivos como financieros \cite{dinero}. 

Un claro ejemplo de esto es el fichaje de Neymar por el Paris Saint-Germain (PSG) en 2017, que se firmó por la desorbitada cantidad de 222 millones de euros. Este traspaso no solo rompió récords, sino que también destacó cómo los grandes clubes están dispuestos a invertir grandes cantidades para atraer a los mejores jugadores y a su vez, generar ingresos mediante la venta de camisetas, patrocinadores y entradas a los partidos. La presencia de estrellas de este nivel en los equipos aumenta el valor comercial de los clubes, atrayendo aún más patrocinadores, haciendo mejores los acuerdos de \textit{marketing} y por lo tanto mejorando el estado financiero de los equipos \cite{neymar}.

Por otro lado, los derechos de televisión son otra fuente crucial de ingresos en el fútbol. Las ligas más importantes, como la Premier League inglesa, reciben miles de millones de libras por la transmisión de sus partidos. Estos contratos televisivos proporcionan la capacidad a los clubes de repartir grandes sumas de dinero entre ellos, lo que permite financiar fichajes y salarios enormes. El torneo británico concedió sus derechos televisivos en la temporada 2022/2023 por todo el mundo por un valor total de 8.136 millones de libras \cite{premier-tv}.

Además, los estadios modernos se han convertido en auténticas obras de ingeniería. El Santiago Bernabéu del Real Madrid, por ejemplo, está siendo remodelado con una inversión cercana a los 1950 millones de euros. Esta obra no solo incrementará considerablemente la experiencia de los aficionados, sino que también permitirá al club generar ingresos adicionales a través de eventos no deportivos como conciertos y nuevas áreas comerciales, como ya está sucediendo. Así, el fútbol se está convirtiendo en una industria multimillonaria que atrae inversiones de todo el mundo, consolidándose como el principal espectáculo mundial \cite{bernabeu}.

En este escenario, la gestión eficiente de los recursos humanos, como en este caso los jugadores, se ha vuelto prioritaria para el éxito de un equipo. Los entrenadores y directivos se enfrentan al desafío de optimizar el rendimiento de sus jugadores tratando de minimizar el riesgo de lesiones y el cansancio físico. La inteligencia artificial ofrece herramientas potentes para abordar este desafío, permitiendo el análisis de datos relacionados con el estado físico de los jugadores, el rendimiento en partidos anteriores, las lesiones previas y otros factores relevantes.

La gestión de la rotación de jugadores es uno de los aspectos más críticos de la estrategia de un equipo a lo largo de una temporada y que tiene una mayor repercusión sobre su éxito. La inteligencia artificial puede ayudar a los entrenadores a tomar decisiones documentadas sobre cuándo dar descanso a un jugador, cuándo alinear a un futbolista, qué cambios realizar y cómo mantener un equilibrio entre la competitividad y la salud de la plantilla. 

Los equipos de fútbol utilizan el análisis de datos para gestionar eficientemente a sus jugadores y optimizar su rendimiento en el campo, así como para reducir el riesgo de lesiones. Mediante tecnologías avanzadas y técnicas de última generación, los clubes recopilan un amplio campo de datos, incluyendo métricas de rendimiento físico como la distancia recorrida, velocidad, y cargas de trabajo. Estos datos permiten a los entrenadores y al personal médico controlar la condición física que tienen los jugadores en tiempo real, identificar patrones de fatiga, y ajustar los entrenamientos y tiempos de juego según como corresponda en base a los datos recopilados. 

Un club que aplica esto es el Liverpool FC que emplea el análisis de datos para tomar decisiones justificadas sobre cuándo rotar a sus jugadores, garantizando no sobrepasar sus límites físicos y que se mantengan al máximo rendimiento durante toda la temporada. Este enfoque basado en datos no solo aumenta el rendimiento individual y colectivo del equipo, sino que también ayuda a evitar lesiones, prolongando así la carrera de los jugadores y garantizando su disponibilidad y rendimiento para los partidos clave de la temporada donde es necesario que estén al 100\% \cite{liverpool}.

Por lo tanto, el uso de la inteligencia artificial en el fútbol no solo es una oportunidad para mejorar el rendimiento deportivo, sino también una necesidad para mejorar sobre los rivales en un entorno cada vez más competitivo y exigente. Los equipos que puedan lograr aprovechar e incorporar de manera efectiva estas herramientas tendrán una ventaja considerable sobre el resto para conseguir sus objetivos deportivos y financieros.

















\section{Motivación}

El crecimiento en los últimos años de la inteligencia artificial ha despertado un interés en su aplicación en diversos campos como en el deporte. En el ámbito del fútbol, la capacidad de utilizar la inteligencia artificial para analizar datos complejos y tomar decisiones estratégicas concretas en base a ellos, ofrece un gran potencial para incrementar el rendimiento de los equipos. Esta motivación se debe a la necesidad de los clubes por mantenerse competitivos en un entorno que está en constante evolución, donde la línea entre el éxito y el fracaso cada vez es más estrecha.

La inteligencia artificial ha transformado el ámbito deportivo al proporcionar herramientas avanzadas para el análisis de datos, la mejora del rendimiento y la toma de decisiones justificadas. En el fútbol, la inteligencia artificial se utiliza para analizar enormes volúmenes de datos de partidos, dando la capacidad a los entrenadores de desarrollar tácticas más precisas, adecuadas y adaptadas. Además, la inteligencia artificial optimiza la gestión de la salud de los deportistas al predecir el riesgo de lesiones y proponer programas de entrenamiento personalizados que se adapten a ellos. Fuera del campo, la inteligencia artificial cada vez mejora más la experiencia de los aficionados. La inteligencia artificial está revolucionando la forma en que los equipos y atletas se preparan y compiten, elevando el deporte de alto nivel a nuevos escalones \cite{ia-deporte}.

El proyecto aparece como respuesta al incremento en la demanda de herramientas que permitan a los clubes mejorar en la gestión de sus recursos humanos, en concreto de sus jugadores. La inteligencia artificial tiene la capacidad de analizar grandes cantidades de datos sobre los jugadores y equipos, detectando patrones y tendencias que pueden pasar desapercibidos para las personas. Al integrar estas conclusiones en la toma de decisiones, los equipos pueden mejorar la eficiencia de su rotación de jugadores, incrementando así sus posibilidades de éxito en el campo.

En los últimos años, la demanda de especialistas en análisis de datos en el ámbito deportivo ha aumentado considerablemente debido a la creciente importancia de los datos para optimizar el rendimiento y la estrategia de los equipos y atletas. Equipos y organizaciones deportivas ahora buscan analistas de datos para analizar métricas de rendimiento, evaluar riesgos de lesiones y desarrollar tácticas más eficientes contra sus rivales. Este crecimiento se debe a los avances tecnológicos y a la comprensión de que el análisis de datos puede generar una ventaja competitiva considerable, incrementando tanto el desempeño individual de los atletas como los resultados colectivos de los equipos \cite{incremento-ia-deporte}.

Este proyecto tiene el potencial para marcar un cambio significativo en la manera en que se realiza la gestión deportiva en el fútbol moderno. Al ofrecer a los clubes herramientas avanzadas de análisis y toma de decisiones, se espera que este proyecto ayude no solo a mejorar los resultados deportivos, sino también a fortalecer la posición competitiva y el rendimiento financiero de los equipos en un mercado cada vez más exigente y competitivo.













\section{Aplicaciones similares}
A continuación, se detallan aplicaciones y proyectos similares a lo que se pretende desarrollar y que pueden servir de referencia.
\begin{itemize}
\item \textbf{LaLiga Beyond Stats:}
esta es una iniciativa de LaLiga que tiene como objetivo emplear las últimas tecnologías, relacionadas con el análisis de datos y la inteligencia artificial, para proporcionar una comprensión más profunda y completa de los partidos. Esta plataforma busca ofrecer a los aficionados, entrenadores, jugadores y clubes herramientas innovadoras para analizar y entender el rendimiento en el fútbol, más allá de las estadísticas habituales, a través de datos en tiempo real y visualizaciones interactivas, proporcionando así un enfoque más inteligente y apasionado hacia el deporte \cite{beyondstats}.

\item \textbf{Aplicación de la inteligencia artificial en la Premier League:}
esta liga utiliza la inteligencia artificial para determinar las probabilidades de que un equipo gane un partido mediante el análisis de un amplio rango de datos. Estos factores abarcan datos históricos de partidos anteriores, como el rendimiento del equipo en casa y fuera de casa, su posición en la tabla de clasificación, su forma actual y lesiones de jugadores clave entre otros. Además, se tienen en cuenta variables más específicas, como la posesión de balón, los tiros a puerta, las oportunidades creadas y el rendimiento en defensa y en ataque. Estos datos son proporcionados a algoritmos de aprendizaje automático que son capaces de analizar patrones complejos y entrenar modelos predictivos que estiman las probabilidades de que gane el local, empaten o gane el visitante. De esta manera, la inteligencia artificial proporciona una herramienta útil para predecir resultados de partidos de fútbol con un alto grado de precisión, lo que puede ser utilizado por equipos, aficionados y casas de apuestas para tomar decisiones en base a diferentes argumentos \cite{oracle} \cite{iaFutbol}.

\item \textbf{Opta:}
es una empresa líder en análisis y datos deportivos que es capaz de proporcionar información detallada y estadísticas sobre una amplia gama de eventos deportivos, incluyendo fútbol, \textit{rugby}, \textit{cricket} y otros. Para ello, utiliza tecnologías avanzadas de recopilación y análisis de datos donde recopila datos en tiempo real durante los eventos deportivos y los convierte en información valiosa y estadísticas significativas que son utilizadas por equipos, entrenadores, medios de comunicación y aficionados para analizar de mejor manera el juego y evaluar el rendimiento de los jugadores y equipos. Opta se ha convertido en un recurso importantísimo en el mundo del deporte para análisis de datos y seguimiento de estadísticas \cite{opta}.

\item \textbf{Mejora del rendimiento de los jugadores mediante la ciencia de datos en el Liverpool FC:} como se ha comentado previamente, uno de los principales equipos y pioneros en este área que utiliza la ciencia de datos e inteligencia artificial para mejorar el rendimiento de sus jugadores es el Liverpool FC. Mediante un enfoque avanzado, emplean grandes cantidades de datos, incluyendo métricas de rendimiento físico, táctico y de salud, recogidos durante sus entrenamientos y partidos. Estos datos son analizados para mejorar las estrategias de juego, establecer entrenamientos personalizados y gestionar la carga de trabajo de los jugadores, lo que proporciona una gran ayuda a prevenir lesiones. Además, el análisis de datos permite al cuerpo técnico tomar decisiones razonadas sobre alineaciones y tácticas, ajustando su enfoque basándose en el rendimiento y las características del oponente, lo que contribuye considerablemente al éxito del equipo en la temporada.

\end{itemize}
















\section{Estructura de la memoria}

Este documento se estructura de la siguiente forma:
\begin{description}

\item[Capítulo 2 Objetivos del proyecto:] en este capítulo se describen los objetivos que se quieren conseguir con la ejecución de este proyecto. Estos se dividen en dos categorías distintas,
los objetivos de desarrollo y los objetivos académicos.


\item[Capítulo 3 Conceptos teóricos:] en este capítulo se expone una explicación teórica de los conceptos más importantes que se han utilizado para el desarrollo de este proyecto.

\item[Capítulo 4 Técnicas y herramientas:] en este capítulo se describen las técnicas utilizadas para la obtención de los datos y las tecnologías utilizadas para el desarrollo del proyecto.


\item[Capítulo 5 Aspectos relevantes del desarrollo del proyecto:] en este capítulo se recogen los aspectos más interesantes del desarrollo del proyecto como los detalles sobre las fases de análisis, diseño e implementación y el tratamiento de los datos.


\item[Capítulo 6 Resultados:] en este capítulo se exponen los resultados obtenidos y todas las conclusiones que se han extraído en el desarrollo del proyecto y que pueden ayudar a los entrenadores a cómo gestionar los jugadores.


\item[Capítulo 7 Conclusiones y líneas de trabajo futuras:] en este capítulo se explican las conclusiones finales adquiridas del proyecto junto a las posibles líneas de trabajo futuras a seguir.

\item[Apéndice Manual de instalación:] en este apéndice se describe el repositorio donde se encuentra el código y cómo se puede instalar y desplegar.

\end{description}



