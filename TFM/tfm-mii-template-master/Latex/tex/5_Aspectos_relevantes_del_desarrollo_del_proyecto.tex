\capitulo{5}{Aspectos relevantes del desarrollo del proyecto}



\section{Introducción}
En este capitulo se recogen los aspectos más interesantes del desarrollo del proyecto, donde se documenta desde la metodologia aplicada para el desarrollo del proyecto, describiendo los pasos a seguir y el alcance que se espera que tenga el proyecto. Por otro lado, tambien se detalla la planificacion del proyecto, cual es el modelo de los datos, que operaciones de transformacion y limpieza han sido necesarias y finalmente se describe la implementacion de los archivos que forman parte del proyecto.


\section{Metodología}
La metodología de este proyecto tiene como base un enfoque que comprende varias etapas clave. En primer lugar, se realizará una revisión de diferentes artículos sobre técnicas de inteligencia artificial aplicadas al análisis de datos deportivos, centrándose especialmente en la predicción del rendimiento de equipos de fútbol basándose en la rotación de los jugadores. Esta revisión ayudará a identificar las mejores prácticas y los enfoques que pueden ser más relevantes para el desarrollo del proyecto.

Posteriormente, se llevará a cabo la recopilación y preparación de datos, donde se recogerán conjuntos de datos históricos que abarquen información relevante sobre la rotación de jugadores y el rendimiento deportivo de equipos de fútbol en las ligas seleccionadas. Esta etapa incluye la limpieza de datos, la preparación de los datos para su análisis posterior y un breve análisis sobre ellos para detectar patrones.

Una vez preparados los datos, se realizará la implementación y evaluación de modelos de inteligencia artificial. Se probarán los diferentes algoritmos comentados en la parte teorica y diferentes redes neuronales utilizando los parametros tambien comentados. Los modelos se entrenarán y ajustarán utilizando los datos que hemos obtenido previamente, y se evaluará su rendimiento utilizando kas métricas comentadas. Esta fase permitirá detectar los modelos más eficaces y precisos para predecir el rendimiento deportivo basado en la rotación de jugadores.

\section{Alcance}
El alcance de este proyecto abarca la evaluación y aplicación de diversas técnicas de inteligencia artificial para predecir el rendimiento de equipos de fútbol basándose en la rotación de jugadores. En primer lugar, se definirá la metodología y se seleccionarán las técnicas más correctas para el análisis de datos relacionados con la rotación de jugadores y el rendimiento deportivo. Este apartado incluirá la recopilación, preprocesamiento y análisis de los datos de las ligas, equipos y jugadores de fútbol seleccionados. Para la parte del análisis de los datos, se realizan unos pequeños programas que analicen los datos obtenidos mediante mapas de calor para poder detectar patrones. Al detectar estos patrones se pretende justificar si las estrategias de rotacion aplicadas por los equipos mejoran el rendimiento o no.

Las ligas sobre las que se obtendrán y utilizarán los datos serán LaLiga EA Sports (primera división española), Premier League (primera división inglesa) y Bundesliga (primera división alemana) desde la temporada 2018/2019 hasta la temporada 2023/2024, ambas incluidas.

Además, el alcance del proyecto se pretende que también implique la implementación y ajuste de modelos de inteligencia artificial para la predicción del rendimiento deportivo en función de la rotación de jugadores. Para ello, se explorarán diversas técnicas de inteligencia artificial y \textit{machine learning}, como redes neuronales, árboles de decisión, y métodos de aprendizaje automático supervisado y no supervisado, con el objetivo de detectar aquellas que mejor se adapten a las características de este problema. Sobre cada una de ellas, se realizará una optimización de parámetros para mejorar todo lo posible su precisión.
Finalmente, se realizará una evaluación de los modelos desarrollados, utilizando métricas de rendimiento para definir su eficacia y precisión en la predicción del rendimiento de los equipos. Después de esto, se seleccionará el mejor modelo y se evaluará su rendimiento en la actualidad aplicándolo sobre partidos que estén por jugarse.

Además de todos estos aspectos comentados, se documentarán y analizarán todas las tareas realizadas en el proyecto, con el objetivo de ofrecer recomendaciones para la gestión de la rotación de jugadores en equipos de fútbol, así como posibles áreas de mejora y futuras investigaciones para este proyecto.




\section{Plan de proyecto}

El proyecto comienza las primeras semanas durante la estancia en un GIR para la asignatura de I+D+i donde se realiza una parte de investigacion y se desarrolla el nucleo del proyecto. Para finalizar, el proyecto continua como Trabajo de Fin de Master, donde se sigue profundizando y expandiendo el trabajo realizado previamente. La duracion de cada una de estas secciones es 190 horas y 150 horas respectivamente, sumando en total 340 horas. La fecha de inicio del proyecto es el 29 de abril de 2024 y la fecha limite de finalizacion es el 28 de junio de 2024.


La Tabla \ref{table:planificacion} muestra la planificación de las semanas durante las que se desarrolla el proyecto.

\begin{table}[]
  \centering
  \begin{tabular}{|c|c|c|c|c|}
  \hline
  { \textbf{Semana}} & { \textbf{Fecha de inicio}} & { \textbf{Fecha de fin}} & { \textbf{Carga de trabajo}}& { \textbf{Sección}} \\ \hline
  1 & 29/04/2024 & 05/05/2024 & 40 horas  & I+D+i                                              \\ \hline
  2 & 06/05/2024 & 12/05/2024 & 40 horas   & I+D+i                                         \\ \hline
  3 & 13/05/2024 & 19/05/2024 & 40 horas   & I+D+i                                             \\ \hline
  4 & 20/05/2024 & 26/05/2024 & 30 horas   & I+D+i                                              \\ \hline
  5 & 27/05/2024 & 02/06/2024 & 24 horas   & I+D+i                                         \\ \hline
  6 & 03/06/2024 & 09/06/2024 & 16 horas    & I+D+i                                               \\ \hline
  7 & 10/06/2024 & 16/06/2024 & 60 horas    & TFM                                             \\ \hline
  8 & 17/06/2024 & 23/06/2024 & 60 horas    & TFM                                           \\ \hline
  9 & 24/06/2024 & 28/06/2024 & 30 horas   & TFM                                            \\ \hline
  \end{tabular}
  \caption{Planificación de las semanas.}
\label{table:planificacion}
  \end{table}

\section{Modelo de los datos}
Los datos obtenidos se asocian a diferentes entidades que estan relacionadas entre si y abarcan multitud de campos, por ello, es importante estructurarlos de la manera correcta para que lpuedan ser utilizados adecuadamente en el entrenamiento de los modelos. En la figura \ref{fig:arquitectura-logica} se puede apreciar el modelo de los datos y como se han almacenado de forma estructurada despues de extraerlos mediante scraping.



A continuacion, se realiza una breve descripcion de cada entidad:

\section{Limpieza y transformación de los datos}
Las tareas de limpieza y transformacion de los datos para prepararlos para que puedan ser utilizados en el entrenamiento de los modelos se describen a continuacion:

\begin{itemize}
    \item \textbf{Eliminar datos sobre substituciones de jugadores no detectados:} se han eliminado los registros de datosJugadoresPartidos donde no se ha podido extraer la posicion sobre el jugador sustituido. Esto ha sucedido un con apenas 3 jugadores en todas las ligas y temporadas evaluadas y por lo tanto el numero de registros afectados en minimo.
    \item \textbf{Seleccionar partidos a partir de la jornada 10:} se han filtrado los datos de los partidos dejando solamente los partidos jugados desde la jornada 10 hasta el final. Esto se ha hecho ya que los datos que se tienen en cuenta para cada partido unicamente consideran los partidos previos de los equipo que disputan ese encuentro en esa temporada y por tanto, hasta la jornada 10, no se considera que existen datos suficientes para obtener conclusiones estables sobre como se comporta ese equipo.
    \item \textbf{Eliminacion de ids:} para preparar los datos para entrenar los modelos, se han eliminado tanto el id del partido asociado como el id unico del dato para cada registro con los datos de los indicadores para un partido.
    \item \textbf{Transformacion de la clase:} antes de entrenar las redes neuronales con los datos obtenidos, se han transformado los datos de los registros de la clase a predecir, ya sea el ganador del partido, el numero de goles del local o el numero de goles del visitante, aplicando one-hot para que las redes neuronales puedan utilizar estos datos.
    \item \textbf{Normalizacion de los datos:} esta es una técnica de preprocesamiento que ajusta los valores de los datos para que se encuentren en un rango común que en este caso es [0, 1]. Esto mejora la eficiencia y la precisión de los algoritmos de machine learning al garantizar que todas las características contribuyan equitativamente. En este caso, es crucial para evitar que características con valores más grandes dominen el modelo ya que hay atributos que pueden tomar valores muy grandes y otros valores muy pequeños.
    
\end{itemize}


\section{Implementación}
El codigo del proyecto se ha ejecutado en una maquina virtual proporcionada por la Escuela
Todo el codigo del proyecto se ha dividido en diferentes carpetas. A continuacion se detalla la finalidad de cada una estas carpetas y sus archivos: