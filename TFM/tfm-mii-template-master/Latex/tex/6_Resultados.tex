\capitulo{6}{Resultados y conclusiones obtenidas sobre los datos}

\section{Introducción}
En este capitulo se detallan los resultados y conclusiones que se han podido obtener de los datos y de los modelos creados para asi poder determinar cuales son las mejores estrategias de rotacion de jugadores y asi poder definir conclusiones claras que puedan ayudar a los entrenadores y directivos a tomar decisiones. 

\section{Conclusiones extraidas del analisis previo de los datos}

Previo al entrenamiento de los modelos, se ha realizado un análisis de los datos obtenido para 
extraer posibles patrones sobre ellos. Para realizar esto, se ha extraído el valor de cada indicador 
analizado de forma general para cada equipo en el último partido de la temporada que jugase y 
después se ha realizado la clasificación ordenando de mayor a menor por este indicador entre 
todos los equipos de esa liga. Después se ha evaluado en cuantas posiciones difiere la posición de 
este equipo en esta clasificación por el indicador respecto a la clasificación original ordenando 
por puntos. 

Con estos datos, se ha realizado un mapa de calor de manera que las diferencias negativas se 
marcasen en rojo y las diferencias positivas se marcasen en verde. Mediante este análisis se 
pretenden desmentir o confirmar pensamientos que están extendidos de forma generalizada en el 
mundo de futbol y que pueden ayudar a extraer conclusiones de los datos, como por ejemplo, los 
equipos que hacen más cambios introduciendo delanteros, anotan más goles.
En la siguiente tabla se muestran los datos para los equipos de LaLiga en la temporada 2024 para 
la proporción de cambios en la alineación de delanteros en sus partidos en general:



Aquí se puede observar que el Real Madrid tiene un valor de -19 en verde. Esto se explica porque 
hay 19 posiciones de diferencia entre su posición por la clasificación por puntos y su posición en 
la clasificación por este indicador que es la proporción de cambios en la alineación de delanteros 
en sus partidos en general. Por lo tanto, es un equipo con alto rendimiento ya que ocupa posiciones 
altas en la clasificación por puntos, pero sin embargo no suele realizar muchos cambios de 
delanteros en sus alineaciones iniciales. Por el contrario, el Cádiz que tiene un valor de 16 en rojo, 
se debe a que ocupa una baja posición en la clasificación por puntos, pero por otro lado, ocupa 
una alta posición en la clasificación por este indicador. Por lo tanto, es un equipo que tiene pocos 
puntos y un bajo rendimiento y que tiene una elevada proporción de cambios de delanteros en su 
alineación inicial.

En conclusión, en el análisis de este indicador se aprecia que los equipos de la zona alta de la 
clasificación por puntos tienen valores bajos en la proporción de cambios de delanteros en la 
alineación inicial y los equipos de la zona baja de la clasificación por puntos tienen valores altos 
en la proporción de cambios de delanteros en la alineación inicial.
A continuación, se detallan el resto de las principales conclusiones extraídas que se han podido 
apreciar de manera generalizada en todas las temporadas de las ligas evaluadas mediante este 
análisis:


\begin{itemize}
    \item Los equipos que realizan más cambios en las alineaciones iniciales entre un partido y otro
    tienden a estar en las posiciones más bajas de la clasificación y por el contrario, los 
    equipos que realizan menos cambios en las alineaciones iniciales entre un partido y otro, 
    tienden a estar en las posiciones más altas de la clasificación. Este efecto se aprecia sobre 
    todo si los cambios en las alineaciones iniciales afectan a los defensas o delanteros
    \item Los equipos que realizan más cambios antes del descanso tienden a estar en posiciones 
    más bajas en la clasificación. Esto se puede apreciar sobre todo en la Premier League y 
    Bundesliga. Por otro lado, para todas las ligas, los equipos que hacen menos cambios 
    entre los minutos 61 a 75, tienden a ocupar posiciones más altas en la clasificación
    \item Los equipos que hacen más cambios sacando defensas e introduciendo jugadores más
    ofensivos, suelen ocupar posiciones más bajas en la clasificación. Por otra parte, los 
    equipos que hacen menos cambios sacando delanteros e introduciendo jugadores más
    defensivos, tienden a ocupar posiciones más altas en la clasificación
    \item Respecto a la media de los minutos en la que los equipos realizan los cambios, los equipos 
    de la zona alta de la clasificación tienen valores más bajos en este valor, por lo tanto, de 
    media suelen realizar los cambios antes.
    \item Los equipos de la zona alta de la clasificación suelen hacer menos cambios de jugadores 
    que han sido amonestados con amarilla.
    \item Finalmente, los equipos de la zona alta de la clasificación suelen menores valores en la 
    proporción de cambios que realizan por partido, es decir, en sus partidos no suelen gastar
    los 5 cambios de los que disponen.
    
\end{itemize}


\section{Conclusiones extraidas al entrenar los modelos}

A la hora de entrenar los modelos, para los datos de cada temporada, se han eliminado los datos 
asociados a los partidos anteriores a la jornada 10, ya que hasta entonces no se ha considerado 
que existan datos suficientes sobre los partidos previos que han jugado los equipos y por lo tanto 
los indicadores y porcentajes pueden tener valores extremos. Esto se debe a que para cada partido 
para el cálculo de los indicadores solo se tienen en cuenta los partidos previos del equipo en esa 
temporada.

Después de esto, en total, agrupando los datos de las 3 ligas evaluadas en las temporadas 
comentadas, se han obtenido los datos asociados a 4822 partidos. 
Con Python y Tensorflow se han entrenado diferentes modelos de Machine Learning e 
inteligencia artificial. Entre estos modelos se pueden encontrar árboles de decisión, Naive Bayes, 
bosques aleatorios, regresiones logísticas, SVM y redes neuronales. Los datos utilizados para 
entrenar los modelos se han dividido en tres conjuntos y se han optimizado los parámetros de cada 
uno de los modelos.

En las pruebas iniciales se ha apreciado que las redes neuronales con Tensorflow han obtenido 
una exactitud mayor a los modelos de Machine Learning que se han citado previamente. Por lo 
tanto, se decidió seguir profundizando sobre estas redes neuronales. Para ello, en la siguiente 
tabla, se establecen los parámetros que se han optimizado sobre estas redes neuronales y los 
valores diferentes que podían tomar.


Previo a esta elección de parámetros, se realizó un filtrado eliminando opciones que no tenían 
repercusión sobre la exactitud de los modelos. Por ejemplo, se redujeron los valores de las épocas 
a analizar a 3 valores diferentes, ya que no existía mucha diferencia entre estos valores. Con estas 
combinaciones de parámetros, se pueden obtener 108 combinaciones diferentes que son las que 
se han evaluado para crear los mejores modelos para predecir tanto el resultado del partido, los 
goles del local y los goles del visitante. Por lo tanto, con los datos obtenidos, para cada 
combinación de parámetros, se entrena una red neuronal y se evalúa su exactitud. Finalmente se 
obtienen los parámetros que se utilizaron en la red neuronal que mejor exactitud ha obtenido. Este 
proceso se realiza para predecir el ganador del partido, los goles del equipo local y los goles del 
equipo visitante, obteniendo tres combinaciones de parámetros que se comentan a continuación.

Respecto al ganador del partido, finalmente se ha seleccionado como el mejor modelo con mayor 
precisión, una red neuronal con Tensorflow, en la que se han seleccionado de entre las 108 
combinaciones posibles de parámetros, una estructura con 3 capas, con descenso de gradiente 
estocástico como optimizador, con entropía cruzada categórica como función de perdida, 
entrenada en 20 épocas con un tamaño de batch de 32 que no incorpora callbacks. Esta red ha 
obtenido una exactitud sobre el conjunto de prueba del 56\%.

Respecto a los goles del local, finalmente se ha seleccionado como el mejor modelo con mayor 
precisión, una red neuronal con Tensorflow, en la que se han seleccionado de entre las 108 
combinaciones posibles de parámetros, una estructura con 4 capas, con descenso de gradiente 
estocástico como optimizador, con entropía cruzada categórica como función de perdida, 
entrenada en 20 épocas con un tamaño de batch de 64 que no incorpora callbacks. Esta red ha 
obtenido una exactitud sobre el conjunto de prueba del 37\%.

Respecto a los goles del visitante, finalmente se ha seleccionado como el mejor modelo con mayor 
precisión, una red neuronal con Tensorflow, en la que se han seleccionado de entre las 108 
combinaciones posibles de parámetros, una estructura con 4 capas, con Adam como optimizador, 
con entropía cruzada categórica como función de perdida, entrenada en 20 épocas con un tamaño 
de batch de 64 que no incorpora callbacks. Esta red ha obtenido una exactitud sobre el conjunto 
de prueba del 38\%.

La exactitud de los modelos sobre los goles de los equipos pueden parecer bajas pero se debe 
tener en cuenta, que predecir el número de goles exacto de un equipo es más complejo, ya que 
este valor puede tomar muchos valores diferentes



\section{Conclusiones y resultados generales}
Se ha podido observar que en las predicciones realizadas por el modelo para predecir el ganador 
del partido, las variables que más repercusión tienen para aumentar la probabilidad de victoria del
equipo local son la proporción de este equipo de realizar cambios entre los minutos 61 a 75, la
proporción de cambios de defensas a centrocampistas y la proporción de cambios de
centrocampistas a defensas y para el visitante la proporción de este equipo de cambios de
centrocampistas a defensas, la proporción de cambios de defensas a delanteros y la proporción de 
cambios entre el minuto 76 al final. Los equipos con valores más elevados en estos indicadores 
tienen según el modelo más probabilidad de ganar el partido.

Por otro lado, para el modelo que predice la probabilidad que tiene el equipo local de marcar un 
determinado número de goles, las variables que tienen más repercusión para aumentar las 
probabilidades de que el equipo marque más goles son la proporción de este equipo de cambios 
entre los minutos del 45 al 60, la proporción de cambios de defensas a centrocampistas, la 
proporción de cambios de defensas a delanteros y la proporción de cambios de centrocampistas a 
delanteros. Los equipos locales con valores más elevados en estos indicadores tienen según este 
modelo más probabilidades de anotar un número más elevado de goles.

Para el modelo que predice la probabilidad que tiene el equipo visitante de marcar un determinado
número de goles, las variables que más repercusión tienen para aumentar las probabilidades de 
que el equipo marque más goles son la proporción del equipo de cambios de delanteros en la 
alineación inicial, la proporción de cambios de centrocampistas en la alineación inicial, la 
proporción de cambios de defensas a centrocampistas y la proporción de cambios entre los 
minutos 76 al final. Los equipos visitantes con valores más elevados en estos indicadores tienen 
según este modelo más probabilidades de anotar un número más elevado de goles.

Previo a los modelos, el análisis previo de los datos ha revelado diferentes patrones que pueden 
ayudar enormemente a los entrenadores a tomar decisiones sobre los jugadores para aumentar el 
rendimiento del equipo, como evitar realizar muchos cambios en las alineaciones iniciales o evitar 
realizar muchos cambios antes del descanso, ya que ambos factores en este caso están 
estrechamente relacionados con los equipos de la zona baja de la clasificación. 

Por otro lado, realizar menos cambios en las alineaciones iniciales y menos cambios entre los 
minutos 61 y 75 se asocia a equipos que ocupan las posiciones más altas de la clasificación, y esto 
puede ser una buena señal de que estas estrategias ayudan al buen rendimiento del equipo.
En cuanto a las posiciones de los jugadores que son afectados por los cambios, realizar cambios 
ofensivos quitando defensas es una estrategia que se puede apreciar sobre todo en los equipos de 
la zona baja de la clasificación y por lo tanto no ayuda a su rendimiento. Por otra parte, realizar 
pocos cambios sacando delanteros e introduciendo jugadores más defensivos es una estrategia 
utilizada por los equipos de la zona alta de la clasificación y por lo tanto parece que contribuye a 
su éxito y buen rendimiento.

Sobre los minutos en los que se realizan los cambios, los equipos que de media realizan los 
cambios antes suelen estar en la zona alta de la clasificación y lo mismo es aplicable a los cambios 
que reemplazan jugadores amonestados con amarilla ya que los equipos con menos cambios de 
este tipo ocupan posiciones más altas.

Finalmente, otro dato bastante significativo y curioso es que los equipos de la zona alta de la 
clasificación tienen valores más bajos en el número de cambios por partido que hacen, es decir, 
no gastan todos los cambios de los que disponen. Esto puede contradecir la creencia generalizada 
de que al realizar más cambios el equipo debería de rendir más porque introduce jugadores 
totalmente frescos pero parece que este análisis muestra lo contrario, que conviene mantener los 
jugadores que estén en el campo y no necesariamente gastar todos los cambios de los que 
disponen.