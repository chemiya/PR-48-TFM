\capitulo{1}{Introducción}



\section{Contexto}
Este proyecto se desarrolla como el Trabajo de Fin de Máster del Máster en Ingeniería Informática No Presencial de la Escuela de Ingeniería Informática de la Universidad de Valladolid y como continuación del trabajo realizado durante la estancia en un GIR para la asignatura de I+D+i.

En los últimos años, el campo de la inteligencia artificial ha sufrido un crecimiento considerable, alterando diferentes aspectos de la sociedad moderna \cite{ia2}. En este contexto, el deporte, y en especial el fútbol, no se ha mantenido al margen. La capacidad de la inteligencia artificial para analizar grandes volúmenes de datos y extraer patrones útiles ha encontrado una aplicación cada vez más importante en el ámbito deportivo, proporcionando nuevas herramientas para optimizar el rendimiento de los equipos y la toma de decisiones por parte de los directivos y entrenadores \cite{ia1}.

El fútbol, es algo más que un simple juego, se ha convertido en un fenómeno global que supera todas las fronteras. Los clubes de fútbol son empresas con mucho dinero y todo lo relacionado con este deporte, de manera general, mueve grandes cantidades de dinero. En este contexto altamente competitivo, la presión por obtener resultados positivos es máxima, tanto en términos deportivos como financieros \cite{dinero}. 

En este escenario, la gestión eficiente de los recursos humanos, como en este caso los jugadores, se ha vuelto prioritaria para el éxito de un equipo. Los entrenadores y directivos se enfrentan al desafío de optimizar el rendimiento de sus jugadores tratando de minimizar el riesgo de lesiones y el cansancio físico. La inteligencia artificial ofrece herramientas potentes para abordar este desafío, permitiendo el análisis de datos relacionados con el estado físico de los jugadores, el rendimiento en partidos anteriores, las lesiones previas y otros factores relevantes.

La gestión de la rotación de jugadores es uno de los aspectos más críticos de la estrategia de un equipo a lo largo de una temporada y que tiene una mayor repercusión sobre su éxito. La inteligencia artificial puede ayudar a los entrenadores a tomar decisiones documentadas sobre cuándo dar descanso a un jugador, cuándo alinear a un futbolista, que cambios realizar y cómo mantener un equilibrio entre la competitividad y la salud de la plantilla. 
Por lo tanto, el uso de la inteligencia artificial en el fútbol no solo es una oportunidad para mejorar el rendimiento deportivo, sino también una necesidad para mejorar sobre los rivales en un entorno cada vez más competitivo y exigente. Los equipos que puedan lograr aprovechar de manera efectiva estas herramientas tendrán una ventaja considerable en la consecución de sus objetivos deportivos y financieros.

















\section{Motivación}

El crecimiento en los últimos años de la inteligencia artificial ha despertado un interés en su aplicación en diversos campos como en el deporte. En el ámbito del fútbol, la capacidad de utilizar la inteligencia artificial para analizar datos complejos y tomar decisiones estratégicas concretas ofrece un gran potencial para incrementar el rendimiento de los equipos. Esta motivación se debe a la necesidad de los clubes por mantenerse competitivos en un entorno en constante evolución, donde la línea entre el éxito y el fracaso es muy estrecha.

El proyecto aparece como respuesta al incremento en la demanda de herramientas que permitan a los clubes mejorar en la gestión de sus recursos humanos, en concreto de sus jugadores. La inteligencia artificial tiene la capacidad de analizar grandes cantidades de datos sobre los jugadores y equipos, detectando patrones y tendencias que pueden pasar desapercibidos para las personas. Al integrar estas conclusiones en la toma de decisiones, los equipos pueden mejorar la eficiencia de su rotación de jugadores, incrementando así sus posibilidades de éxito en el campo.

Por último, la motivación detrás de este proyecto también se debe a su potencial para marcar un cambio significativo en la manera en que se realiza la gestión deportiva en el fútbol moderno. Al ofrecer a los clubes herramientas avanzadas de análisis y toma de decisiones, se espera que este proyecto ayude no solo a mejorar los resultados deportivos, sino también a fortalecer la posición competitiva y el rendimiento financiero de los equipos en un mercado cada vez más exigente y competitivo.













\section{Aplicaciones similares}
A continuación, se detallan aplicaciones y proyectos similares a lo que se pretende desarrollar y que pueden servir de referencia.
\begin{itemize}
\item \textbf{LaLiga Beyond Stats:}
esta es una iniciativa de LaLiga que tiene como objetivo emplear las últimas tecnologías, relacionadas con el análisis de datos y la inteligencia artificial, para proporcionar una comprensión más profunda y completa de los partidos. Esta plataforma busca ofrecer a los aficionados, entrenadores, jugadores y clubes herramientas innovadoras para analizar y entender el rendimiento en el fútbol, más allá de las estadísticas habituales, a través de datos en tiempo real y visualizaciones interactivas, proporcionando así un enfoque más inteligente e interesante hacia el deporte \cite{beyondstats}.

\item \textbf{Aplicación de la inteligencia artificial en la Premier League:}
esta liga utiliza la inteligencia artificial para determinar las probabilidades de que un equipo gane un partido mediante el análisis de un amplio rango de datos. Estos factores abarcan datos históricos de partidos anteriores, como el rendimiento del equipo en casa y fuera de casa, su posición en la tabla de clasificación, su forma actual y lesiones de jugadores clave entre otros. Además, se tienen en cuenta variables más específicas, como la posesión de balón, los tiros a puerta, las oportunidades creadas y la efectividad en la defensa y el ataque. Estos datos son proporcionados a algoritmos de aprendizaje automático que son capaces de analizar patrones complejos y entrenar modelos predictivos para estimar las probabilidades de resultados de los partidos. De esta manera, la inteligencia artificial proporciona una herramienta poderosa para predecir resultados de partidos de fútbol con un alto grado de precisión, lo que puede ser utilizado por equipos, aficionados y casas de apuestas para tomar decisiones justificadas \cite{oracle} \cite{iaFutbol}.

\item \textbf{Opta:}
es una empresa líder en análisis y datos deportivos que es capaz de proporcionar información detallada y estadísticas sobre una amplia gama de eventos deportivos, incluyendo fútbol, rugby, cricket y otros. Para ello, utiliza tecnologías avanzadas de recopilación y análisis de datos donde recopila datos en tiempo real durante los eventos deportivos y los convierte en información valiosa y estadísticas significativas que son utilizadas por equipos, entrenadores, medios de comunicación y aficionados para comprender mejor el juego, evaluar el rendimiento de los jugadores y equipos, y tomar decisiones justificadas. Opta se ha convertido en un recurso fundamental en el mundo del deporte para análisis de datos y seguimiento de estadísticas \cite{opta}.

\end{itemize}
















\section{Estructura de la memoria}

Este documento se estructura de la siguiente forma:
\begin{description}

\item[Capítulo 2 Objetivos del proyecto:] en este capítulo se describen los objetivos que se quieren conseguir con la ejecuccion de este proyecto. Estos se dividen en dos categorías distintas,
los objetitvos de desarrollo y los objetivos academicos.


\item[Capítulo 3 Conceptos teóricos:] en este capítulo se expone una explicación teórica de los conceptos más importantes que se han utilizado para el desarrollo de este proyecto.

\item[Capítulo 4 Técnicas y herramientas:] en este capítulo se describen las tecnicas utilizadas para la obtencion de los datos y las tecnologias utilizadas para el desarrollo del proyecto.


\item[Capítulo 5 Aspectos relevantes del desarrollo del proyecto:] en este capítulo se recogen los aspectos más interesantes del desarrollo del proyecto como los detalles sobre las fases de análisis, diseño e implementación y el tratamiento de los datos.


\item[Capítulo 6 Conclusiones y lineas de trabajo futuras:] en este capítulo se explican las conclusiones finales adquiridas del proyecto junto a las posibles líneas de trabajo futuras a seguir.


\end{description}




