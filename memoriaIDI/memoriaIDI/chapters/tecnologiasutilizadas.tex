\section{Introducción}
En este capítulo se explican las tecnologías utilizadas comentando brevemente sus principales ventajas y su funcionamiento. También se explica el motivo por el qué se han elegido esas tecnologías.

\section{Base de datos}
\subsection{Elección del tipo de base de datos}
Dentro de los diferentes tipos de bases de datos, existen dos principales grupos, las bases de datos relacionales y las no relacionales. La elección de uno de estos tipos para cualquier tipo de proyecto determina en gran medida el éxito del mismo. El contenido deberá estar estructurado y organizado para que se pueda acceder a él eficientemente. Una mala elección de la base de datos puede desembocar en una larga lista de problemas durante el desarrollo de la aplicación, lo que provocaría consecuencias fatales para el proyecto, por lo que es una decisión crítica \cite{bases-diferencias}.

Las bases de datos relacionales \cite{relacional} consisten en una colección ordenada de registros que se organizan en diferentes tablas que se relacionan entre sí, de esta manera se pueden acceder a los contenidos de diferentes tablas y realizar cambios en los datos almacenados sin tener que reorganizar las tablas. Se utiliza SQL \textit{(Structured Query Language)} para acceder a los datos y alterarlos. Estas bases de datos utilizan identificadores únicos para cada tabla, de esta manera se pueden establecer relaciones con otras tablas. Cada fila en una tabla es un registro que se identifica por este identificador único. Los principales sistemas gestores de bases de datos relacionales son MySQL, MariaDB o PostgreSQL entre otros.

Las bases de datos no relacionales \cite{no-relacional} fueron diseñadas para modelos de datos específicos, y que al contrario que las bases de datos relacionales, no necesitan relacionarse con otros modelos. Este tipo de base de datos está cobrando actualmente mucha importancia debido a su acceso a los datos de manera muy sencilla. También utilizan un identificador único para cada registro de cada tabla, pero este identificador no se utiliza para relacionarlo con otro registro de otra tabla, como sucede con las bases de datos relacionales. El formato más habitual para este tipo de base de datos es el documento. Los sistemas gestores de bases de datos principales para las bases de datos no relacionales son MongoDB o Cassandra entre otros.


 En este proyecto, donde hay diferentes tipos de datos que cada uno define una tabla, cada una con varias relaciones con las demás tablas, una base de datos relacional se ajustaría mejor. Esta es la principal razón por la que se ha escogido una base de datos relacional, ya que este proyecto contiene diferentes tipos de datos, con siempre los mismos atributos y cada tipo de dato que representa una tabla, se relaciona con varias tablas más.


Las principales ventajas de un base de datos relacional son:
\begin{itemize}
    \item La sencillez del modelo relacional, lo que permite almacenar grandes cantidades de datos relacionados entre sí.
     \item Garantiza la uniformidad de los datos.
      \item No existe la duplicidad de registros.
       \item Evita conflictos cuando varios usuarios quieren acceder a los mismos datos en el mismo momento.
        \item Buen rendimiento debido a la gran cantidad de herramientas que existen para su uso.
\end{itemize}

Las bases de datos relacionales cumplen con los principios \textit{ACID (Atomicity, Consistency, Isolation and Durability)} \cite{ACID}. Estas son unas propiedades para garantizar la seguridad en las transacciones, es decir, cada operación que se realiza sobre la base de datos. La descripción de estos principios es:
\begin{itemize}
    \item \textbf{Atomicidad:} esta propiedad define que para que una transacción se dé completada, se tienen que haber realizado todas las partes que la componen o ninguna de ellas. De esta forma, si se completan todas las partes de la transacción, se habrá modificado los datos de la manera deseada. Pero si una parte falla, tienen que fallar el resto de las partes.
    \item \textbf{Consistencia:} se refiere a que el sistema debe tener la capacidad de iniciar solo las operaciones que puede concluir, lo que significa que solo pueda ejecutar las partes de una transacción que cumplan las reglas de integridad definidas. Una transacción lleva al sistema de una condición válida a otra que también lo sea. 
    \item \textbf{Aislamiento:} define el momento en el que los cambios realizados por una transacción se harán visibles al resto de transacciones concurrentes. Si se realiza una operación, no se debe afectar a otras, debido a que cada una debe ser ejecutada en un aislamiento total. El estado intermedio de cualquier transacción no debe ser visible a las demás.
    \item \textbf{Durabilidad:} una vez que se realiza una operación, tiene que persistir y no puede ser deshecha incluso si el sistema falla o se presenta un error. Los datos y cambios en una transacción que ha finalizado deben ser persistentes.
\end{itemize}

\subsection{MySQL}
MySQL \cite{MySQL} es el sistema gestor de base de datos relacional que más extendido está en la actualidad, principalmente por estar basado en código abierto. Fue desarrollado originalmente por MySQL AB. Como está basado en código abierto, es accesible por todos los desarrolladores, lo que hace que esté ampliamente extendido con una gran comunidad que ofrece soporte a otros usuarios. Sus principales ventajas son: 
\begin{itemize}
    \item \textbf{Arquitectura cliente y servidor:} cada cliente hace consultas para obtener los datos o realizar modificaciones y el servidor responde realizando estas peticiones.
    \item \textbf{Compatibilidad con SQL:} SQL está generalizado en la industria, y MySQL al ser un estándar tiene plena compatibilidad, lo que facilita las migraciones de base de datos. 
    \item \textbf{Vistas:} se pueden configurar vistas personalizadas.
    \item \textbf{Procedimientos almacenados:} no procesa las tablas directamente, sino utilizando los procedimientos almacenados hace que la eficacia sea mayor.
    \item \textbf{Desencadenantes:} permite automatizar tareas en la base de datos, de forma que cuando se produce un evento, otro evento puede realizar otra funcionalidad.
    \item \textbf{Transacciones:} cada operación en la base de datos debe cumplir los principios ACID antes comentados. 
\end{itemize}

Se ha utilizado MySQL Workbench como herramienta visual para el diseño, administración y gestión de la base de datos MySQL. Se ha utilizado la versión MySQL Workbench 8.0.32.



\section{\textit{Frontend}}
\subsection{Angular}
\label{sec:angular}
Angular \cite{angular} \cite{angular1} es un \textit{framework} de código abierto que es utilizado para crear aplicaciones Web de una sola página llamadas \textit{Single-Page Application (SPA)} \cite{SPA}. Una \textit{Single-Page Application} es una aplicación Web que ejecuta todo su contenido en una sola página, y que funciona mediante la carga de todo el contenido HTML, CSS, Javascript por completo al momento de abrir la Web. Si se pasa a otra sección solo se necesita cargar el nuevo contenido, pero no se tiene que cargar toda la página por completo. Esto provoca mejoras en los tiempos de respuesta y en la agilidad de la navegación que hace que la experiencia del Usuario sea mejor.  


Angular fue creado por Google y su primera versión fue en 2012. Está basado en componentes para que las aplicaciones Web creadas sean escalables, tiene una colección de bibliotecas bien integradas que cubren muchas características, y un gran conjunto de herramientas. 

Está construido sobre Typescript que es un lenguaje programación construido a un nivel superior de Javascript, por lo que lo dota de características adicionales. Todo el código que esté escrito en Javascript sirve para Typescript \cite{typescript}.

Como las aplicaciones Web en angular están basadas en componentes, cada componente contiene un archivo HTML, que será lo que vea el usuario, un archivo CSS para dar estilo a ese componente y un archivo Typescript con la lógica del componente.

Además de los componentes, los elementos principales en la arquitectura de Angular son los módulos, servicios y las directivas.

Las principales ventajas de Angular son:
\begin{itemize}
    \item \textbf{Enlace bidireccional de datos:}  la arquitectura de Angular enlaza Typescript y HTML, por lo tanto, el código de ambos está sincronizado.
    \item \textbf{Directivas:} se amplía la funcionalidad de los archivos HTML con el uso de directivas entre las que destacan ngModel, ngIf y ngFor. 
    \item \textbf{Estructura de código:} contiene plantillas, lo que permite producir aplicaciones con código limpio. 
    \item \textbf{Pruebas:} Angular permite pruebas unitarias y de integración.  
    \item \textbf{Compatibilidad móvil y escritorio:} puede ejecutarse en la mayoría de los navegadores Web, y en tanto equipos de escritorio como dispositivos móviles.
    \item \textbf{Servicios:} permiten compartir información entre componentes y obtener datos del \textit{backend} para mostrarlos en las vistas.
    \item \textbf{Rutas:} permite gestionar las rutas de la aplicación en el \textit{frontend}.
    \item \textbf{Inyección de dependencias:} permite a los componentes consumir servicios mediante la inyección del servicio en el componente.
\end{itemize}

 Debido a que se tiene experiencia previa con Angular por la realización de proyectos previos, y que los componentes y servicios de Angular se ajustaban correctamente a la estructura del proyecto, se eligió Angular como el \textit{framework} para el \textit{frontend}. 

Se utiliza la versión 15.1.4 de Angular para el desarrollo del proyecto.

\subsection{HTML5}
\textit{HyperText Markup Lenguage 5} \cite{HTML} es el componente más básico de la Web y se utiliza para definir el significado y la estructura del contenido que se presenta en la Web. El hipertexto se refiere a los enlaces que se utilizan para conectar páginas Web entre sí y son un aspecto fundamental en la Web. HTML utiliza marcas para etiquetar todo el contenido que se muestra en el navegador Web.

HTML es un estándar a cargo de W3C \textit{(World Wide Web Consortium)} que es la principal organización dedicada a la estandarización de las tecnologías ligadas a la Web.

Se utiliza para este proyecto HTML en su versión 5.

\subsection{CSS}
\textit{Cascading Style Sheets} \cite{CSS} es un lenguaje de estilos que se utiliza para definir la presentación de los documentos HTML, lo que describe como se debe renderizar un elemento en la pantalla del Usuario al visitar una página Web. Tiene una especificación estandarizada por parte de W3C. Un gran avance en los últimos años para este lenguaje ha sido que el alcance de las especificaciones se ha incrementado enormemente, lo que hizo este lenguaje más efectivo para desarrollar.

El significado de \textit{Cascading Style Sheets} hace referencia a que los estilos que se aplican a un elemento HTML son también propagados a los elementos que contiene en cascada. Lo que se crea con este lenguaje son estilos que se aplican a los elementos de la Web y los estilos declarados están en ficheros aparte de los archivos HTML.

Para este proyecto se utiliza CSS en su versión 3.


\subsection{Bootstrap}
Bootstrap \cite{bootstrap} es un \textit{framework} CSS desarrollado por Twitter en 2010 cuyo objetivo era estandarizar las herramientas de la compañía. Combina CSS y Javascript para dar estilos a los elementos que componen una página HTML, de manera que está compuesto por unos archivos CSS y Javascript que asignan las características deseadas a cada elemento. Proporciona una serie de componentes que facilitan el desarrollo con él. Estos componentes se almacenan en una biblioteca que además ayudan a una mejor interacción con el Usuario y, por lo tanto, que la comunicación con él sea mejor.  El objetivo que persigue este \textit{framework} es la construcción de sitios Web \textit{responsive} de manera que se puedan ver en dispositivos móviles, además de en equipos de escritorio y otros dispositivos con diferentes tamaños de pantalla. 

Se ha escogido Bootstrap para el desarrollo de este proyecto para dar estilos a las vistas de la aplicación de manera que se puedan adaptar a cualquier tamaño de pantalla. Se ha elegido esta tecnología debido a la experiencia acumulada con el uso de este \textit{framework}.

Se ha utilizado para este proyecto la versión 5.0 de Bootstrap.



\subsection{ChartJS}
ChartJS \cite{chartJS} es una librería de Javascript de código abierto que permite crear gráficos en páginas Web. No requiere de dependencias externas y tiene la gran ventaja de que se puede integrar con cualquier \textit{framework}. Tiene una amplia documentación y comunidad y al estar basada en Javascript no se requieren más conocimientos específicos para su uso, lo que permite realizar multitud de gráficos sin la necesidad de tener que realizar un gran esfuerzo de aprendizaje. 

ChartJS se utiliza en el proyecto para la realización de un gráfico circular con la información nutricional de los alimentos y se ha escogido debido a su fácil implementación en la aplicación Web. Además al basarse en Javascript no requería aprendizaje, ya que se está familiarizado con este lenguaje de programación. 

Se ha utilizado la versión 4.2.1 de ChartJS en este proyecto

\subsection{Angular Material}
Angular Material \cite{angularMaterial} es una librería utilizada para los proyectos desarrollados en Angular que acelera el desarrollo de las interfaces y hace que estas sean más elegantes. Proporciona componentes ya definidos que son reutilizables y que facilitan la interacción del Usuario con la página Web. Sus componentes están basados en las especificaciones de \textit{Material Design} que es un lenguaje de diseño creado por Google para facilitar el diseño de sitios Web. 

Se escogió Angular Material para realizar ciertos detalles de la aplicación Web debido a las facilidades que ofrece, de manera que Angular Material proporciona componentes ya definidos que requerirían mucho trabajo su creación desde cero simplemente con CSS. Por lo tanto, la utilización de Angular Material ahorra esfuerzo y  proporciona una interfaz mucho más elegante.

Se ha utilizado la versión 15.2.0 de Angular Material en el proyecto.

\section{\textit{Backend}}
\subsection{NodeJS}
NodeJS \cite{nodeJS} es un entorno de ejecución de Javascript, por lo que incluye todo lo necesario para ejecutar cualquier programa que esté escrito en Javascript. Fue creado por los desarrolladores de Javascript, con el objetivo de transformar Javascript que solo se podía ejecutar en navegador a algo que se pudiese ejecutar en el PC. Se ejecuta en el motor de tiempo de ejecución Javascript V8, que es el motor de Javascript que alimenta Google Chrome. La función del motor es transformar el código en Javascript en código máquina rápidamente. NodeJS usa un modelo de entrada y salida sin bloqueo que es controlado por eventos, de esta forma se mantiene liviano y eficiente. Su objetivo no es realizar operaciones intensivas con el procesador, sino para la creación de aplicaciones de red rápidas, ya que es capaz de manejar muchas conexiones simultáneas lo que hace que tenga gran escalabilidad.





Para la parte de \textit{backend}, las dos opciones principales eran NodeJS y Spring Boot. Con Spring Boot sí que se tenía experiencia previa, pero NodeJS es actualmente la tecnología más demandada para la parte de \textit{backend} de las aplicaciones, y aunque no se tuviese tanta experiencia como con Spring boot, al estar relacionados Javascript con NodeJS, y teniendo conocimientos sobre Javascript, la curva de aprendizaje de NodeJS no iba a ser muy elevada.

Se utiliza la versión 18.14.0 de NodeJS para el desarrollo del proyecto.

\subsection{ExpressJS}
ExpressJS \cite{expressJS} es un \textit{framework} de NodeJS que proporciona herramientas para desarrollar aplicaciones de \textit{backend} que sean escalables. Su principal cualidad es que ofrece un sistema de enrutamiento y permite ampliar su funcionalidad con componentes dependiendo el uso que se le vaya a dar a una aplicación. Proporciona un conjunto de herramientas para aplicaciones Web, peticiones y respuestas HTTP, enrutamiento y \textit{middleware} y también incorpora opciones para gestionar sesiones y \textit{cookies}.

Se ha utilizado ExpressJS en el proyecto en la parte de \textit{backend} para crear una API REST. De esta manera, la parte de \textit{frontend} envía peticiones a esta API REST con las operaciones que se quieran realizar.

Se ha utilizado ExpressJS en su versión 4.18.2 para este proyecto.

\section{Desarrollo del proyecto}
\subsection{Git}
Git \cite{git} es el sistema de control de versiones más utilizado actualmente. Es de código abierto y con un mantenimiento activo. Fue creado por Linus Torvalds en 2005. Los desarrolladores globalmente están muy familiarizados con la ventaja de que funciona con una amplia variedad de sistemas operativos y entornos de desarrollo. 

Un sistema de control de versiones \cite{control-versiones} es una herramienta \textit{software} que permite a los equipos de \textit{software} gestionar todos los cambios en el código fuente y así trabajar de manera más rápida y eficiente. Realiza un seguimiento de todas las modificaciones del código en una especie de base de datos, de forma que si ocurre un error se puede volver a una versión anterior.

Git se basa en un repositorio y ramas. El repositorio es un espacio de alojamiento virtual donde se encuentra el código del proyecto. Una rama es una copia exacta del proyecto y se crea a partir de otra rama. Cuando se quiera añadir una nueva funcionalidad o realizar una corrección se genera una nueva rama para encapsular los cambios realizados.

Las principales operaciones en Git son \textit{pull} para descargar al entorno local los últimos cambios de una determinada rama del repositorio, \textit{commit} para guardar los cambios realizados en local y \textit{push} para subir al repositorio remoto los cambios realizados en local guardados en los \textit{commits}. Previamente, \textit{add} permite ver los cambios realizados sobre los archivos en local antes de hacer \textit{commit} para guardar estos cambios. Una vez que se compruebe que una rama tiene la funcionalidad deseada, se realiza \textit{merge} para añadir los cambios realizados en la rama a otra rama principal que recoja todos los avances.

Git cuenta con una arquitectura distribuida, de manera que no tienen un único espacio para toda la historia de versiones del \textit{software}, sino que la copia de trabajo del código de cada desarrollador consiste también en un repositorio que alberga todos los cambios realizados.

Las principales ventajas de Git son el rendimiento, la seguridad y la flexibilidad. Git está diseñado para conservar la integridad del código fuente gestionado, por lo tanto, todos los contenidos almacenados en un repositorio de Git están protegidos con un algoritmo de \textit{hash}. Git además es flexible ya que es compatible con muchos sistemas y proyectos, permite varios tipos de flujos de trabajo y es eficiente tanto para proyectos grandes como pequeños.

Se ha utilizado Git en su versión 2.35.1 para este proyecto.


\subsection{Gitlab}
Gitlab \cite{gitlab} es un servicio web de control de versiones y \textit{DevOps} de código abierto basado en Git. Permite el desarrollo colaborativo y permite gestionar, crear, administrar y conectar repositorios de Git. Además permite a los desarrolladores gestionar y realizar las diferentes tareas del proyecto mediante múltiples funcionalidades que aumentarán la eficiencia y velocidad de trabajo en el proyecto. Por lo tanto, sirve tanto como sistema de control de versiones como herramienta de gestión de proyectos.


Se ha utilizado Gitlab como herramienta de gestión del repositorio de Git y como herramienta de gestión del proyecto para controlar las tareas de desarrollo debido a la gran cantidad de herramientas que proporciona y que la Escuela de Ingeniería Informática proporciona licencia para su uso. 





\subsection{Astah}
Astah \cite{astah} es una herramienta de modelado UML que permite crear gran variedad de diagramas con las necesidades del sistema que se quiera construir. Contiene numerosas herramientas para que la creación de estos diagramas sea más eficiente. Ha sido utilizada para realizar los Diagramas UML y la Escuela de Ingeniería Informática proporciona licencia a los estudiantes en versión profesional lo que da numerosas ventajas.

En este proyecto se ha utilizado la versión de Astah Proffesional 9.0.



\subsection{Railway}
Railway \cite{railway} es una plataforma como servicio que se utiliza para desplegar en la nube las aplicaciones desarrolladas. Una plataforma como servicio \cite{paas} consiste en un conjunto de servicios que están basados en la nube y que permite a los desarrolladores crear aplicaciones sin preocuparse por la configuración y el mantenimiento de los servidores, de manera que se tienen que centrar únicamente en crear la mejor experiencia para el Usuario posible.

Railway ha sido la tecnología utilizada para el despliegue de la aplicación ya que permite el despliegue tanto de la parte de \textit{frontend} como la de \textit{backend} y proporciona una versión gratuita con las capacidades suficientes como para desplegar la aplicación desarrollada.

\subsection{Cloudinary}
Cloudinary \cite{cloudinary} provee un servicio para almacenar imágenes y vídeos en la nube. Permite a sus usuarios almacenar, modificar y recuperar imágenes y vídeos para que se puedan utilizar en una aplicación. Este servicio en la nube almacena las imágenes en su servidor y permite el acceso a ellas mediante una URL personalizada. La cuenta gratuita permite almacenar fotos de hasta 10MB.

Se escogió Cloudinary para el almacenamiento de las fotos de la aplicación debido a que es una herramienta ampliamente utilizada por los desarrolladores, por lo que cuenta con una extensa comunidad y la amplia capacidad de almacenamiento que proporciona su plan gratuito.


\subsection{JWT}
\textit{Json Web Token} \cite{jwt} es un estándar abierto RFC 7519 en el que se define un método para comunicar datos entre dos entidades de manera segura. De esta forma se puede transmitir información de manera segura que es verificada ya que se firma de manera virtual. Está compuesto por demandas en las que se realiza la transmisión de la información de una entidad a otra y la estructura de un \textit{Json Web Token} es una cadena compuesta de tres partes separadas por un punto que se serializa. Las partes son las siguientes:

\begin{itemize}
    \item \textit{Header}: es el primer componente y está compuesto por el tipo del \textit{token} que en este caso es JWT y el algoritmo utilizado que en este caso es SHA256.
    \item \textit{Payload}: es donde se encuentran las demandas antes comentadas del \textit{token}. Generalmente, son sobre una entidad y otra información asociada y pueden ser registradas, públicas o privadas. Las registradas no son obligatorias y las privadas sirven para compartir información. 
    \item \textit{Signature}: es donde se debe firmar el \textit{header} y el \textit{payload} codificado, el \textit{secret} y el algoritmo que se definió en el \textit{header}. Esta parte tiene como finalidad verificar que no haya ningún cambio en el contenido.
\end{itemize}

Las principales ventajas son su tamaño compacto y por lo tanto se puede utilizar para cualquier tipo de datos. Además es autocontenido, de manera que tiene toda la información necesaria en su interior y que se puede utilizar en varias plataformas. 

Su uso principal es para el intercambio de información entre dos sitios, de manera que, al ser firmado digitalmente, se puede verificar la información. En este proyecto se ha utilizado para la autenticación de usuarios de manera que las contraseñas se encripten para aumentar la seguridad.

\subsection{Trello}
Trello \cite{trello} es una aplicación para gestionar proyectos en línea que ayuda a equipos que pueden estar compuestos por grandes números de personas y en diferentes localizaciones, organizar y priorizar sus tareas para así colaborar en tiempo real hacia la consecución del proyecto. Su interfaz utiliza tableros y tarjetas para representar cada proyecto y tarea lo que ayuda a que se tenga una buena visibilidad de lo que falta por hacer y una mejor organización. Debido a estas ventajas es utilizado por pequeñas \textit{startups} hasta grandes empresas del sector. Trello implementa un tablero Kanban.

Se ha escogido esta herramienta para la organización de las tareas a realizar en el proyecto debido a su interfaz gráfica que facilita la rápida visualización de qué tareas faltan por hacer. Además su fácil acceso desde cualquier sitio por cualquier persona relacionada con el proyecto y su experiencia previa con la utilización de esta tecnología han sido determinantes en su elección.




\subsection{Visual Studio Code}
Visual Studio Code \cite{vsc} es un editor de código fuente desarrollado por Microsoft para Windows, MacOS, Linux y Web. Facilita el desarrollo de código mediante múltiples herramientas como el soporte para la depuración, control integrado de Git, resaltado de sintaxis o refactorizaciones de código. Es altamente personalizable, lo que permite que se pueda adaptar y contiene gran cantidad de extensiones lo que facilitan la tarea de desarrollo. Es de código abierto y cuenta con una gran comunidad de usuarios, debido a que es uno de los editores más utilizado entre los desarrolladores. 

Se ha utilizado este editor de código para todo el proyecto, tanto la parte de \textit{backend} como \textit{frontend} como la memoria. Para la parte de \textit{frontend} en Angular, Visual Studio Code proporciona numerosas extensiones que facilitan el desarrollo en este \textit{framework}. Para la parte de \textit{backend} en NodeJS, Visual Studio Code al contar con línea de comandos integrada facilita la depuración y corrección de los errores. Finalmente para la memoria, Visual Studo Code permite la edición de archivos en Latex.



\subsection{Balsamiq}
Balsamiq Wireframes \cite{balsamiq} es una herramienta \textit{online} que se utiliza para crear \textit{wireframes}, es decir, esquemas de las páginas que contendrá la aplicación Web y que servirán como guía del esqueleto que tendrán las interfaces de la aplicación Web.

Se ha utilizado para crear los bocetos de las interfaces que contendrá la aplicación Web debido a que se ha trabajado previamente con esta herramienta y que es de gran popularidad entre los desarrolladores para crear bocetos de interfaces que luego se plasmen en una aplicación. Además contiene gran cantidad de herramientas que pueden ayudar a la creación de interfaces que luego hagan que la experiencia del Usuario sea mejor al utilizar la aplicación Web. 







