\section{Introducción}
En este capítulo se exponen los actores que se relacionarán con el sistema y se describen los diferentes Casos de Uso que se podrán realizar. Se especifican los Requisitos del Sistema tanto Funcionales, No Funcionales y de Información.


\section{Requisitos Funcionales}

Los Requisitos Funcionales son declaraciones de cómo se comporta el sistema, es decir, define lo que el sistema debe hacer para satisfacer las necesidades de los clientes. Es la funcionalidad que tendrá el sistema que se va a diseñar y proporcionan una descripción de cómo el sistema responderá ante las acciones de los usuarios \cite{requisitos}.

Los Requisitos Funcionales para este sistema son: 

\begin{itemize}
    \item \textbf{RF-01} El sistema deberá permitir a un Usuario registrarse en la aplicación Web.
 \item \textbf{RF-02} El sistema deberá permitir a un Usuario identificarse en la aplicación Web.
 \item \textbf{RF-03} El sistema deberá permitir a un Usuario consultar las publicaciones de los usuarios a los que sigue.
 \item \textbf{RF-04} El sistema deberá permitir a un Usuario seguir a otro Usuario.
 \item \textbf{RF-05} El sistema deberá permitir a un Usuario dejar de seguir a un Usuario.
 \item \textbf{RF-06} El sistema deberá permitir a un Usuario crear una receta.
 \item \textbf{RF-07} El sistema deberá permitir a un Usuario consultar recetas por nombre.
 \item \textbf{RF-08} El sistema deberá permitir a un Usuario consultar las publicaciones enlazadas a una receta.
 \item \textbf{RF-09} El sistema deberá permitir a un Usuario consultar los detalles de una publicación.
 \item \textbf{RF-10} El sistema deberá permitir a un Usuario subir una publicación.
 \item \textbf{RF-11} El sistema deberá permitir a un Usuario consultar los datos de su cuenta.
 \item \textbf{RF-12} El sistema deberá permitir a un Usuario editar los datos de su cuenta.
 \item \textbf{RF-13} El sistema deberá permitir a un Usuario añadir una receta a favoritas.
 \item \textbf{RF-14} El sistema deberá permitir a un Usuario consultar sus recetas favoritas.
 \item \textbf{RF-15} El sistema deberá permitir a un Usuario eliminar una receta de sus favoritas.
 \item \textbf{RF-16} El sistema deberá permitir a un Usuario consultar los detalles de otro Usuario.
 \item \textbf{RF-17} El sistema deberá permitir a un Usuario consultar los detalles de una receta.
 \item \textbf{RF-18} El sistema deberá permitir a un Usuario dejar un comentario en una publicación.
 \item \textbf{RF-19} El sistema deberá permitir a un Usuario consultar alimentos por nombre.
 \item \textbf{RF-20} El sistema deberá permitir a un Usuario consultar las publicaciones enlazadas a un alimento.
  \item \textbf{RF-21} El sistema deberá permitir al Administrador eliminar un Usuario.
  \item \textbf{RF-22} El sistema deberá permitir al Administrador eliminar un alimento.
  \item \textbf{RF-23} El sistema deberá permitir al Administrador eliminar una receta.
  \item \textbf{RF-24} El sistema deberá permitir al Administrador eliminar una publicación.
  \item \textbf{RF-25} El sistema deberá permitir al Administrador editar un alimento.
  \item \textbf{RF-26} El sistema deberá permitir al Administrador crear un alimento.
   \item \textbf{RF-27} El sistema deberá mostrar al Administrador una tabla con todos los alimentos almacenados.
   \item \textbf{RF-28} El sistema deberá mostrar al Administrador una tabla con todos los usuarios registrados.
   \item \textbf{RF-29} El sistema deberá mostrar al Administrador una tabla con todas las recetas creadas por los usuarios.
   \item \textbf{RF-30} El sistema deberá mostrar al Administrador una tabla con todas las publicaciones creadas por los usuarios.
   \item \textbf{RF-31} El sistema deberá permitir a un Usuario cerrar su sesión.

\end{itemize}

\section{Requisitos No Funcionales}

Los Requisitos No Funcionales son las restricciones impuestas al sistema y especifican atributos de calidad del \textit{software} entre otros. Se ocupan de problemas como la escalabilidad, mantenibilidad, rendimiento, portabilidad y seguridad y por lo tanto abordan cuestiones vitales de calidad para los sistemas \textit{software} \cite{requisitos}.

Los Requisitos No Funcionales para este sistema son: 

\begin{itemize}
    
\item \textbf{RNF-01} El sistema debe estar implementado con Angular para la parte de \textit{frontend}.
\item \textbf{RNF-02} El sistema debe poderse desplegar en Railway.
\item \textbf{RNF-03} El sistema debe poderse utilizar en cualquier dispositivo.
\item \textbf{RNF-04} El sistema debe utilizar una base de datos relacional SQL.
\item \textbf{RNF-05} El sistema debe adaptarse a cualquier tamaño de pantalla.
\item \textbf{RNF-06 }El sistema debe almacenar las contraseñas encriptadas en la base de datos.
\item \textbf{RNF-07} El sistema debe tener un diseño de interfaz de usuario fácil de aprender.
\item \textbf{RNF-08} El sistema debe tener un flujo de navegación sencillo.
\item \textbf{RNF-09} El sistema debe tener una interfaz sencilla de utilizar.
\item \textbf{RNF-10} El sistema debe ser accesible desde los principales navegadores Web.
\item \textbf{RNF-11} El sistema debe almacenar todos los datos de la aplicación Web de manera segura.

\end{itemize}


\section{Requisitos de Información}



Los Requisitos de Información describen la información que se va a almacenar y gestionar en el sistema \cite{requisitos-informacion}. Los Requisitos de Información para este sistema son: 

\begin{itemize}
    
\item \textbf{RI-01} El sistema debe almacenar la información correspondiente a cada usuario, en concreto: identificador(id), \textit{username}, \textit{password}, foto, \textit{email}, rol y descripción.
\item \textbf{RI-02} El sistema debe almacenar la información correspondiente a cada receta, en concreto: identificador(id), resumen, tiempo, título, foto, dificultad e identificador(id) del Usuario creador.
\item \textbf{RI-03} El sistema debe almacenar la información correspondiente a cada alimento, en concreto: identificador(id), nombre, descripción, foto, calorías, grasas, carbohidratos, proteínas, cantidad, medida y enlace.
\item \textbf{RI-04} El sistema debe almacenar la información correspondiente a cada comentario, en concreto: identificador(id), identificador(id) del Usuario que lo realizó, identificador(id) de la publicación a la que pertenece y comentario.
\item \textbf{RI-05} El sistema debe almacenar la información correspondiente a cada seguidor, en concreto: identificador(id), identificador(id) del Usuario que ejerce como seguidor e identificador(id) del Usuario que ejerce como seguido.
\item \textbf{RI-06} El sistema debe almacenar la información correspondiente a cada receta favorita, en concreto: identificador(id), identificador(id) del Usuario al que corresponde e identificador(id) de la receta favorita.
\item \textbf{RI-07} El sistema debe almacenar la información correspondiente a cada alimento en receta, en concreto: identificador(id), identificador(id) del alimento al que corresponde, identificador(id) de la receta a la que pertenece, cantidad y medida.
\item \textbf{RI-08} El sistema debe almacenar la información correspondiente a cada paso, en concreto: identificador(id), identificador(id) de la receta a la que pertenece, paso y orden.
\item \textbf{RI-09} El sistema debe almacenar la información correspondiente a cada publicación, en concreto: identificador(id), descripción, foto, fecha de publicación, título, identificador(id) del Usuario que la creó, identificador(id) del alimento con la que está enlazada e identificador(id) de la receta con la que está enlazada.


\end{itemize}

\section{Actores Principales}
Se han identificado dos actores que se relacionarán con el sistema, el Usuario y el Administrador. A continuación, se define cada uno de ellos.

\subsection{Usuario}

El Actor Principal de la aplicación Web es el Usuario. El Usuario es un cliente de la aplicación que la utiliza tanto para crear contenidos como recetas o publicaciones como para consumir contenido de otros usuarios. Por lo tanto, el Usuario puede producir contenido mediante la creación de una receta añadiendo toda la información relevante para esta receta para que otros usuarios la puedan realizar, como son los alimentos para su preparación, sus pasos y su dificultad. También puede producir contenido mediante la creación de una publicación enlazada a un alimento o receta, incluyendo en la publicación información sobre su experiencia que puede ser de gran ayuda a otros usuarios. 

Por otro lado, el Usuario puede consumir contenido consultando publicaciones de otros usuarios y viendo los detalles de estas publicaciones. Puede consultar las recetas creadas por otros usuarios con todos los detalles para su preparación y de esta manera obtener la información necesaria para prepararlas. Puede añadir las recetas que desee a favoritas para así poder acceder a su información de una manera más rápida. Puede ver los alimentos almacenados en la aplicación con toda su información relacionada para así poder encontrar información valiosa sobre ese alimento. Además, cada Usuario puede ver los perfiles de los demás usuarios para así obtener información sobre otros usuarios, seguir a los usuarios que más le gusten y poder estar al tanto de sus publicaciones.

\subsection{Administrador}

El Administrador de la aplicación se encargará de que todos los datos que muestre la aplicación y que se almacenen en la base de datos sean adecuados a este tipo de aplicación y cumplan las normas de convivencia entre los usuarios de la aplicación. De esta manera, podrá gestionar todos los contenidos almacenados en la aplicación Web, como pueden ser las recetas, publicaciones, usuarios y alimentos y eliminar los contenidos que no sean adecuados para esta aplicación Web. Además, tiene la capacidad de crear los alimentos que se muestran en la aplicación para que así los usuarios de la aplicación los puedan usar en sus recetas y puedan ver su información relevante.


\section{Diagrama de Casos de Uso}
Los Casos de Uso son la descripción de una acción o actividad que se puede realizar en el sistema para llevar a cabo un proceso \cite{casos-uso}. Mediante la identificación y descripción de los Casos de Uso se definen las acciones más relevantes que se ejecutarán en el sistema y cómo se comportará durante la realización de estas acciones.

En la Figura \ref{fig:casos-uso-usuario-1}  se muestra la primera parte del Diagrama de Casos de Uso para el actor Usuario.


\begin{figure}
   \centering
   \begin{normalsize}
       \import{svg/}{casos1-escalado.pdf_tex}
   \end{normalsize}
   \caption{Primera parte del Diagrama de Casos de Uso para el actor Usuario.}
    \label{fig:casos-uso-usuario-1}

\end{figure}

En la Figura \ref{fig:casos-uso-usuario-2}  se muestra la segunda parte del Diagrama de Casos de Uso para el actor Usuario.


\begin{figure}
   \centering
   \begin{normalsize}
       \import{svg/}{casos2-escalado.pdf_tex}
   \end{normalsize}
   \caption{Segunda parte del Diagrama de Casos de Uso para el actor Usuario.}
    \label{fig:casos-uso-usuario-2}

\end{figure}

En la Figura \ref{fig:casos-uso-admin}  se muestra el Diagrama de Casos de Uso para el actor Administrador.


\begin{figure}
   \centering
   \begin{normalsize}
       \import{svg/}{casos3-escalado.pdf_tex}
   \end{normalsize}
   \caption{Diagrama de Casos de Uso para el actor Administrador.}
    \label{fig:casos-uso-admin}

\end{figure}






\section{Descripción Casos de Uso}
 En la Descripción de los Casos de Uso se especificará el flujo normal de los principales Casos de Uso, la realización exitosa de una acción o actividad y aquellos flujos alternativos a la hora de realizar cada uno de los Casos de Uso.


Las Tablas \ref{tab:caso_uso_identificarse} a \ref{tab:caso_uso_crear_alimento} muestran las Descripciones de los Casos de Uso.

%CU-1 Identificarse
\begin{table}[H]
\begin{tabularx}{1\linewidth}{P}
\toprule
\textbf{Caso de uso:} &  CU-1 Iniciar sesión\\ \midrule
\textbf{Descripción:} & El Usuario introduce sus datos para identificarse en el sistema y así poder acceder a su funcionalidad.\\ \hline
\textbf{Actores:} & Usuario y Administrador.\\ \hline
\textbf{Pre-condiciones:} & El Usuario no se ha identificado en el sistema.\\ \hline
\textbf{Post-condiciones:} & El Usuario se ha identificado en el sistema y puede acceder a su funcionalidad.  \\ \midrule
\multicolumn{2}{c}{\textbf{Secuencia normal}}\\ \midrule
\multicolumn{2}{Z}{\hspace{0.5cm}\textbf{1.} El Usuario comienza el proceso para iniciar sesión.}\\ 
\multicolumn{2}{Z}{\hspace{.5cm}\textbf{2.} El sistema solicita \textit{username} y \textit{password}. }\\ 
\multicolumn{2}{Z}{\hspace{0.5cm}\textbf{3.} El Usuario introduce los datos. }\\
\multicolumn{2}{Z}{\hspace{0.5cm}\textbf{4.} El sistema valida el formato de los datos introducidos.}\\ 
\multicolumn{2}{Z}{\hspace{0.5cm}\textbf{5.} El sistema solicita confirmación.}\\ 
\multicolumn{2}{Z}{\hspace{0.5cm}\textbf{6.} El Usuario confirma el inicio de sesión.}\\ 
\multicolumn{2}{Z}{\hspace{0.5cm}\textbf{7.} El sistema comprueba que los datos introducidos corresponden a los de un Usuario registrado.}\\ 
\multicolumn{2}{Z}{\hspace{0.5cm}\textbf{8.} El sistema informa al Usuario que ha iniciado sesión con éxito.}\\ 
\multicolumn{2}{Z}{\hspace{0.5cm}\textbf{9.} El sistema carga la información con la que el Usuario se ha identificado.}\\

\multicolumn{2}{Z}{\hspace{0.5cm}\textbf{   } \textbf{ Punto de inclusión: CU-5 Ver publicaciones usuarios seguidos.}}\\ 

 

\multicolumn{2}{Z}{\hspace{0.5cm}\textbf{11.} El Caso de Uso finaliza.}\\ \midrule

\multicolumn{2}{c}{\textbf{Secuencias alternativas}}\\ \midrule
\multicolumn{2}{Z}{\hspace{0.5cm}\textbf{3.a, 6.a)} El Usuario cancela el proceso para identificarse y el Caso de Uso queda sin efecto.}\\ \hline


\multicolumn{2}{c}{\textbf{Excepciones}}\\ \midrule

\multicolumn{2}{Z}{\hspace{0.5cm}\textbf{4.a)} El sistema comprueba que el formato de los datos es incorrecto, el sistema informa al Usuario, el Caso de Uso continúa en el paso 3. }\\ 
\multicolumn{2}{Z}{\hspace{0.5cm}\textbf{7.a)} El sistema comprueba que los datos introducidos no corresponden a ningún Usuario, el sistema informa al Usuario, el Caso de Uso continúa en el paso 3.}\\ \hline
\bottomrule
\end{tabularx}
\caption{Descripción del Caso de Uso \textit{Iniciar sesión}.} \label{tab:caso_uso_identificarse}
\end{table}



%CU-2 Registrarse
\begin{table}[H]
\begin{tabularx}{1\linewidth}{P}
\toprule
\textbf{Caso de uso:} &  CU-2 Registrarse\\ \midrule
\textbf{Descripción:} & Un Usuario nuevo desea registrarse en la aplicación para poder acceder a toda la funcionalidad.\\ \hline
\textbf{Actores:} & Usuario. \\ \hline
\textbf{Pre-condiciones:} & El Usuario no se ha identificado en el sistema.\\ \hline
\textbf{Post-condiciones:} & El Usuario se ha registrado en el sistema y puede identificarse en él mediante su \textit{username} y \textit{password} para acceder a la funcionalidad.  \\ \midrule
\multicolumn{2}{c}{\textbf{Secuencia normal}}\\ \midrule
\multicolumn{2}{Z}{\hspace{0.5cm}\textbf{1.} El Usuario comienza el proceso para registrarse en la aplicación.}\\ 
\multicolumn{2}{Z}{\hspace{.5cm}\textbf{2.} El sistema solicita \textit{username}, \textit{password} y \textit{email} al Usuario. }\\ 
\multicolumn{2}{Z}{\hspace{0.5cm}\textbf{3.} El Usuario introduce los datos solicitados. }\\
\multicolumn{2}{Z}{\hspace{0.5cm}\textbf{4.} El sistema valida el formato de los datos solicitados.}\\
\multicolumn{2}{Z}{\hspace{0.5cm}\textbf{5.} El sistema comprueba que el \textit{email} no ha sido utilizado por otro Usuario.}\\ 
\multicolumn{2}{Z}{\hspace{0.5cm}\textbf{6.} El sistema comprueba que el \textit{username} no ha sido utilizado por otro Usuario.}\\ 
\multicolumn{2}{Z}{\hspace{0.5cm}\textbf{7.} El sistema solicita confirmación para el registro.}\\ 
\multicolumn{2}{Z}{\hspace{0.5cm}\textbf{8.} El Usuario confirma el registro.}\\ 

\multicolumn{2}{Z}{\hspace{0.5cm}\textbf{9.} El sistema informa al Usuario de que se ha registrado con éxito.}\\ 
\multicolumn{2}{Z}{\hspace{0.5cm}\textbf{10.} El Caso de Uso finaliza.}\\ \midrule

\multicolumn{2}{c}{\textbf{Secuencias alternativas}}\\ \midrule
\multicolumn{2}{Z}{\hspace{0.5cm}\textbf{3.a, 8.a)} El Usuario cancela el proceso de registro y el Caso de Uso queda sin efecto.}\\ \hline


\multicolumn{2}{c}{\textbf{Excepciones}}\\ \midrule

\multicolumn{2}{Z}{\hspace{0.5cm}\textbf{4.a)} El sistema comprueba que el formato de los datos es incorrecto, el sistema informa al Usuario, el Caso de Uso continúa en el paso 3. }\\ 
\multicolumn{2}{Z}{\hspace{0.5cm}\textbf{5.a)} El sistema comprueba que el \textit{email} ha sido utilizado por otro Usuario, el sistema informa al Usuario, el Caso de Uso continúa en el paso 3.}\\ 
\multicolumn{2}{Z}{\hspace{0.5cm}\textbf{6.a)} El sistema comprueba que el \textit{username} ha sido utilizado por otro Usuario, el sistema informa al Usuario, el Caso de Uso continúa en el paso 3.}\\
\hline
\bottomrule
\end{tabularx}
\caption{Descripción del Caso de Uso \textit{Registrarse}.} \label{tab:caso_uso_registrarse}
\end{table}


%CU-3 Cerrar sesion
\begin{table}[H]
\begin{tabularx}{1\linewidth}{P}
\toprule
\textbf{Caso de uso:} &  CU-3 Cerrar sesión\\ \midrule
\textbf{Descripción:} & Un Usuario identificado en el sistema quiere salir de la aplicación y cerrar su sesión.\\ \hline
\textbf{Actores:} & Usuario y Administrador. \\ \hline
\textbf{Pre-condiciones:} & El Usuario se ha identificado en el sistema y tiene una sesión activa.\\ \hline
\textbf{Post-condiciones:} & Se ha cerrado la sesión del Usuario y ya no podrá acceder a la funcionalidad del sistema hasta que se vuelva a identificar.  \\ \midrule
\multicolumn{2}{c}{\textbf{Secuencia normal}}\\ \midrule
\multicolumn{2}{Z}{\hspace{0.5cm}\textbf{1.} El Usuario solicita cerrar sesión.}\\ 
\multicolumn{2}{Z}{\hspace{.5cm}\textbf{2.} El sistema solicita confirmación. }\\ 
\multicolumn{2}{Z}{\hspace{0.5cm}\textbf{3.} El Usuario confirma el cierre de sesión. }\\
\multicolumn{2}{Z}{\hspace{0.5cm}\textbf{4.} El sistema cierra sesión y elimina la información sobre el Usuario que haya cargado.}\\ 
\multicolumn{2}{Z}{\hspace{0.5cm}\textbf{5.} El Caso de Uso finaliza.}\\ \midrule

\multicolumn{2}{c}{\textbf{Secuencias alternativas}}\\ \midrule
\multicolumn{2}{Z}{\hspace{0.5cm}\textbf{3.a)} El Usuario cancela el proceso de cerrar sesión y el Caso de Uso queda sin efecto.}\\ \hline




\hline
\bottomrule
\end{tabularx}
\caption{Descripción del Caso de Uso \textit{Cerrar sesión}.} \label{tab:caso_uso_cerrar_sesion}
\end{table}





%CU-4 Editar información del usuario
\begin{table}[H]
\begin{tabularx}{1\linewidth}{P}
\toprule
\textbf{Caso de uso:} &  CU-4 Editar información del Usuario\\ \midrule
\textbf{Descripción:} & El Usuario desea cambiar la información de su perfil y los datos de su cuenta\\ \hline
\textbf{Actores:} & Usuario. \\ \hline
\textbf{Pre-condiciones:} &  El Usuario se ha identificado en el sistema.\\ \hline
\textbf{Post-condiciones:} &  La información del perfil del Usuario y los datos de su cuenta se habrán actualizado con los nuevos campos introducidos. \\ \midrule
\multicolumn{2}{c}{\textbf{Secuencia normal}}\\ \midrule
\multicolumn{2}{Z}{\hspace{0.5cm}\textbf{1.} El Usuario selecciona editar la información de su cuenta.}\\ 
\multicolumn{2}{Z}{\hspace{.5cm}\textbf{2.}  El sistema muestra la información del perfil del Usuario y los datos de su cuenta.}\\ 
\multicolumn{2}{Z}{\hspace{0.5cm}\textbf{3.}  El sistema solicita los nuevos valores para estos campos.}\\
\multicolumn{2}{Z}{\hspace{0.5cm}\textbf{4.} El Usuario introduce los nuevos valores.}\\ 
\multicolumn{2}{Z}{\hspace{0.5cm}\textbf{5.} El sistema valida el formato de los nuevos valores.}\\ 
\multicolumn{2}{Z}{\hspace{0.5cm}\textbf{6.} El sistema solicita confirmación.}\\ 
\multicolumn{2}{Z}{\hspace{0.5cm}\textbf{7.} El Usuario confirma el cambio.}\\ 
\multicolumn{2}{Z}{\hspace{0.5cm}\textbf{8.} El sistema guarda la nueva información del Usuario. }\\ 
\multicolumn{2}{Z}{\hspace{0.5cm}\textbf{9.} El sistema informa al Usuario de que la operación se ha realizado con éxito.}\\ 
\multicolumn{2}{Z}{\hspace{0.5cm}\textbf{10.} El Caso de Uso finaliza. }\\ \midrule

\multicolumn{2}{c}{\textbf{Secuencias alternativas}}\\ \midrule
\multicolumn{2}{Z}{\hspace{0.5cm}\textbf{4.a,7.a)} El Usuario cancela el proceso de editar información de su cuenta y el Caso de Uso queda sin efecto. }\\ \hline


\multicolumn{2}{c}{\textbf{Excepciones}}\\ \midrule

\multicolumn{2}{Z}{\hspace{0.5cm}\textbf{5.a)} El sistema comprueba que el formato de los datos introducidos es incorrecto, el sistema informa al Usuario, el Caso de Uso continúa en el paso 4. }\\ 

\hline
\bottomrule
\end{tabularx}
\caption{Descripción del Caso de Uso \textit{Editar información del Usuario}.} \label{tab:caso_uso_editar_usuario}
\end{table}





%CU-5 Ver publicaciones usuarios seguidos
\begin{table}[H]
\begin{tabularx}{1\linewidth}{P}
\toprule
\textbf{Caso de uso:} &  CU-5 Ver publicaciones usuarios seguidos\\ \midrule
\textbf{Descripción:} & El Usuario desea ver las últimas publicaciones de los usuarios a los que sigue.\\ \hline
\textbf{Actores:} & Usuario. \\ \hline
\textbf{Pre-condiciones:} & El Usuario está identificado en el sistema.\\ \hline
\textbf{Post-condiciones:} & El Usuario podrá ver las publicaciones de los usuarios a los que sigue ordenadas por fecha de publicación mostrando las más recientes primero. \\ \midrule
\multicolumn{2}{c}{\textbf{Secuencia normal}}\\ \midrule
\multicolumn{2}{Z}{\hspace{0.5cm}\textbf{1.} El Usuario selecciona ver las últimas publicaciones de los usuarios a los que sigue. }\\ 
\multicolumn{2}{Z}{\hspace{.5cm}\textbf{2.} El sistema busca los usuarios a los que sigue.  }\\ 
\multicolumn{2}{Z}{\hspace{0.5cm}\textbf{3.} El sistema busca las publicaciones de los usuarios a los que sigue.  }\\
\multicolumn{2}{Z}{\hspace{0.5cm}\textbf{4.} El sistema ordena las publicaciones por fecha de publicación colocando las más recientes primero. }\\ 
\multicolumn{2}{Z}{\hspace{0.5cm}\textbf{5.} El sistema muestra las publicaciones de los usuarios a los que sigue ordenadas por fecha de publicación al Usuario.}\\ 
\multicolumn{2}{Z}{\hspace{0.5cm}\textbf{6.} El Caso de Uso finaliza.}\\ 
 \midrule

 


\multicolumn{2}{c}{\textbf{Excepciones}}\\ \midrule

\multicolumn{2}{Z}{\hspace{0.5cm}\textbf{2.a)} El sistema detecta que el Usuario no sigue a ningún otro usuario, el sistema informa al Usuario, el Caso de Uso finaliza. }\\ 
\multicolumn{2}{Z}{\hspace{0.5cm}\textbf{3.a)} El sistema detecta que no hay publicaciones de los usuarios a los que sigue, el sistema informa al Usuario, el Caso de Uso finaliza.}\\ 

\hline
\bottomrule
\end{tabularx}
\caption{Descripción del Caso de Uso \textit{Ver publicaciones usuarios seguidos}.} \label{tab:caso_uso_ver_publicaciones_seguidos}
\end{table}


%CU-6 Crear receta
\begin{table}[H]
\begin{tabularx}{1\linewidth}{P}
\toprule
\textbf{Caso de uso:} &  CU-6 Crear receta\\ \midrule
\textbf{Descripción:} & El Usuario desea registrar una receta en la aplicación para que así todos los usuarios la puedan ver y puedan realizarla.\\ \hline
\textbf{Actores:} & Usuario \\ \hline
\textbf{Pre-condiciones:} & El Usuario se ha identificado en el sistema.\\ \hline
\textbf{Post-condiciones:} & La receta se ha guardado en el sistema y su información es accesible por todos los usuarios de la aplicación. \\ \midrule
\multicolumn{2}{c}{\textbf{Secuencia normal}}\\ \midrule
\multicolumn{2}{Z}{\hspace{0.5cm}\textbf{1.} El Usuario selecciona crear una receta. }\\ 
\multicolumn{2}{Z}{\hspace{.5cm}\textbf{2.} El sistema solicita título, resumen, tiempo de preparación, foto y dificultad. }\\ 
\multicolumn{2}{Z}{\hspace{0.5cm}\textbf{3.} El Usuario introduce los datos solicitados.  }\\
\multicolumn{2}{Z}{\hspace{0.5cm}\textbf{4.} El sistema valida el formato de los datos introducidos. }\\ 
\multicolumn{2}{Z}{\hspace{0.5cm}\textbf{5.} El sistema solicita los pasos para realizar la receta. }\\ 
\multicolumn{2}{Z}{\hspace{0.5cm}\textbf{6.} El Usuario introduce los pasos para realizar la receta.}\\ 
\multicolumn{2}{Z}{\hspace{0.5cm}\textbf{7.} El sistema valida que el formato de los pasos introducidos sea correcto.}\\ 
\multicolumn{2}{Z}{\hspace{0.5cm}\textbf{8.} El sistema solicita los alimentos para la preparación de la receta. }\\ 
\multicolumn{2}{Z}{\hspace{0.5cm}\textbf{  } \textbf{ Punto de inclusión: CU-7 Buscar alimento.}}\\
\multicolumn{2}{Z}{\hspace{0.5cm}\textbf{10.} El Usuario selecciona uno de alimentos encontrados en la búsqueda. }\\ 
\multicolumn{2}{Z}{\hspace{0.5cm}\textbf{11.} El sistema solicita la cantidad de ese alimento utilizada en la preparación de la receta.}\\
\multicolumn{2}{Z}{\hspace{0.5cm}\textbf{12.} El Usuario introduce la cantidad del alimento necesaria para la preparación de la receta. }\\
\multicolumn{2}{Z}{\hspace{0.5cm}\textbf{13.} El sistema valida que el formato de la cantidad introducida sea correcta.}\\
\multicolumn{2}{Z}{\hspace{0.5cm}\textbf{14.} El sistema solicita confirmación para crear la receta. }\\
\multicolumn{2}{Z}{\hspace{0.5cm}\textbf{15.} El Usuario confirma la creación de la receta. }\\
\multicolumn{2}{Z}{\hspace{0.5cm}\textbf{16.} El sistema guarda la información de la receta. }\\
\multicolumn{2}{Z}{\hspace{0.5cm}\textbf{17.} El Caso de Uso finaliza. }\\
 \midrule

\multicolumn{2}{c}{\textbf{Secuencias alternativas}}\\ \midrule
\multicolumn{2}{Z}{\hspace{0.5cm}\textbf{3.a,6.a,10.a,12.a,15.a)} El Usuario cancela el proceso de crear receta y el Caso de Uso queda sin efecto.}\\ 
\multicolumn{2}{Z}{\hspace{0.5cm}\textbf{15.b)}  El Usuario desea añadir más alimentos para la preparación de la receta, el Caso de Uso continúa en 8.}\\ 

\hline
\multicolumn{2}{c}{\textbf{Excepciones}}\\ \midrule

 \multicolumn{2}{Z}{\hspace{0.5cm}\textbf{4.a)} El sistema comprueba que el formato de los datos introducidos es incorrecto, el sistema informa al Usuario, el Caso de Uso continúa en el paso 3. }\\

\multicolumn{2}{Z}{\hspace{0.5cm}\textbf{7.a)} El sistema comprueba que el formato de los datos introducidos es incorrecto, el sistema informa al Usuario, el Caso de Uso continúa en el paso 6. }\\ 

\multicolumn{2}{Z}{\hspace{0.5cm}\textbf{13.a)} El sistema comprueba que el formato de los datos introducidos es incorrecto, el sistema informa al Usuario, el Caso de Uso continúa en el paso 12. }\\  

\hline
\bottomrule
\end{tabularx}
\caption{Descripción del Caso de Uso \textit{Crear receta}.} \label{tab:caso_uso_crear_receta}
\end{table}




%CU-7 Buscar alimento 
\begin{table}[H]
\begin{tabularx}{1\linewidth}{P}
\toprule
\textbf{Caso de uso:} &  CU-7 Buscar alimento\\ \midrule
\textbf{Descripción:} & Un Usuario pretende encontrar un determinado alimento de la aplicación buscando por su nombre.\\ \hline
\textbf{Actores:} & Usuario. \\ \hline
\textbf{Pre-condiciones:} &  El Usuario está identificado en el sistema.\\ \hline
\textbf{Post-condiciones:} & El Usuario obtiene una lista de alimentos con nombre similar al realizado en la búsqueda. \\ \midrule
\multicolumn{2}{c}{\textbf{Secuencia normal}}\\ \midrule
\multicolumn{2}{Z}{\hspace{0.5cm}\textbf{1.} El Usuario inicia el proceso de búsqueda de alimentos. }\\ 
\multicolumn{2}{Z}{\hspace{.5cm}\textbf{2.} El sistema solicita al Usuario que introduzca el nombre del alimento a buscar.  }\\ 
\multicolumn{2}{Z}{\hspace{0.5cm}\textbf{3.} El Usuario introduce el nombre del alimento a buscar.  }\\
\multicolumn{2}{Z}{\hspace{0.5cm}\textbf{4.} El sistema valida el nombre introducido por el Usuario. }\\ 
\multicolumn{2}{Z}{\hspace{0.5cm}\textbf{5.} El sistema busca los alimentos en el sistema con el nombre introducido. }\\ 
\multicolumn{2}{Z}{\hspace{0.5cm}\textbf{6.} El sistema muestra los resultados de la búsqueda al Usuario. }\\ 
\multicolumn{2}{Z}{\hspace{0.5cm}\textbf{7.} El Caso de Uso finaliza.}\\ 
 \hline
\multicolumn{2}{c}{\textbf{Secuencias alternativas}}\\ \midrule
\multicolumn{2}{Z}{\hspace{0.5cm}\textbf{3.a)} El Usuario cancela el proceso de buscar alimento y el Caso de Uso queda sin efecto.}\\ 

\hline

\multicolumn{2}{c}{\textbf{Excepciones}}\\ \midrule

\multicolumn{2}{Z}{\hspace{0.5cm}\textbf{4.a)}  El sistema comprueba que el formato del dato introducido no es correcto, el sistema informa al Usuario, el Caso de Uso continúa en el paso 3.}\\ 

\multicolumn{2}{Z}{\hspace{0.5cm}\textbf{5.a)}  El sistema no encuentra ningún resultado que satisfaga la búsqueda, el sistema informa al Usuario, el Caso de Uso finaliza.}\\ 

\hline
\bottomrule
\end{tabularx}
\caption{Descripción del Caso de Uso \textit{Buscar alimento}.} \label{tab:caso_uso_buscar_alimento}
\end{table}



%CU-8 Buscar receta por titulo
\begin{table}[H]
\begin{tabularx}{1\linewidth}{P}
\toprule
\textbf{Caso de uso:} &  CU-8 Buscar receta\\ \midrule
\textbf{Descripción:} & Un Usuario pretende encontrar una determinada receta de la aplicación buscando por su título de entre todas las recetas registradas en la aplicación. \\ \hline
\textbf{Actores:} & Usuario. \\ \hline
\textbf{Pre-condiciones:} &  El Usuario está identificado en el sistema.\\ \hline
\textbf{Post-condiciones:} & El Usuario obtiene una lista de recetas con título similar al realizado en la búsqueda. \\ \midrule
\multicolumn{2}{c}{\textbf{Secuencia normal}}\\ \midrule
\multicolumn{2}{Z}{\hspace{0.5cm}\textbf{1.} El Usuario inicia el proceso de búsqueda de recetas. }\\ 
\multicolumn{2}{Z}{\hspace{.5cm}\textbf{2.} El sistema solicita al Usuario que introduzca el título de la receta a buscar.  }\\ 
\multicolumn{2}{Z}{\hspace{0.5cm}\textbf{3.} El Usuario introduce el título de la receta a buscar.  }\\
\multicolumn{2}{Z}{\hspace{0.5cm}\textbf{4.} El sistema valida el título introducido por el Usuario. }\\ 
\multicolumn{2}{Z}{\hspace{0.5cm}\textbf{5.} El sistema busca las recetas en el sistema con el título introducido.}\\ 
\multicolumn{2}{Z}{\hspace{0.5cm}\textbf{6.} El sistema muestra los resultados de la búsqueda al Usuario. }\\ 
\multicolumn{2}{Z}{\hspace{0.5cm}\textbf{7.} El Caso de Uso finaliza.}\\ 
 \midrule

\multicolumn{2}{c}{\textbf{Secuencias alternativas}}\\ \midrule
\multicolumn{2}{Z}{\hspace{0.5cm}\textbf{3.a)} El Usuario cancela el proceso de búsqueda de recetas y el Caso de Uso queda sin efecto. }\\ \hline


\multicolumn{2}{c}{\textbf{Excepciones}}\\ \midrule

\multicolumn{2}{Z}{\hspace{0.5cm}\textbf{4.a)}  El sistema comprueba que el formato del dato introducido no es correcto, el sistema informa al Usuario, el Caso de Uso continúa en el paso 3.}\\ 

\multicolumn{2}{Z}{\hspace{0.5cm}\textbf{5.a)} El sistema no encuentra ningún resultado que satisfaga la búsqueda, el sistema informa al Usuario, el Caso de Uso finaliza.  }\\ 

\hline
\bottomrule
\end{tabularx}
\caption{Descripción del Caso de Uso \textit{Buscar receta}.} \label{tab:caso_uso_buscar_receta}
\end{table}



%CU-9 Crear publicacion
\begin{table}[H]
\begin{tabularx}{1\linewidth}{P}
\toprule
\textbf{Caso de uso:} &  CU-9 Crear publicación\\ \midrule
\textbf{Descripción:} & El Usuario desea guardar una publicación en la aplicación para que así todos los usuarios la puedan ver.\\ \hline
\textbf{Actores:} & Usuario. \\ \hline
\textbf{Pre-condiciones:} & El Usuario se ha identificado en el sistema.\\ \hline
\textbf{Post-condiciones:} & La publicación se ha guardado en el sistema y es visible por todos los usuarios de la aplicación. \\ \midrule
\multicolumn{2}{c}{\textbf{Secuencia normal}}\\ \midrule
\multicolumn{2}{Z}{\hspace{0.5cm}\textbf{1.} El Usuario selecciona crear una publicación. }\\ 
\multicolumn{2}{Z}{\hspace{.5cm}\textbf{2.} El sistema solicita título, descripción y foto. }\\ 
\multicolumn{2}{Z}{\hspace{0.5cm}\textbf{3.} El Usuario introduce los datos solicitados.  }\\
\multicolumn{2}{Z}{\hspace{0.5cm}\textbf{4.} El sistema valida el formato de los datos introducidos. }\\ 
\multicolumn{2}{Z}{\hspace{0.5cm}\textbf{5.} El sistema solicita si la publicación se va a enlazar con una receta o un alimento. }\\ 
\multicolumn{2}{Z}{\hspace{0.5cm}\textbf{6.} El Usuario introduce si la publicación se va a enlazar con una receta o un alimento.}\\ 
\multicolumn{2}{Z}{\hspace{0.5cm}\textbf{7.} Si el Usuario selecciona receta.}\\
\multicolumn{2}{Z}{\hspace{0.5cm}\textbf{  } \textbf{ Punto de extensión: CU-8 Buscar receta.}}\\


\multicolumn{2}{Z}{\hspace{0.5cm}\textbf{8.} El Usuario selecciona una de las recetas encontradas en la búsqueda. }\\ 
\multicolumn{2}{Z}{\hspace{0.5cm}\textbf{9.} El sistema solicita confirmación para crear la publicación. }\\ 
\multicolumn{2}{Z}{\hspace{0.5cm}\textbf{10.} El Usuario confirma la creación de la publicación. }\\
\multicolumn{2}{Z}{\hspace{0.5cm}\textbf{11.} El sistema guarda la publicación. }\\
\multicolumn{2}{Z}{\hspace{0.5cm}\textbf{12.} El Caso de Uso finaliza.}\\
\midrule

\multicolumn{2}{c}{\textbf{Secuencias alternativas}}\\ \midrule
\multicolumn{2}{Z}{\hspace{0.5cm}\textbf{3.a,6.a,8.a,10.a)} El Usuario cancela el proceso de crear publicación y el Caso de Uso queda sin efecto. }\\
\multicolumn{2}{Z}{\hspace{0.5cm}\textbf{7.a)} Si el Usuario selecciona alimento. \textbf{Punto de extensión: CU-7 Buscar alimento.} El Usuario selecciona uno de los alimentos encontrados en la búsqueda, el Caso de Uso continúa en 9. }\\
\hline


\multicolumn{2}{c}{\textbf{Excepciones}}\\ \midrule

\multicolumn{2}{Z}{\hspace{0.5cm}\textbf{4.a)}  El sistema comprueba que el formato de los datos introducidos es incorrecto, el sistema informa al Usuario, el Caso de Uso continúa en el paso 3.}\\ 

\hline
\bottomrule
\end{tabularx}
\caption{Descripción del Caso de Uso \textit{Crear publicación}.} \label{tab:caso_uso_crear_publicacion}
\end{table}



%CU-10 Buscar Usuario por username
\begin{table}[H]
\begin{tabularx}{1\linewidth}{P}
\toprule
\textbf{Caso de uso:} &  CU-10 Buscar Usuario\\ \midrule
\textbf{Descripción:} & El Usuario pretende encontrar un determinado Usuario de la aplicación buscando por su \textit{username}.\\ \hline
\textbf{Actores:} &  Usuario\\ \hline
\textbf{Pre-condiciones:} &  El Usuario está identificado en el sistema.\\ \hline
\textbf{Post-condiciones:} & El Usuario obtiene una lista de usuarios con \textit{username} similar al realizado en la búsqueda. \\ \midrule
\multicolumn{2}{c}{\textbf{Secuencia normal}}\\ \midrule
\multicolumn{2}{Z}{\hspace{0.5cm}\textbf{1.} El Usuario inicia el proceso de búsqueda de usuarios.}\\ 
\multicolumn{2}{Z}{\hspace{.5cm}\textbf{2.} El sistema solicita al Usuario que introduzca el \textit{username} del Usuario a buscar. }\\ 
\multicolumn{2}{Z}{\hspace{0.5cm}\textbf{3.} El Usuario introduce el \textit{username} del Usuario a buscar. }\\
\multicolumn{2}{Z}{\hspace{0.5cm}\textbf{4.} El sistema valida el \textit{username} introducido por el Usuario.}\\ 
\multicolumn{2}{Z}{\hspace{0.5cm}\textbf{5.} El sistema busca los usuarios en el sistema con el \textit{username} introducido.}\\ 
\multicolumn{2}{Z}{\hspace{0.5cm}\textbf{6.} El sistema muestra los resultados de la búsqueda al Usuario.}\\ 
\multicolumn{2}{Z}{\hspace{0.5cm}\textbf{7.} El Caso de Uso finaliza.}\\ 
 \midrule

\multicolumn{2}{c}{\textbf{Secuencias alternativas}}\\ \midrule
\multicolumn{2}{Z}{\hspace{0.5cm}\textbf{3.a)} El Usuario cancela el proceso de búsqueda de usuarios y el Caso de Uso queda sin efecto.}\\ \hline


\multicolumn{2}{c}{\textbf{Excepciones}}\\ \midrule

\multicolumn{2}{Z}{\hspace{0.5cm}\textbf{4.a)}  El sistema comprueba que el formato del dato introducido no es correcto, el sistema informa al Usuario, el Caso de Uso continúa en el paso 3.}\\ 

\multicolumn{2}{Z}{\hspace{0.5cm}\textbf{5.a)} El sistema no encuentra ningún resultado que satisfaga la búsqueda, el sistema informa al Usuario, el Caso de Uso finaliza. }\\ 

\hline
\bottomrule
\end{tabularx}
\caption{Descripción del Caso de Uso \textit{Buscar usuario}.} \label{tab:caso_uso_buscar_usuario}
\end{table}


%CU-11 Añadir receta a favoritas
\begin{table}[H]
\begin{tabularx}{1\linewidth}{P}
\toprule
\textbf{Caso de uso:} &  CU-11 Añadir receta a favoritas\\ \midrule
\textbf{Descripción:} & El Usuario desea guardar una receta como favorita para así poder acceder a ella más rápidamente.\\ \hline
\textbf{Actores:} &  Usuario.\\ \hline
\textbf{Pre-condiciones:} & El Usuario está identificado en el sistema.\\ \hline
\textbf{Post-condiciones:} &  La lista de recetas favoritas del Usuario incluye la nueva receta favorita.\\ \midrule
\multicolumn{2}{c}{\textbf{Secuencia normal}}\\ \midrule
\multicolumn{2}{Z}{\hspace{0.5cm}\textbf{1.} El Usuario selecciona guardar la receta como favorita.}\\ 
\multicolumn{2}{Z}{\hspace{.5cm}\textbf{2.}  El sistema guarda la receta en la lista de recetas favoritas del Usuario.}\\ 
\multicolumn{2}{Z}{\hspace{0.5cm}\textbf{3.} El sistema informa al Usuario. }\\
\multicolumn{2}{Z}{\hspace{0.5cm}\textbf{4.} El Caso de Uso finaliza.}\\ 


\hline
\bottomrule
\end{tabularx}
\caption{Descripción del Caso de Uso \textit{Añadir receta a favoritas}.} \label{tab:caso_uso_añadir_favorita}
\end{table}


%CU-12 Consultar recetas favoritas
\begin{table}[H]
\begin{tabularx}{1\linewidth}{P}
\toprule
\textbf{Caso de uso:} &  CU-12 Consultar recetas favoritas\\ \midrule
\textbf{Descripción:} & El Usuario desea ver su lista de recetas favoritas.\\ \hline
\textbf{Actores:} & Usuario. \\ \hline
\textbf{Pre-condiciones:} & El Usuario está identificado en el sistema.\\ \hline
\textbf{Post-condiciones:} &  El Usuario podrá ver la lista de recetas que ha marcado como favoritas.\\ \midrule
\multicolumn{2}{c}{\textbf{Secuencia normal}}\\ \midrule
\multicolumn{2}{Z}{\hspace{0.5cm}\textbf{1.} El Usuario selecciona ver sus recetas favoritas.}\\ 
\multicolumn{2}{Z}{\hspace{.5cm}\textbf{2.}  El sistema busca sus recetas favoritas.}\\ 
\multicolumn{2}{Z}{\hspace{0.5cm}\textbf{3.} El sistema muestra sus recetas favoritas. Si el Usuario selecciona eliminar una receta de favoritas. }\\
\multicolumn{2}{Z}{\hspace{0.5cm}\textbf{  }  \textbf{  Punto de extensión: CU-13 Borrar receta de favoritas.}}\\ 
\multicolumn{2}{Z}{\hspace{0.5cm}\textbf{4.} El Caso de Uso finaliza.}\\ 
 


\hline

\multicolumn{2}{c}{\textbf{Excepciones}}\\ \midrule
\multicolumn{2}{Z}{\hspace{0.5cm}\textbf{2.a)} El sistema detecta que el Usuario no tiene ninguna receta favorita, el sistema informa al Usuario, el Caso de Uso finaliza.}\\ 


\hline
\bottomrule
\end{tabularx}
\caption{Descripción del Caso de Uso \textit{Consultar recetas favoritas}.} \label{tab:caso_uso_consultar_favoritas}
\end{table}


%CU-13 Borrar receta de favoritas
\begin{table}[H]
\begin{tabularx}{1\linewidth}{P}
\toprule
\textbf{Caso de uso:} &  CU-13 Borrar receta de favoritas\\ \midrule
\textbf{Descripción:} & El Usuario decide que una de las recetas que ha marcado como favoritas ya no la quiere guardar en su lista de favoritas.\\ \hline
\textbf{Actores:} & Usuario \\ \hline
\textbf{Pre-condiciones:} &  El Usuario está identificado en el sistema y se ha ejecutado el Caso de Uso  CU-12 Consultar recetas favoritas.\\ \hline
\textbf{Post-condiciones:} &  La lista de recetas favoritas del Usuario no contará con la receta eliminada de favoritas.\\ \midrule
\multicolumn{2}{c}{\textbf{Secuencia normal}}\\ \midrule
\multicolumn{2}{Z}{\hspace{0.5cm}\textbf{1.} El Usuario selecciona la receta a eliminar de sus recetas favoritas.}\\ 
\multicolumn{2}{Z}{\hspace{.5cm}\textbf{2.} El sistema elimina la receta de su lista de recetas favoritas. }\\ 
\multicolumn{2}{Z}{\hspace{0.5cm}\textbf{3.} El sistema informa al Usuario. }\\
\multicolumn{2}{Z}{\hspace{0.5cm}\textbf{4.} El Caso de Uso finaliza.}\\ 
 


\hline
\bottomrule
\end{tabularx}
\caption{Descripción del Caso de Uso \textit{Borrar receta de favoritas}.} \label{tab:caso_uso_borrar_favorita}
\end{table}










%CU-14 Consultar usuarios seguidos
\begin{table}[H]
\begin{tabularx}{1\linewidth}{P}
\toprule
\textbf{Caso de uso:} &  CU-14 Consultar usuarios seguidos\\ \midrule
\textbf{Descripción:} & El Usuario desea ver los usuarios que sigue.\\ \hline
\textbf{Actores:} &  Usuario\\ \hline
\textbf{Pre-condiciones:} & El Usuario está identificado en el sistema.\\ \hline
\textbf{Post-condiciones:} & El Usuario podrá ver la lista de usuarios que sigue. \\ \midrule
\multicolumn{2}{c}{\textbf{Secuencia normal}}\\ \midrule
\multicolumn{2}{Z}{\hspace{0.5cm}\textbf{1.} El Usuario selecciona ver los usuarios que sigue. }\\ 
\multicolumn{2}{Z}{\hspace{.5cm}\textbf{2.} El sistema busca los usuarios que sigue. }\\ 
\multicolumn{2}{Z}{\hspace{0.5cm}\textbf{3.} El sistema muestra los usuarios que sigue. Si el Usuario selecciona dejar de seguir a un Usuario. }\\
\multicolumn{2}{Z}{\hspace{0.5cm}\textbf{  } \textbf{  Punto de extensión: CU-15 Dejar de seguir Usuario.}}\\ 
\multicolumn{2}{Z}{\hspace{0.5cm}\textbf{4.} El Caso de Uso finaliza. }\\ 
 

 \hline


\multicolumn{2}{c}{\textbf{Excepciones}}\\ \midrule

\multicolumn{2}{Z}{\hspace{0.5cm}\textbf{2.a)} El sistema detecta que el Usuario no tiene ningún Usuario que  sigue, el sistema informa al Usuario, el Caso de Uso finaliza. }\\ 

\hline
\bottomrule
\end{tabularx}
\caption{Descripción del Caso de Uso \textit{Consultar usuarios seguidos}.} \label{tab:caso_uso_consultar_seguidos}
\end{table}


%CU-15 Dejar de seguir usuario
\begin{table}[H]
\begin{tabularx}{1\linewidth}{P}
\toprule
\textbf{Caso de uso:} &  CU-15 Dejar de seguir Usuario\\ \midrule
\textbf{Descripción:} & El Usuario decide que ya no quiere continuar siguiendo uno de los usuarios a los que sigue.\\ \hline
\textbf{Actores:} &  Usuario\\ \hline
\textbf{Pre-condiciones:} & El Usuario está identificado en el sistema y se ha ejecutado el Caso de Uso CU-14 Consultar usuarios seguidos.\\ \hline
\textbf{Post-condiciones:} & La lista de usuarios seguidos por el Usuario no incluye el Usuario que se ha dejado de seguir. \\ \midrule
\multicolumn{2}{c}{\textbf{Secuencia normal}}\\ \midrule
\multicolumn{2}{Z}{\hspace{0.5cm}\textbf{1.} El Usuario selecciona el Usuario que quiere dejar de seguir.}\\ 
\multicolumn{2}{Z}{\hspace{.5cm}\textbf{2.}  El sistema elimina el Usuario de los usuarios a los que sigue.}\\ 
\multicolumn{2}{Z}{\hspace{0.5cm}\textbf{3.}  El sistema informa al Usuario.}\\
\multicolumn{2}{Z}{\hspace{0.5cm}\textbf{4.} El Caso de Uso finaliza.}\\ 
 



\hline
\bottomrule
\end{tabularx}
\caption{Descripción del Caso de Uso \textit{Dejar de seguir Usuario}.} \label{tab:caso_uso_dejar_seguir}
\end{table}



%CU-16 Seguir usuario
\begin{table}[H]
\begin{tabularx}{1\linewidth}{P}
\toprule
\textbf{Caso de uso:} &  CU-16 Seguir Usuario\\ \midrule
\textbf{Descripción:} & El Usuario desea seguir a un Usuario para conocer sus publicaciones.\\ \hline
\textbf{Actores:} &  Usuario.\\ \hline
\textbf{Pre-condiciones:} & El Usuario está identificado en el sistema.\\ \hline
\textbf{Post-condiciones:} & La lista de usuarios seguidos del Usuario incluye el nuevo Usuario. \\ \midrule
\multicolumn{2}{c}{\textbf{Secuencia normal}}\\ \midrule
\multicolumn{2}{Z}{\hspace{0.5cm}\textbf{1.} El Usuario selecciona seguir al Usuario. }\\ 
\multicolumn{2}{Z}{\hspace{.5cm}\textbf{2.} El sistema guarda el Usuario en la lista de usuarios que sigue el Usuario. }\\ 
\multicolumn{2}{Z}{\hspace{0.5cm}\textbf{3.} El sistema informa al Usuario. }\\
\multicolumn{2}{Z}{\hspace{0.5cm}\textbf{4.} El Caso de Uso finaliza. }\\ 


 

\hline
\bottomrule
\end{tabularx}
\caption{Descripción del Caso de Uso \textit{Seguir Usuario}.} \label{tab:caso_uso_seguir_usuario}
\end{table}











%CU-17 Añadir comentario a publicación
\begin{table}[H]
\begin{tabularx}{1\linewidth}{P}
\toprule
\textbf{Caso de uso:} &  CU-17 Añadir comentario a publicación\\ \midrule
\textbf{Descripción:} & El Usuario desea guardar un comentario en una publicación.\\ \hline
\textbf{Actores:} & Usuario  \\ \hline
\textbf{Pre-condiciones:} & El Usuario está identificado en el sistema.\\ \hline
\textbf{Post-condiciones:} &  La publicación tiene el nuevo comentario guardado.\\ \midrule
\multicolumn{2}{c}{\textbf{Secuencia normal}}\\ \midrule
\multicolumn{2}{Z}{\hspace{0.5cm}\textbf{1.} El Usuario selecciona crear un comentario.}\\ 
\multicolumn{2}{Z}{\hspace{.5cm}\textbf{2.}  El sistema solicita el comentario a guardar.}\\ 
\multicolumn{2}{Z}{\hspace{0.5cm}\textbf{3.} El Usuario introduce el comentario. }\\
\multicolumn{2}{Z}{\hspace{0.5cm}\textbf{4.} El sistema valida el contenido del comentario introducido. }\\ 
\multicolumn{2}{Z}{\hspace{0.5cm}\textbf{5.} El sistema solicita confirmación.}\\ 
\multicolumn{2}{Z}{\hspace{0.5cm}\textbf{6.} El Usuario confirma la acción para que se guarde el comentario. }\\ 
\multicolumn{2}{Z}{\hspace{0.5cm}\textbf{7.} El sistema guarda el comentario de la publicación.}\\ 
\multicolumn{2}{Z}{\hspace{0.5cm}\textbf{8.} El Caso de Uso finaliza. }\\ 
 \midrule

\multicolumn{2}{c}{\textbf{Secuencias alternativas}}\\ \midrule
\multicolumn{2}{Z}{\hspace{0.5cm}\textbf{3.a,6.a)} El Usuario cancela el proceso de crear comentario y el Caso de Uso queda sin efecto. }\\ \hline


\multicolumn{2}{c}{\textbf{Excepciones}}\\ \midrule

\multicolumn{2}{Z}{\hspace{0.5cm}\textbf{4.a)} El sistema comprueba que el contenido del comentario es incorrecto, el sistema informa al Usuario, el Caso de Uso continúa en el paso 3. }\\ 

\hline
\bottomrule
\end{tabularx}
\caption{Descripción del Caso de Uso \textit{Añadir comentario a publicación}.} \label{tab:caso_uso_añadir_comentario}
\end{table}







%CU-24 Eliminar usuario
\begin{table}[H]
\begin{tabularx}{1\linewidth}{P}
\toprule
\textbf{Caso de uso:} &  CU-18 Eliminar Usuario\\ \midrule
\textbf{Descripción:} & El Administrador decide que un Usuario registrado en la aplicación debe ser eliminado. \\ \hline
\textbf{Actores:} & Administrador. \\ \hline
\textbf{Pre-condiciones:} & El Administrador está identificado en el sistema.\\ \hline
\textbf{Post-condiciones:} & El Administrador ha eliminado el Usuario y ya no aparecerá en la aplicación.  \\ \midrule
\multicolumn{2}{c}{\textbf{Secuencia normal}}\\ \midrule
\multicolumn{2}{Z}{\hspace{0.5cm}\textbf{1.} El Administrador selecciona el Usuario a eliminar. }\\ 
\multicolumn{2}{Z}{\hspace{.5cm}\textbf{2.}  El sistema solicita confirmación de la acción.}\\ 
\multicolumn{2}{Z}{\hspace{0.5cm}\textbf{3.}  El Administrador confirma que quiere eliminar ese usuario.}\\
\multicolumn{2}{Z}{\hspace{0.5cm}\textbf{4.} El sistema elimina el usuario.}\\ 
\multicolumn{2}{Z}{\hspace{0.5cm}\textbf{5.} El sistema elimina la lista de seguidores que le siguen, la lista de seguidos que sigue, sus recetas favoritas, sus comentarios realizados, sus recetas creadas y sus publicaciones creadas.}\\ 
\multicolumn{2}{Z}{\hspace{0.5cm}\textbf{6.} El sistema informa al Administrador.}\\ 
\multicolumn{2}{Z}{\hspace{0.5cm}\textbf{7.} El Caso de Uso finaliza.}\\ 
 \midrule

\multicolumn{2}{c}{\textbf{Secuencias alternativas}}\\ \midrule
\multicolumn{2}{Z}{\hspace{0.5cm}\textbf{3.a)} El Administrador cancela el proceso de eliminar Usuario y el Caso de Uso queda sin efecto. }\\ \hline




\hline
\bottomrule
\end{tabularx}
\caption{Descripción del Caso de Uso \textit{Eliminar Usuario}.} \label{tab:caso_uso_eliminar_usuario}
\end{table}



%CU-19 Eliminar publicación
\begin{table}[H]
\begin{tabularx}{1\linewidth}{P}
\toprule
\textbf{Caso de uso:} &  CU-19 Eliminar publicación\\ \midrule
\textbf{Descripción:} & El Administrador decide que una publicación mostrada en la aplicación y creada por un Usuario debe ser eliminada.\\ \hline
\textbf{Actores:} &  Administrador.\\ \hline
\textbf{Pre-condiciones:} & El Administrador está identificado en el sistema.\\ \hline
\textbf{Post-condiciones:} &  El Administrador ha eliminado la publicación y ya no aparecerá en la aplicación. \\ \midrule
\multicolumn{2}{c}{\textbf{Secuencia normal}}\\ \midrule
\multicolumn{2}{Z}{\hspace{0.5cm}\textbf{1.} El Administrador selecciona la publicación a eliminar.  }\\ 
\multicolumn{2}{Z}{\hspace{.5cm}\textbf{2.} El sistema solicita confirmación de la acción.}\\ 
\multicolumn{2}{Z}{\hspace{0.5cm}\textbf{3.} El Administrador confirma que quiere eliminar esa publicación. }\\
\multicolumn{2}{Z}{\hspace{0.5cm}\textbf{4.} El sistema elimina la publicación. }\\ 
\multicolumn{2}{Z}{\hspace{0.5cm}\textbf{5.} El sistema elimina los comentarios realizados en esa publicación.}\\ 
\multicolumn{2}{Z}{\hspace{0.5cm}\textbf{6.}El sistema informa al Administrador. }\\ 
\multicolumn{2}{Z}{\hspace{0.5cm}\textbf{7.} El Caso de Uso finaliza.}\\ 
 \midrule

\multicolumn{2}{c}{\textbf{Secuencias alternativas}}\\ \midrule
\multicolumn{2}{Z}{\hspace{0.5cm}\textbf{3.a)} El Administrador cancela el proceso de eliminar publicación y el Caso de Uso queda sin efecto. }\\ \hline




\hline
\bottomrule
\end{tabularx}
\caption{Descripción del Caso de Uso \textit{Eliminar publicación}.} \label{tab:caso_uso_eliminar_publicacion}
\end{table}



%CU-20 Eliminar alimento
\begin{table}[H]
\begin{tabularx}{1\linewidth}{P}
\toprule
\textbf{Caso de uso:} &  CU-20 Eliminar alimento\\ \midrule
\textbf{Descripción:} & El Administrador decide que un alimento mostrado en la aplicación debe ser eliminado.\\ \hline
\textbf{Actores:} & Administrador. \\ \hline
\textbf{Pre-condiciones:} & El Administrador está identificado en el sistema.\\ \hline
\textbf{Post-condiciones:} & El Administrador ha eliminado el alimento y ya no aparecerá en la aplicación.  \\ \midrule
\multicolumn{2}{c}{\textbf{Secuencia normal}}\\ \midrule
\multicolumn{2}{Z}{\hspace{0.5cm}\textbf{1.} El Administrador selecciona el alimento a eliminar.  }\\ 
\multicolumn{2}{Z}{\hspace{.5cm}\textbf{2.} El sistema solicita confirmación de la acción. }\\ 
\multicolumn{2}{Z}{\hspace{0.5cm}\textbf{3.}  El Administrador confirma que quiere eliminar ese alimento.}\\
\multicolumn{2}{Z}{\hspace{0.5cm}\textbf{4.} El sistema elimina el alimento.}\\ 
\multicolumn{2}{Z}{\hspace{0.5cm}\textbf{5.} El sistema elimina las publicaciones enlazadas con este alimento y de las listas de alimentos utilizados en cada receta los lugares donde se utilice este alimento.}\\ 
\multicolumn{2}{Z}{\hspace{0.5cm}\textbf{6.} El sistema informa al Administrador.}\\ 
\multicolumn{2}{Z}{\hspace{0.5cm}\textbf{7.} El Caso de Uso finaliza.}\\ 
 \midrule

\multicolumn{2}{c}{\textbf{Secuencias alternativas}}\\ \midrule
\multicolumn{2}{Z}{\hspace{0.5cm}\textbf{3.a)} El Administrador cancela el proceso de eliminar alimento y el Caso de Uso queda sin efecto. }\\ \hline




\hline
\bottomrule
\end{tabularx}
\caption{Descripción del Caso de Uso \textit{Eliminar alimento}.} \label{tab:caso_uso_eliminar_alimento}
\end{table}



%CU-21 Editar información de alimento
\begin{table}[H]
\begin{tabularx}{1\linewidth}{P}
\toprule
\textbf{Caso de uso:} &  CU-21 Editar información de alimento\\ \midrule
\textbf{Descripción:} & El Administrador desea cambiar la información de un alimento.\\ \hline
\textbf{Actores:} &  Administrador\\ \hline
\textbf{Pre-condiciones:} & El Administrador se ha identificado en el sistema.\\ \hline
\textbf{Post-condiciones:} & La información del alimento se habrá actualizado con los nuevos campos introducidos. \\ \midrule
\multicolumn{2}{c}{\textbf{Secuencia normal}}\\ \midrule
\multicolumn{2}{Z}{\hspace{0.5cm}\textbf{1.} El Administrador selecciona editar la información de un alimento.}\\ 
\multicolumn{2}{Z}{\hspace{.5cm}\textbf{2.} El sistema muestra la información del alimento. }\\ 
\multicolumn{2}{Z}{\hspace{0.5cm}\textbf{3.} El sistema solicita los nuevos valores para estos campos. }\\
\multicolumn{2}{Z}{\hspace{0.5cm}\textbf{4.} El Administrador introduce los nuevos valores.}\\ 
\multicolumn{2}{Z}{\hspace{0.5cm}\textbf{5.} El sistema valida el formato de los datos de los nuevos valores.}\\ 
\multicolumn{2}{Z}{\hspace{0.5cm}\textbf{6.} El sistema solicita confirmación.}\\ 
\multicolumn{2}{Z}{\hspace{0.5cm}\textbf{7.} El Administrador confirma el cambio.}\\ 
\multicolumn{2}{Z}{\hspace{0.5cm}\textbf{8.} El sistema guarda la nueva información del alimento. }\\ 
\multicolumn{2}{Z}{\hspace{0.5cm}\textbf{9.} El sistema informa al Administrador de que la operación se ha realizado con éxito.}\\ 
\multicolumn{2}{Z}{\hspace{0.5cm}\textbf{10.} El Caso de Uso finaliza.}\\ \midrule

\multicolumn{2}{c}{\textbf{Secuencias alternativas}}\\ \midrule
\multicolumn{2}{Z}{\hspace{0.5cm}\textbf{4.a,7.a)} El Administrador cancela el proceso de editar información del alimento y el Caso de Uso queda sin efecto. }\\ \hline


\multicolumn{2}{c}{\textbf{Excepciones}}\\ \midrule

\multicolumn{2}{Z}{\hspace{0.5cm}\textbf{5.a)} El sistema comprueba que el formato de los datos introducidos es incorrecto, el sistema informa al Administrador, el Caso de Uso continúa en el paso 4. }\\ 

\hline
\bottomrule
\end{tabularx}
\caption{Descripción del Caso de Uso \textit{Editar información de alimento}.} \label{tab:caso_uso_editar_alimento}
\end{table}



%CU-22 Eliminar receta
\begin{table}[H]
\begin{tabularx}{1\linewidth}{P}
\toprule
\textbf{Caso de uso:} &  CU-22 Eliminar receta\\ \midrule
\textbf{Descripción:} & El Administrador decide que una receta mostrada en la aplicación y creada por un Usuario debe ser eliminada.\\ \hline
\textbf{Actores:} & Administrador. \\ \hline
\textbf{Pre-condiciones:} & El Administrador está identificado en el sistema.\\ \hline
\textbf{Post-condiciones:} & El Administrador ha eliminado la receta y ya no aparecerá en la aplicación.  \\ \midrule
\multicolumn{2}{c}{\textbf{Secuencia normal}}\\ \midrule
\multicolumn{2}{Z}{\hspace{0.5cm}\textbf{1.} El Administrador selecciona la receta a eliminar. }\\ 
\multicolumn{2}{Z}{\hspace{.5cm}\textbf{2.} El sistema solicita confirmación de la acción. }\\ 
\multicolumn{2}{Z}{\hspace{0.5cm}\textbf{3.} El Administrador confirma que quiere eliminar esa receta. }\\
\multicolumn{2}{Z}{\hspace{0.5cm}\textbf{4.} El sistema elimina la receta.}\\ 
\multicolumn{2}{Z}{\hspace{0.5cm}\textbf{5.} El sistema elimina las publicaciones enlazadas con esa receta, elimina de la lista de recetas favoritas de cada Usuario esta receta y elimina los alimentos que se utilizan como ingredientes para la receta.}\\ 
\multicolumn{2}{Z}{\hspace{0.5cm}\textbf{6.} El sistema informa al Administrador.}\\ 
\multicolumn{2}{Z}{\hspace{0.5cm}\textbf{7.} El Caso de Uso finaliza.}\\ 
 \midrule

\multicolumn{2}{c}{\textbf{Secuencias alternativas}}\\ \midrule
\multicolumn{2}{Z}{\hspace{0.5cm}\textbf{3.a)} El Administrador cancela el proceso de eliminar receta y el Caso de Uso queda sin efecto.}\\ \hline




\hline
\bottomrule
\end{tabularx}
\caption{Descripción del Caso de Uso \textit{Eliminar receta}.} \label{tab:caso_uso_eliminar_receta}
\end{table}



%CU-23 Crear alimento
\begin{table}[H]
\begin{tabularx}{1\linewidth}{P}
\toprule
\textbf{Caso de uso:} &  CU-23 Crear alimento\\ \midrule
\textbf{Descripción:} & El Administrador desea guardar un alimento en la aplicación para que así todos los usuarios lo puedan ver.\\ \hline
\textbf{Actores:} &  Administrador.\\ \hline
\textbf{Pre-condiciones:} & El Administrador se ha identificado en el sistema.\\ \hline
\textbf{Post-condiciones:} & El alimento se ha guardado en el sistema y es visible por todos los usuarios de la aplicación. \\ \midrule
\multicolumn{2}{c}{\textbf{Secuencia normal}}\\ \midrule
\multicolumn{2}{Z}{\hspace{0.5cm}\textbf{1.} El Administrador selecciona crear un alimento. }\\ 
\multicolumn{2}{Z}{\hspace{.5cm}\textbf{2.} El sistema solicita nombre, descripción, calorías, foto, enlace, grasas, carbohidratos, proteínas, cantidad y medida. }\\ 
\multicolumn{2}{Z}{\hspace{0.5cm}\textbf{3.} El Administrador introduce los datos solicitados. }\\
\multicolumn{2}{Z}{\hspace{0.5cm}\textbf{4.} El sistema valida que el formato de los datos introducidos sea correcto.}\\ 
\multicolumn{2}{Z}{\hspace{0.5cm}\textbf{5.} El sistema solicita confirmación para crear el alimento.}\\ 
\multicolumn{2}{Z}{\hspace{0.5cm}\textbf{6.} El Administrador confirma la creación del alimento.}\\ 
\multicolumn{2}{Z}{\hspace{0.5cm}\textbf{7.} El sistema guarda el alimento.}\\ 
\multicolumn{2}{Z}{\hspace{0.5cm}\textbf{8.} El Caso de Uso finaliza.}\\ 
 \midrule

\multicolumn{2}{c}{\textbf{Secuencias alternativas}}\\ \midrule
\multicolumn{2}{Z}{\hspace{0.5cm}\textbf{3.a,6.a)} El Administrador cancela el proceso de crear alimento y el Caso de Uso queda sin efecto.}\\ \hline


\multicolumn{2}{c}{\textbf{Excepciones}}\\ \midrule

\multicolumn{2}{Z}{\hspace{0.5cm}\textbf{4.a)} El sistema comprueba que el formato de los datos introducidos es incorrecto, el sistema informa al Administrador, el Caso de Uso continúa en el paso 3.  }\\ 

\hline
\bottomrule
\end{tabularx}
\caption{Descripción del Caso de Uso \textit{Crear alimento}.} \label{tab:caso_uso_crear_alimento}
\end{table}





























