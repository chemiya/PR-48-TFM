\section{Introducción}

En este capítulo se explica cómo se ha desarrollado el proyecto con respecto a la planificación inicial. Se define su evolución con las tareas que se han realizado en cada \textit{sprint} ya que se ha aplicado el marco de trabajo ágil Scrum. Además se exponen los problemas encontrados en los \textit{sprints}, los riesgos que han aparecido y qué mejoras se han propuesto en las \textit{Sprint Review}. 

\section{Seguimiento del proyecto}
Para cada tarea realizada en el proyecto se ha definido un código para clasificarla según el tipo del que sea:
\begin{itemize}
    \item \textbf{DOC.} Tareas relacionadas con la documentación del proyecto.
     \item \textbf{DEV.} Tareas relacionadas con el desarrollo de la aplicación del proyecto.
     \item \textbf{TEST.} Tareas relacionadas con la realización de pruebas sobre la aplicación del proyecto.
     \item \textbf{BUG.} Tareas relacionadas con la corrección de errores en la aplicación del proyecto.
     \item \textbf{REU.} Tareas relacionadas con reuniones con los tutores.
\end{itemize}

Para cada tarea dentro de cada \textit{sprint} se define su identificador único, su tipo, una descripción de en qué consiste la tarea, el tiempo empleado para completarla y su estado al finalizar el \textit{sprint}. Los estados que puede tener una tarea son:
\begin{itemize}
    \item \textbf{No iniciada.} Al finalizar el \textit{sprint}, esta tarea no se ha comenzado y por lo tanto no se ha realizado ningún trabajo de ella.
    \item \textbf{En progreso.} Al finalizar el \textit{sprint}, esta tarea se ha comenzado y se ha realizado trabajo de ella pero no se ha completado.
    \item \textbf{Completada.} Al finalizar el \textit{sprint}, esta tarea se ha finalizado.
\end{itemize}



\subsection{Sprint 1: 30/1/2023 - 16/2/2023}
Al comenzar el \textit{sprint} 1 se ha realizado un análisis de las posibles opciones que hay para este proyecto. Se han valorado las diferentes ideas propuestas y se ha estudiado su viabilidad y posibles problemas a enfrentar. Se ha llevado a cabo una reunión con los tutores para comentar estas ideas y escoger la idea que se desarrollaría en el proyecto. Una vez escogida la idea, se han clarificado los detalles del proyecto y se han concretado los objetivos que se pretenden lograr con este proyecto. A continuación se han empezado a realizar las primeras partes de la documentación. Se ha realizado una primera versión de los capítulos de introducción, planificación, requisitos, análisis y diseño. 

Se han priorizado las secciones de la documentación sobre los Casos de Uso, Modelo de Dominio, Diseño de la Base de Datos y Diseño de la Interfaz Gráfica debido a que con estas secciones realizadas se podría empezar en el siguiente \textit{sprint} a desarrollar la aplicación. 

Este \textit{sprint} ha tenido una duración de 80 horas.


En las Tablas \ref{table:sprint1A} y \ref{table:sprint1B} se pueden ver las tareas que se han realizado en este \textit{sprint}.




\begin{table}[]
  \centering
\begin{tabular}{
  |p{\dimexpr.18\linewidth-2\tabcolsep-1.3333\arrayrulewidth}% column 1
  |p{\dimexpr.09\linewidth-2\tabcolsep-1.3333\arrayrulewidth}% column 2
  |p{\dimexpr.42\linewidth-2\tabcolsep-1.3333\arrayrulewidth}
  |p{\dimexpr.15\linewidth-2\tabcolsep-1.3333\arrayrulewidth}
  |p{\dimexpr.16\linewidth-2\tabcolsep-1.3333\arrayrulewidth}
  |% column 3
  }
  \hline
  \textbf{Identificador} & \textbf{Tipo} & \textbf{Descripción}                                             & \textbf{Tiempo Empleado} & \textbf{Estado} \\ \hline
  T001                   & DOC           & Búsqueda y análisis de ideas para el proyecto.                   & 6h                       & Completada      \\ \hline

  T002                   & REU           & Reunión de arranque con los tutores.                             & 2h                       & Completada      \\ \hline


  T003                   & DOC           & Documentar resumen del proyecto.                                  & 2h                       & Completada      \\ \hline

  

  T004                      & DOC           & Documentar contexto en el capítulo de introducción.               & 2h                     &    Completada             \\ \hline

  T005                      & DOC           & Documentar objetivos en el capítulo de introducción.              & 2h                     &   Completada              \\ \hline

  T006                      & DOC           & Documentar motivación en el capítulo de introducción.             & 1h                       &   Completada              \\ \hline

  T007                      & DOC           & Documentar aplicaciones similares en el capítulo de introducción. & 1h                       &   Completada              \\ \hline

  T007                      & DOC           & Documentar estructura de la memoria en el capítulo de introducción. & 1h                       &   Completada              \\ \hline










  T008                      & DOC           & Documentar introducción en el capítulo de planificación.                                                               &     1h                     &    Completada             \\ \hline

  T009                    &    DOC            & Documentar Scrum en el capítulo de planificación.                                                                &     2h                     &     Completada            \\ \hline

  T010                      &    DOC            & Documentar adaptación de Scrum al proyecto en el capítulo de planificación.                                                                &      1h                    &     Completada            \\ \hline

  T011                       &    DOC            & Documentar análisis de riesgos en el capítulo de planificación.                                                                &          2h                &      Completada           \\ \hline

  T012                       &   DOC             & Documentar presupuesto en el capítulo de planificación.                                                                & 2h                        &         Completada        \\ \hline

  T013                      &    DOC            & Documentar \textit{Product Backlog} inicial en el capítulo de planificación.                                                                &         2h                 &     Completada            \\ \hline

 








  T014                     &     DOC           & Documentar introducción en el capítulo de requisitos.                                                                &       1h                   & Completada               \\ \hline

  T015                      &    DOC            & Desarrollar y documentar Requisitos Funcionales, No Funcionales y de Información en el capítulo de requisitos.                                                                &        4h                  &              Completada   \\ \hline

  T016                      &    DOC            & Desarrollar y documentar Actores Principales en el capítulo de requisitos.                                                               &        2h                  &               Completada  \\ \hline

  T017                      &    DOC            & Desarrollar y documentar Diagrama de Casos de Uso en el capítulo de requisitos.                                                                &         3h                 &              Completada   \\ \hline

  



 


 
\end{tabular}
\caption{Tareas de \textit{sprint} 1.}
\label{table:sprint1A}
\end{table}


\begin{table}[]
  \centering
\begin{tabular}{
  |p{\dimexpr.18\linewidth-2\tabcolsep-1.3333\arrayrulewidth}% column 1
  |p{\dimexpr.09\linewidth-2\tabcolsep-1.3333\arrayrulewidth}% column 2
  |p{\dimexpr.42\linewidth-2\tabcolsep-1.3333\arrayrulewidth}
  |p{\dimexpr.15\linewidth-2\tabcolsep-1.3333\arrayrulewidth}
  |p{\dimexpr.16\linewidth-2\tabcolsep-1.3333\arrayrulewidth}
  |% column 3
  }
  \hline
  \textbf{Identificador} & \textbf{Tipo} & \textbf{Descripción}                                             & \textbf{Tiempo Empleado} & \textbf{Estado} \\ \hline
 
  T018                   & DOC           & Desarrollar y documentar Descripción de los Casos de Uso en el capítulo de requisitos.                   & 6h                       & Completada      \\ \hline

  T019                      &    DOC            & Documentar introducción en el capítulo de análisis.                                                                 &         1h                 &          Completada       \\ \hline

  T020                      &    DOC            & Desarrollar y documentar Modelo de Dominio en el capítulo de análisis.                                                                &         5h                 &         Completada        \\ \hline

  T021                      &    DOC            & Desarrollar y documentar Modelo de Análisis en el capítulo de análisis.                                                                &         5h                 &         Completada        \\ \hline

  T022                      &    DOC            & Desarrollar y documentar Realización de Casos de Uso de Análisis en el capítulo de análisis.                                                                &         5h                 &         Completada        \\ \hline


  T023                      &    DOC            & Documentar introducción en el capítulo de diseño.                                                                &        1h                  &           Completada      \\ \hline

  T024                      &    DOC            & Desarrollar y documentar Arquitectura Lógica del Sistema en el capítulo de diseño.                                                                &          2h                &            Completada     \\ \hline

  T025                      &    DOC            & Desarrollar y documentar Arquitectura del Cliente y del Servidor en el capítulo de diseño.                                                                &        2h                  &            Completada     \\ \hline






  T026                      &    DOC            & Desarrollar y documentar Patrones de Diseño en el capítulo de diseño.                                                                &         3h                 &          Completada       \\ \hline

  T027                      &    DOC            & Desarrollar y documentar Arquitectura Física del Sistema en el capítulo de diseño.                                                                &          2h                &            Completada     \\ \hline

  T028                      &    DOC            & Desarrollar y documentar Diseño de la Base de Datos en el capítulo de diseño.                                                                &        4h                  &            Completada     \\ \hline

  T029                      &    DOC            & Desarrollar y documentar Diseño de la Interfaz Gráfica en el capítulo de diseño.                                                                &        7h                  &           Completada      \\ \hline



 
\end{tabular}
\caption{Tareas de \textit{sprint} 1.}
\label{table:sprint1B}
\end{table}






    


   















\subsection{Sprint 2: 16/2/2023 - 5/3/2023}
Durante el \textit{sprint} 2 se ha dedicado más tiempo a avanzar en el desarrollo de las interfaces de la aplicación. Se han desarrollado las principales funcionalidades de la aplicación en base a la primera versión de los capítulos realizados de la documentación en el \textit{sprint} previo. Con estas secciones iniciales de la documentación donde se explicaba de manera general la aplicación, se ha podido empezar a desarrollar muchas de las interfaces de la aplicación.

Para el desarrollo de identificación y registro han aparecido numerosos problemas con la utilización de JWT para que las contraseñas de los usuarios se almacenasen de manera segura. Entre estos problemas cabe destacar que cuando un Usuario se registraba en la aplicación, la contraseña se encriptaba de una manera distinta cada vez. Esto se ha resuelto definiendo una clave secreta para que la contraseña se encriptase siempre siguiendo un patrón establecido para que después se pudiese recuperar. 

Para las interfaces donde se debían cargar imágenes de un elemento como un alimento, una receta, un Usuario o una publicación se han encontrado varios problemas para almacenar estas imágenes en Cloudinary. Para conseguir esta funcionalidad se han realizado varias pruebas y se han probado numerosas opciones. Finalmente se ha conseguido almacenar la imagen en Cloudinary enviando la imagen desde el \textit{frontend} al \textit{backend} y desde aquí enviando una petición POST para almacenar la imagen.

En la \textit{Sprint Review} se ha realizado una demostración del incremento realizado durante el \textit{sprint} y se han propuesto numerosas mejoras sobre la aplicación como mostrar todos los usuarios, alimentos o recetas en los buscadores al inicio cuando el Usuario no haya introducido ningún valor para buscar. 

En este \textit{sprint} ha aparecido el riesgo \textit{R-01 Falta de cualificación con las tecnologías utilizadas} ya que se desconocía como se iban a almacenar las fotos cargadas en la aplicación en la nube mediante Cloudinary. Se ha aplicado un plan de contingencia que ha consistido en consultar en internet mediante vídeos y foros como almacenar y recuperar fotos en Cloudinary.

Este \textit{sprint} ha tenido una duración de 80 horas.

En la Tabla \ref{table:sprint2} se pueden ver las tareas que se han realizado en este \textit{sprint}.


\begin{table}[]
    \centering
\begin{tabular}{
    |p{\dimexpr.18\linewidth-2\tabcolsep-1.3333\arrayrulewidth}% column 1
    |p{\dimexpr.09\linewidth-2\tabcolsep-1.3333\arrayrulewidth}% column 2
    |p{\dimexpr.42\linewidth-2\tabcolsep-1.3333\arrayrulewidth}
    |p{\dimexpr.15\linewidth-2\tabcolsep-1.3333\arrayrulewidth}
    |p{\dimexpr.16\linewidth-2\tabcolsep-1.3333\arrayrulewidth}
    |% column 3
    }
    \hline
    \textbf{Identificador} & \textbf{Tipo} & \textbf{Descripción}                                             & \textbf{Tiempo Empleado} & \textbf{Estado} \\ \hline
    


    T030                      &    DEV            & Desarrollo de la interfaz de identificación.                                                               &        3h                  &           Completada      \\ \hline

    T031                      &    DEV            & Desarrollo de la interfaz de registro.                                                                &        3h                  &           Completada      \\ \hline

    T032                   & DEV           & Desarrollo de la interfaz de descripción de la aplicación.                   & 4h                       & Completada      \\ \hline

    T033                   & DEV           & Desarrollo de la interfaz de muro de publicaciones.                   & 3h                       & Completada      \\ \hline




    T034                   & DEV           & Desarrollo de la interfaz de perfil de usuario.                   & 3h                       & Completada      \\ \hline

    T035                   & DEV           & Desarrollo de la interfaz de editar usuario.                   & 4h                       & Completada      \\ \hline


    T036                   & DEV           & Desarrollo de la interfaz de seguidores del usuario.                   & 1h                       & Completada      \\ \hline

    T037                   & DEV           & Desarrollo de la interfaz de seguidos del usuario.                  & 1h                       & Completada      \\ \hline

    T038                   & DEV           & Desarrollo de la interfaz de recetas favoritas.                   & 1h                       & Completada      \\ \hline





    T039                   & DEV           & Desarrollo de la interfaz de detalles de una receta.                   & 4h                       & Completada      \\ \hline

    T040                   & DEV           & Desarrollo de la interfaz de detalles de un usuario.                   & 4h                       & Completada      \\ \hline

    T041                   & DEV           & Desarrollo de la interfaz de detalles de una publicación.                   & 4h                       & Completada      \\ \hline

    T042                   & DEV           & Desarrollo de la interfaz de detalles de un alimento.                   & 4h                       & Completada      \\ \hline

    T043                   & DEV           & Desarrollo de la interfaz de crear receta.                  & 6h                       & Completada      \\ \hline

    T044                   & DEV           & Desarrollo de la interfaz de crear publicación.                   & 5h                       & Completada      \\ \hline


    

    T045                   & DEV           & Desarrollo de la interfaz de crear alimento.                  & 5h                       & Completada      \\ \hline

    T046                   & DEV           & Desarrollo de la interfaz de buscar receta.                  & 5h                       & Completada      \\ \hline

    T047                   & DEV           & Desarrollo de la interfaz de buscar usuario.                  & 5h                       & Completada      \\ \hline

    T048                   & DEV           & Desarrollo de la interfaz de buscar alimento.                  & 5h                       & Completada      \\ \hline

    T049                   & DEV           & Desarrollo de la interfaz de vista de administrador.                  & 10h                       & Completada      \\ \hline




   
  \end{tabular}
  \caption{Tareas de \textit{sprint} 2.}
\label{table:sprint2}
\end{table}

















\subsection{Sprint 3: 5/3/2023 - 22/3/2023}
En este \textit{sprint} se ha corregido el resumen, el capítulo de introducción y el capítulo de requisitos que son de los más importantes en la memoria. Respecto a las correcciones del capítulo de requisitos, se han modificado los Requisitos Funcionales, No Funcionales y de Información definidos en el \textit{sprint} 1 para definirlos de una manera más clara y que se ajusten a lo que se va a desarrollar en la aplicación. De manera similar, el Diagrama de Casos de Uso y la Descripción de los Casos de Uso también se han corregido para exponer más claramente la funcionalidad de la aplicación. El hecho de haber desarrollado en el \textit{sprint} 2 las interfaces ha ayudado a refinar el capítulo de requisitos.

Se ha documentado el capítulo de tecnologías utilizadas debido a que ya se tienen claras las tecnologías que se van a utilizar en el proyecto. Se han corregido las interfaces de la aplicación de identificación, registro y descripción de la aplicación.

Para la interfaz de descripción de la aplicación se han corregido los textos que se mostraban, ya que está vista funciona como una \textit{landing page} para atraer a los usuarios. Se han definido textos que describen más claramente las ventajas de la aplicación y se han organizado las secciones de esta interfaz por orden de importancia. 

Para la interfaz de registro, se ha corregido la validación del formulario, ya que no pueden existir dos usuarios con el mismo \textit{email} o con el mismo \textit{username}. Para conseguirlo, cada vez que se pulsa una tecla en el campo correspondiente se realiza una comprobación para ver si existe un Usuario con ese \textit{email} o \textit{username} introducido. 

Este \textit{sprint} ha tenido una duración de 30 horas.

En la Tabla \ref{table:sprint3} se pueden ver las tareas que se han realizado en este \textit{sprint}.

\begin{table}[]
  \centering
\begin{tabular}{
  |p{\dimexpr.18\linewidth-2\tabcolsep-1.3333\arrayrulewidth}% column 1
  |p{\dimexpr.09\linewidth-2\tabcolsep-1.3333\arrayrulewidth}% column 2
  |p{\dimexpr.42\linewidth-2\tabcolsep-1.3333\arrayrulewidth}
  |p{\dimexpr.15\linewidth-2\tabcolsep-1.3333\arrayrulewidth}
  |p{\dimexpr.16\linewidth-2\tabcolsep-1.3333\arrayrulewidth}
  |% column 3
  }
  \hline
  \textbf{Identificador} & \textbf{Tipo} & \textbf{Descripción}                                             & \textbf{Tiempo Empleado} & \textbf{Estado} \\ \hline

  T050                   & REU           & Reunión de seguimiento con los tutores.                             & 1h                       & Completada      \\ \hline
 
  T051                   & DOC          & Corregir errores del resumen.                   & 2h                       & Completada      \\ \hline

  T052                   & DOC           & Corregir errores contexto, objetivos, motivación, aplicaciones similares y estructura de la memoria del capítulo introducción.                  & 4h                       & Completada      \\ \hline

  T053                   & DOC           & Corregir errores introducción, Requisitos Funcionales, No Funcionales y de Información, Actores Principales, Diagrama de Casos de Uso y Descripción de los Casos de Uso del capítulo de requisitos.                  & 5h                       & Completada      \\ \hline

  T054                   & DOC           & Documentar introducción en el capítulo de tecnologías utilizadas.            & 1h                       & Completada      \\ \hline

  T055                   & DOC           & Documentar base de datos en el capítulo de tecnologías utilizadas.            & 3h                       & Completada      \\ \hline

  T056                   & DOC           & Documentar \textit{frontend} en el capítulo de tecnologías utilizadas.            & 2h                       & Completada      \\ \hline

  T057                   & DOC           & Documentar \textit{backend} en el capítulo de tecnologías utilizadas.            & 2h                       & Completada      \\ \hline

  T058                   & DOC           & Documentar tecnologías utilizadas en el desarrollo del proyecto en el capítulo de tecnologías utilizadas.            & 3h                       & Completada      \\ \hline





  T059                      &    BUG            & Correcciones de la interfaz de identificación.                                                                &        1.75h                  &           Completada      \\ \hline

    T060                      &    BUG           & Correcciones de la interfaz de registro.                                                                &        1.75h                  &           Completada      \\ \hline

    T061                   & BUG           & Correcciones de la interfaz de descripción de la aplicación.                   & 2.75h                       & Completada      \\ \hline





    T062                      &    TEST            & Pruebas de la interfaz de identificación.                                                                &        0.25h                  &           Completada      \\ \hline

    T063                      &    TEST           & Pruebas de la interfaz de registro.                                                                &        0.25h                  &           Completada      \\ \hline

    T064                   & TEST           & Pruebas de la interfaz de descripción de la aplicación.                   & 0.25h                       & Completada      \\ \hline
 
\end{tabular}
\caption{Tareas de \textit{sprint} 3.}
\label{table:sprint3}
\end{table}















\subsection{Sprint 4: 22/3/2023 - 8/4/2023}
Este \textit{sprint} se ha dedicado a corregir los primeros capítulos de la memoria entre los que se encuentran la planificación y análisis. En el capítulo de análisis se han realizado correcciones sobre el Modelo de Dominio, principalmente añadiendo nuevos atributos a las diferentes clases y creando ciertas relaciones entre clases que no existían y que al desarrollar las interfaces se han descubierto.

Además se ha documentado el capítulo de implementación y pruebas debido a que se han realizado grandes correcciones en la interfaces de la aplicación y por lo tanto se deben ir realizando pruebas de las interfaces completas.


Para la interfaz de editar los datos del usuario, se ha encontrado el problema de que si no se editaba ningún campo el Usuario también podía pulsar de editar y por lo tanto no se actualizaba ningún campo. Para solucionar esto se ha realizado una validación de qué campos del formulario había editado el Usuario de manera que cuando el Usuario editase algún campo se activase el botón de guardar los cambios.

En la \textit{Sprint Review} se ha realizado una demostración del incremento realizado durante el \textit{sprint} y se han propuesto numerosas mejoras sobre la aplicación como dejar la barra de navegación fija, hacer más visibles los botones para crear una publicación o una receta y dejar márgenes mayores en las interfaces para que las vistas se ajusten mejor a los dispositivos móviles.

En este \textit{sprint} ha aparecido el riesgo \textit{R-03 Desarrollo de la interfaz de la aplicación incorrecta} debido a que al realizar pruebas con potenciales usuarios de la aplicación, se ha detectado que no comprendían correctamente las vistas de la aplicación. Se ha aplicado un plan de contingencia que ha consistido en modificar las interfaces para que los usuarios las comprendiesen mejor por lo que se ha simplificado y se ha mejorado la visibilidad de los elementos más importantes.

Este \textit{sprint} ha tenido una duración de 30 horas.

En las Tablas \ref{table:sprint4A} y \ref{table:sprint4B} se pueden ver las tareas que se han realizado en este \textit{sprint}.

\begin{table}[]
  \centering
\begin{tabular}{
  |p{\dimexpr.18\linewidth-2\tabcolsep-1.3333\arrayrulewidth}% column 1
  |p{\dimexpr.09\linewidth-2\tabcolsep-1.3333\arrayrulewidth}% column 2
  |p{\dimexpr.42\linewidth-2\tabcolsep-1.3333\arrayrulewidth}
  |p{\dimexpr.15\linewidth-2\tabcolsep-1.3333\arrayrulewidth}
  |p{\dimexpr.16\linewidth-2\tabcolsep-1.3333\arrayrulewidth}
  |% column 3
  }
  \hline
  \textbf{Identificador} & \textbf{Tipo} & \textbf{Descripción}                                             & \textbf{Tiempo Empleado} & \textbf{Estado} \\ \hline
  T065                   & REU           & Reunión de seguimiento con los tutores.                             & 2h                       & Completada      \\ \hline

  T066                   & DOC           & Corregir errores introducción, Scrum, adaptación de Scrum al proyecto, análisis de riesgos, presupuesto y \textit{Product Backlog} inicial del capítulo de planificación.                  & 3h                       & Completada      \\ \hline

  T067                   & DOC           & Corregir errores introducción, Modelo de Dominio, Modelo de Análisis y Realización de Casos de Uso de Análisis del capítulo de análisis.                  & 5h                       & Completada      \\ \hline

  
  T068                   & DOC           & Documentar introducción en el capítulo de implementación y pruebas.            & 1h                       & Completada      \\ \hline

  T069                   & DOC           & Documentar estructura del código de \textit{frontend} en el capítulo de implementación y pruebas.            & 2h                       & Completada      \\ \hline

  T070                   & DOC           & Documentar estructura del código de \textit{backend} en el capítulo de implementación y pruebas.            & 2h                       & Completada      \\ \hline

  T071                   & DOC           & Documentar pruebas en el capítulo de implementación y pruebas.            & 2h                       & Completada      \\ \hline

  T072                   & DOC           & Documentar problemas y dificultades encontradas en el capítulo de implementación y pruebas.            & 2h                       & Completada      \\ \hline

  T073                   & DOC           & Desarrollar y documentar Casos de Prueba en el capítulo de implementación y pruebas.            & 3h                       & Completada      \\ \hline







  T074                   & BUG           & Correcciones de la interfaz de muro de publicaciones.                   & 0.75h                       & Completada      \\ \hline

  T075                   & BUG           & Correcciones de la interfaz de perfil de usuario.                   & 1.75h                       & Completada      \\ \hline

  T076                   & BUG           & Correcciones de la interfaz de editar usuario.                   & 1.75h                       & Completada      \\ \hline


  T077                   & BUG           & Correcciones de la interfaz de seguidores del usuario.                   & 0.75h                       & Completada      \\ \hline

  T078                   & BUG           & Correcciones de la interfaz de seguidos del usuario.                   & 0.75h                       & Completada      \\ \hline

  T079                   & BUG           & Correcciones de la interfaz de recetas favoritas.                   & 0.75h                       & Completada      \\ \hline






 
 

\end{tabular}
\caption{Tareas de \textit{sprint} 4.}
\label{table:sprint4A}
\end{table}



\begin{table}[]
  \centering
\begin{tabular}{
  |p{\dimexpr.18\linewidth-2\tabcolsep-1.3333\arrayrulewidth}% column 1
  |p{\dimexpr.09\linewidth-2\tabcolsep-1.3333\arrayrulewidth}% column 2
  |p{\dimexpr.42\linewidth-2\tabcolsep-1.3333\arrayrulewidth}
  |p{\dimexpr.15\linewidth-2\tabcolsep-1.3333\arrayrulewidth}
  |p{\dimexpr.16\linewidth-2\tabcolsep-1.3333\arrayrulewidth}
  |% column 3
  }
  \hline
  \textbf{Identificador} & \textbf{Tipo} & \textbf{Descripción}                                             & \textbf{Tiempo Empleado} & \textbf{Estado} \\ \hline
 





  T080                   & TEST           & Pruebas de la interfaz de muro de publicaciones.                   & 0.25h                       & Completada      \\ \hline

  T081                   & TEST           & Pruebas de la interfaz de perfil de usuario.                   & 0.25h                       & Completada      \\ \hline

  T082                   & TEST           & Pruebas de la interfaz de editar usuario.                   & 0.25h                       & Completada      \\ \hline


  T083                   & TEST           & Pruebas de la interfaz de seguidores del usuario.                   & 0.25h                       & Completada      \\ \hline

  T084                   & TEST           & Pruebas de la interfaz de seguidos del usuario.                   & 0.25h                       & Completada      \\ \hline

  T085                   & TEST           & Pruebas de la interfaz de recetas favoritas.                   & 0.25h                       & Completada      \\ \hline
 
 

\end{tabular}
\caption{Tareas de \textit{sprint} 4.}
\label{table:sprint4B}
\end{table}















\subsection{Sprint 5: 8/4/2023 - 25/4/2023}
Este \textit{sprint} se han corregido los capítulos de diseño y tecnologías utilizadas. En el capítulo de diseño principalmente se han realizado correcciones sobre el Diseño de la Base de Datos para que el Diseño Lógico de la Base de Datos se ajuste a las correcciones introducidas en Modelo de Dominio. La Arquitectura Lógica del Sistema y los Patrones de Diseño se han corregido en este capítulo para que reflejen el estado actual de la aplicación desarrollada.

Además se han corregido las vistas de la aplicación que más funcionalidad tienen y por tanto requerían más trabajo. Estas vistas son las que muestran los detalles de un elemento como una receta, publicación, alimento o Usuario y las vistas que se usan para crear elementos como los alimentos, recetas o publicaciones.

Para la vista de creación de una receta se han corregido numerosos problemas como la validación de que no se añadiese el mismo alimento dos veces a los alimentos necesarios para su preparación. Esto se pudo solucionar recorriendo los alimentos ya seleccionados y comprobando que el alimento nuevo a añadir no se encontraba entre ellos. También se han encontrado problemas en cómo se mostraban los pasos registrados en la receta en pantallas de dispositivos móviles. Esto supuso un gran contratiempo ya que las correcciones realizadas no arreglaban el problema. Se ha optado por cambiar ligeramente la interfaz para que los pasos registrados se mostrasen de manera ordenada también en dispositivos móviles.

En este \textit{sprint} ha aparecido el riesgo \textit{R-12 Aparición de bugs con complicada solución} debido que como se ha comentado se han encontrado problemas a la hora de mostrar los pasos registrados de una receta en pantallas de dispositivos móviles. Se ha aplicado un plan de contingencia que ha consistido en buscar soluciones alternativas para este problema, de manera que se ha cambiado ligeramente la interfaz para que los pasos se mostraran correctamente. 

Este \textit{sprint} ha tenido una duración de 30 horas.

En la Tabla \ref{table:sprint5} se pueden ver las tareas que se han realizado en este \textit{sprint}.

\begin{table}[]
  \centering
\begin{tabular}{
  |p{\dimexpr.18\linewidth-2\tabcolsep-1.3333\arrayrulewidth}% column 1
  |p{\dimexpr.09\linewidth-2\tabcolsep-1.3333\arrayrulewidth}% column 2
  |p{\dimexpr.42\linewidth-2\tabcolsep-1.3333\arrayrulewidth}
  |p{\dimexpr.15\linewidth-2\tabcolsep-1.3333\arrayrulewidth}
  |p{\dimexpr.16\linewidth-2\tabcolsep-1.3333\arrayrulewidth}
  |% column 3
  }
  \hline
  \textbf{Identificador} & \textbf{Tipo} & \textbf{Descripción}                                             & \textbf{Tiempo Empleado} & \textbf{Estado} \\ \hline
  T086                   & REU           & Reunión de seguimiento con los tutores.                             & 1h                       & Completada      \\ \hline

  T087                   & DOC           & Corregir introducción, Arquitectura Lógica del Sistema, patrones de diseño, Arquitectura Física del Sistema, Diseño de la Base de Datos y Diseño de la Interfaz Gráfica del capítulo de diseño.                 & 4h                       & Completada      \\ \hline

  T088                   & DOC          & Corregir errores de la introducción, base de datos, \textit{frontend}, \textit{backend} y desarrollo del proyecto del capítulo tecnologías utilizadas.               & 3h                       & Completada      \\ \hline







  T089                   & BUG           & Correcciones de la interfaz de detalles de la receta.                   & 2.75h                       & Completada      \\ \hline

  T090                   & BUG           & Correcciones de la interfaz de detalles del usuario.                   & 2.75h                       & Completada      \\ \hline

  T091                   & BUG           & Correcciones de la interfaz de detalles de la publicación.                   & 1.75h                       & Completada      \\ \hline

  T092                   & BUG           & Correcciones de la interfaz de detalles del alimento.                   & 1.75h                       & Completada      \\ \hline


  T093                   & BUG           & Correcciones de la interfaz de crear receta.                  & 3.75h                       & Completada      \\ \hline

  T094                   & BUG           & Correcciones de la interfaz de crear publicación.                   & 3.75h                       & Completada      \\ \hline

  T095                   & BUG           & Correcciones de la interfaz de crear alimento.                  & 3.75h                       & Completada      \\ \hline






  T096                   & TEST           & Pruebas de la interfaz de detalles de la receta.                   & 0.25h                       & Completada      \\ \hline

  T097                   & TEST           & Pruebas de la interfaz de detalles del usuario.                   & 0.25h                       & Completada      \\ \hline

  T098                   & TEST           & Pruebas de la interfaz de detalles de la publicación.                   & 0.25h                       & Completada      \\ \hline

  T099                   & TEST           & Pruebas de la interfaz de detalles del alimento.                   & 0.25h                       & Completada      \\ \hline


  T100                   & TEST           & Pruebas de la interfaz de crear receta.                  & 0.25h                       & Completada      \\ \hline

  T101                   & TEST           & Pruebas de la interfaz de crear publicación.                   & 0.25h                       & Completada      \\ \hline

  T102                   & TEST           & Pruebas de la interfaz de crear alimento.                  & 0.25h                       & Completada      \\ \hline
  

\end{tabular}
\caption{Tareas de \textit{sprint} 5.}
\label{table:sprint5}
\end{table}














\subsection{Sprint 6: 25/4/2023 - 12/5/2023}
En este \textit{sprint} se han realizado los últimos capítulos de la documentación que son el de seguimiento y conclusiones. Además se ha realizado el apéndice de manuales y se han realizado las últimas correcciones en las interfaces de la aplicación dedicadas a la búsqueda de elementos como recetas, alimentos o usuarios. Además, se ha corregido la vista de Administrador.

Para la interfaz de vista de Administrador se han corregido varios detalles en las tablas que muestran los usuarios, alimentos, recetas y publicaciones almacenadas en la aplicación. Cabe destacar que para la tabla que mostraba las publicaciones almacenadas en la aplicación no se mostraba el nombre del Usuario que la había realizado. Este error se debía a que el \textit{frontend} no sabía interpretar correctamente los datos que le mandaba el \textit{backend} y por lo tanto había campos que no se almacenaban. Esto se ha corregido estableciendo que tanto el \textit{frontend} como el \textit{backend} llamasen cada dato que se comunicaban del mismo modo.

Este \textit{sprint} ha tenido una duración de 30 horas.

En la Tabla \ref{table:sprint6} se pueden ver las tareas que se han realizado en este \textit{sprint}.

\begin{table}[]
  \centering
\begin{tabular}{
  |p{\dimexpr.18\linewidth-2\tabcolsep-1.3333\arrayrulewidth}% column 1
  |p{\dimexpr.09\linewidth-2\tabcolsep-1.3333\arrayrulewidth}% column 2
  |p{\dimexpr.42\linewidth-2\tabcolsep-1.3333\arrayrulewidth}
  |p{\dimexpr.15\linewidth-2\tabcolsep-1.3333\arrayrulewidth}
  |p{\dimexpr.16\linewidth-2\tabcolsep-1.3333\arrayrulewidth}
  |% column 3
  }
  \hline
  \textbf{Identificador} & \textbf{Tipo} & \textbf{Descripción}                                             & \textbf{Tiempo Empleado} & \textbf{Estado} \\ \hline

  T103                   & DOC          & Corregir errores de la introducción, estructura del código de \textit{frontend}, estructura del código de \textit{backend}, pruebas, problemas y dificultades encontradas y Casos de Prueba del capítulo implementación y pruebas.                & 3h                       & Completada      \\ \hline

  T104                   & REU           & Reunión de seguimiento con los tutores.                             & 1h                       & Completada      \\ \hline

  T105                   & DOC           & Documentar el trabajo realizado en los \textit{sprints} en el capítulo de seguimiento.            & 7h                       & Completada      \\ \hline

 

  T106                   & DOC           & Documentar conclusiones en el capítulo de conclusiones.           & 3h                       & Completada      \\ \hline
  T107                   & DOC           & Documentar líneas de trabajo futuras en el capítulo de conclusiones.           & 1h                       & Completada      \\ \hline



  T108                   & DOC           & Documentar el manual de despliegue en el apéndice de manuales.        & 2h                       & Completada      \\ \hline

  T109                   & DOC           & Documentar el manual de Usuario en el apéndice de manuales.        & 3h                       & Completada      \\ \hline






  T110                   & BUG           & Correcciones de la interfaz de buscar receta.                  & 1.75h                       & Completada      \\ \hline

  T111                   & BUG           & Correcciones de la interfaz de buscar usuario.                  & 1.75h                       & Completada      \\ \hline

  T112                   & BUG           & Correcciones de la interfaz de buscar alimento.                  & 1.75h                       & Completada      \\ \hline



  T113                   & BUG           & Correcciones de la interfaz de vista de administrador                  & 2.75h                       & Completada      \\ \hline




  T114                   & TEST           & Pruebas de la interfaz de buscar receta.                  & 0.25h                       & Completada      \\ \hline

  T115                   & TEST           & Pruebas de la interfaz de buscar usuario.                  & 0.25h                       & Completada      \\ \hline

  T116                   & TEST           & Pruebas de la interfaz de buscar alimento.                  & 0.25h                       & Completada      \\ \hline



  T117                   & TEST           & Pruebas de la interfaz de vista de administrador.                  & 0.25h                       & Completada      \\ \hline
  

\end{tabular}
\caption{Tareas de \textit{sprint} 6.}
\label{table:sprint6}
\end{table}











\subsection{Sprint 7: 12/5/2023 - 29/5/2023}
En este \textit{sprint} se han revisado los capítulos de seguimiento y conclusiones y el apéndice de manuales. Se han corregido los errores encontrados y se han actualizado estos capítulos según las últimas correcciones en la aplicación.

En este \textit{sprint} ha aparecido el riesgo \textit{R-10 Falta de tiempo para trabajar en el proyecto} debido que en estas fechas se realizaban los exámenes de las asignaturas. Se ha aplicado un plan de contingencia que ha consistido en posponer al siguiente \textit{sprint} las tareas que no se han podido realizar en este \textit{sprint}.

Este \textit{sprint} tenía una duración planificada de 10 horas pero finalmente ha tenido una duración de 6 horas ya que no se han podido realizar todas las tareas.

En la Tabla \ref{table:sprint7}  se pueden ver las tareas que se han realizado en este \textit{sprint}.
\begin{table}[]
  \centering
\begin{tabular}{
  |p{\dimexpr.18\linewidth-2\tabcolsep-1.3333\arrayrulewidth}% column 1
  |p{\dimexpr.09\linewidth-2\tabcolsep-1.3333\arrayrulewidth}% column 2
  |p{\dimexpr.42\linewidth-2\tabcolsep-1.3333\arrayrulewidth}
  |p{\dimexpr.15\linewidth-2\tabcolsep-1.3333\arrayrulewidth}
  |p{\dimexpr.16\linewidth-2\tabcolsep-1.3333\arrayrulewidth}
  |% column 3
  }
  \hline
  \textbf{Identificador} & \textbf{Tipo} & \textbf{Descripción}                                             & \textbf{Tiempo Empleado} & \textbf{Estado} \\ \hline

  T118                   & DOC           & Corregir el trabajo realizado en los \textit{sprints} en el capítulo de seguimiento.            & 4h                       & Completada      \\ \hline

  T119                   & DOC           & Corregir conclusiones y líneas de trabajo futuras en el capítulo de conclusiones.           & 0h                       & No Iniciada     \\ \hline

  T120                   & DOC           & Corregir el manual de Usuario y de despliegue en el apéndice de manuales.        & 2h                       & En Progreso      \\ \hline
 

 
\end{tabular}
\caption{Tareas de \textit{sprint} 7.}
\label{table:sprint7}
\end{table}


















\subsection{Sprint 8: 29/5/2023 - 15/6/2023}
En este último \textit{sprint} se ha comprobado que todo el trabajo final sea correcto. Para ello se ha revisado la memoria para comprobar que no contenga errores. Además se ha probado la aplicación a fondo para comprobar que toda la funcionalidad cumple con lo requerido, que todas las vistas funcionan correctamente y que se almacenan los datos de la manera deseada.

Este \textit{sprint} tenía una duración planificada de 10 horas pero finalmente ha tenido una duración de 14 horas ya que se han tenido que completar las tareas pendientes del \textit{sprint} anterior.

En la Tabla \ref{table:sprint8} se pueden ver las tareas que se han realizado en este \textit{sprint}.

\begin{table}[]
  \centering
\begin{tabular}{
  |p{\dimexpr.18\linewidth-2\tabcolsep-1.3333\arrayrulewidth}% column 1
  |p{\dimexpr.09\linewidth-2\tabcolsep-1.3333\arrayrulewidth}% column 2
  |p{\dimexpr.42\linewidth-2\tabcolsep-1.3333\arrayrulewidth}
  |p{\dimexpr.15\linewidth-2\tabcolsep-1.3333\arrayrulewidth}
  |p{\dimexpr.16\linewidth-2\tabcolsep-1.3333\arrayrulewidth}
  |% column 3
  }
  \hline
  \textbf{Identificador} & \textbf{Tipo} & \textbf{Descripción}                                             & \textbf{Tiempo Empleado} & \textbf{Estado} \\ \hline

  
  T119                   & DOC           & Corregir conclusiones y líneas de trabajo futuras en el capítulo de conclusiones.           & 2h                       & Completada      \\ \hline

  T120                   & DOC           & Corregir el manual de Usuario y de despliegue en el apéndice de manuales.        & 2h                       & Completada      \\ \hline


  T121                   & DOC           & Revisión final de la memoria.       & 7h                       & Completada      \\ \hline

  T122                   & TEST           & Prueba final de la aplicación.       & 3h                       & Completada      \\ \hline

\end{tabular}
\caption{Tareas de \textit{sprint} 8.}
\label{table:sprint8}
\end{table}












